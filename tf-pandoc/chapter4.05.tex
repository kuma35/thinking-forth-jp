\section{Algorithms and Data Structures}%
\index{A!Algorithms|(}%
\index{D!Detailed design!algorithms|(}

In \Chap{2} we learned how to describe a problem's requirements in terms
of interfaces and rules. In this section we'll refine the conceptual model
for each component into clearly defined algorithms and data structures.

An algorithm is a procedure, described as a finite number of rules, for
accomplishing a certain task. The rules must be unambiguous and guaranteed
to terminate after a finite number of applications. (The word is named for
the ninth century Persian mathematician \person{al-Khowarizmi}.)

An algorithm lies halfway between the imprecise directives of human
speech, such as ``Please sort these letters chronologically,'' and the
precise directives of computer language, such as ``\forth{BEGIN 2DUP < IF}
\dots{}'' etc. The algorithm for sorting letters chronologically might be
this:
\begin{enumerate}
\item Take an unsorted letter and note its date.
\item Find the correspondence folder for that month and year.
\item Flip through the letters in the folder, starting from the front, until
   you find the first letter dated later than your current letter.
\item Insert your current letter just in front of the letter dated later.
   (If the folder is empty, just insert the letter.)
\end{enumerate}
There may be several possible algorithms for the same job. The algorithm
given above would work fine for folders containing ten or fewer letters,
but for folders with a hundred letters, you'd probably resort to a more
efficient algorithm, such as this:
\begin{enumerate}
\item (same)
\item (same)
\item If the date falls within the first half of the month, open the
folder a third of the way in. If the letter you find there is dated later
than your current letter, search forward until you find a letter dated the
same or before your current letter. Insert your letter at that point. If
the letter you find is dated earlier than your current letter, search
backward\dots{}
\end{enumerate}
\dots{} You get the point. This second algorithm is more complicated than
the first. But in execution it will require fewer steps on the average
(because you don't have to search clear from the beginning of the folder
every time) and therefore can be performed faster.

\index{D!Data structures:!defined|(}
A data structure is an arrangement of data or locations for data,
organized especially to match the problem. In the last example, the file
cabinet containing folders and the folders containing individual letters
can be thought of as data structures.
\index{D!Data structures:!defined|)}%

The new conceptual model includes the filing cabinets and folders
(data structures) plus the steps for doing the filing (algorithms).
\index{A!Algorithms|)}%
\index{D!Detailed design!algorithms|)}

