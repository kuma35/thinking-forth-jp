\section{The Analysis Phase}%
\index{A!Analysis|(}%

In the remainder of this chapter we'll discuss the analysis phase.
Analysis is an organized way of understanding and documenting what
the program should do.%
\index{A!Analysis!defined}

With a simple program that you write for yourself in less than an
hour, the analysis phase may take about 250 microseconds. At the other
extreme, some projects will take many man-years to build. On such a
project, the analysis phase is critical to the success of the entire
project.

We've indicated three parts to the analysis phase:

\begin{enumerate}\parsep=0pt\itemsep=0pt
\item Discovering the requirements and constraints
\item Building a conceptual model of the solution
\item Estimating cost, scheduling, and performance
\end{enumerate}

\noindent Let's briefly describe each part:

\subsection{Discovering the Requirements}%
\index{A!Analysis!requirements|(}%
\index{R!Requirements|(}

The first step is to determine what the application should do. The
customer, or whoever wants the system, should supply a ``requirements
specification.'' This is a modest document that lists the minimum
capabilities for the finished product.

The analyst may also probe further by conducting interviews and
sending out questionnaires to the users.%
\index{A!Analysis!requirements|)}%
\index{R!Requirements|)}

\subsection{Discovering the Constraints}%
\index{A!Analysis!constraints|(}%
\index{C!Constraints|(}

The next step is to discover any limiting factors. How important is
speed? How much memory is available? How soon do you need it?

No matter how sophisticated our technology becomes, programmers will
always be bucking limitations. System capacities inexplicably
%Page 046 in first edition
diminish over time. The double-density disk drives that once were the
answer to my storage prayers no longer fill the bill. The
double-sided, double-density drives I'll get next will seem like a
vast frontier---for a while. I've heard guys with 10-megabyte hard
disks complain of feeling cramped.

Whenever there's a shortage of something---and there always will
be---tradeoffs have to be made. It's best to use the analysis phase to
anticipate most limitations and decide which tradeoffs to make.

On the other hand, you should \emph{not} consider other types of
constraints during analysis, but should instead impose them gradually
during implementation, the way one stirs flour into gravy.

The type of constraint to consider during analysis includes those that
might affect the overall approach. The type to defer includes those
that can be handled by making iterative refinements to the planned
software design.

As we heard in our earlier interviews, finding out about \emph{hardware}
constraints\index{H!Hardware constraints} often requires writing
some test code and trying things out.

Finding out about the \emph{customer's} constraints%
\index{C!Customer constraints}
is usually a matter of asking the customer, or of taking written
surveys. ``How fast do you need such-and-such, on a scale of one to
ten?'', etc.%
\index{A!Analysis!constraints|)}%
\index{C!Constraints|)}

\subsection{Building a Conceptual Model of the Solution}%
\index{A!Analysis!conceptual model|(}%
\index{C!Conceptual model|(}

A conceptual model is an imaginary solution to the problem. It is a
view of how the system \emph{appears} to work. It is an answer to all
the requirements and constraints.%
\index{C!Conceptual model!defined}

\wepsfigp{img2-047}{Refining the conceptual model to meet
requirements and constraints.}

If the requirements definition is for ``something to stand on to paint
the ceiling,'' then a description of the conceptual model is ``a
device that is free-standing (so you can paint the center of the
room), with several steps spaced at convenient intervals (so you can
climb up and down), and having a small shelf near the top (to hold
your paint can).''

A conceptual model is not quite a design, however. A design begins to
describe how the system \emph{really} works. In design, the image of a
step ladder would begin to emerge.

\Forth{} blurs the distinction a little, because all definitions are
written in conceptual terms, using the lexicons of lower level
components. In fact, later in this chapter we'll use \Forth{}
``pseudocode'' to describe conceptual model solutions.

Nevertheless, it's useful to make the distinction. A conceptual model
is more flexible than a design. It's easier to fit the requirements
and constraints into the model than into a design.

\begin{tip}
Strive to build a solid conceptual model before beginning the design.
\end{tip}

%Page 047 in first edition
%img2-47 moved forward
%Page 048 in first edition

\noindent Analysis consists of expanding the requirements definition
into a conceptual model. The technique involves two-way communication
with the customer in successive attempts to describe the model.

Like the entire development cycle, the analysis phase is best approached
iteratively. Each new requirement will tend to suggest something in
your mental model. Your job is to juggle all the requirements and
constraints until you can weave a pattern that fits the bill.

\wepsfigb{fig2-2}{An iterative approach to analysis.}

\Fig{fig2-2} illustrates the iterative approach to the analysis phase.
The final step is one of the most important: show the documented model
to the customer. Use whatever means of communication are
necessary---diagrams, tables, or cartoons---to convey your
understanding to the customer and get the needed feedback. Even if you
cycle through this loop a hundred times, it's worth the effort.

In the next three sections we'll explore three techniques for defining
and documenting the conceptual model:

\begin{enumerate}
\item defining the interfaces
\item defining the rules
\item defining the data structures.
\end{enumerate}%
\index{A!Analysis|)}

