\section{Choosing Names: The Art}%
\index{I!Implementation!choosing names|(}%
\index{N!Names:!choosing|(}

\begin{tfquot}
Besides a mathematical inclination, an exceptionally good mastery of
one's native tongue is the most vital asset of a competent programmer
\emph{(Prof.  \person{Edsger W. Dijkstra} \cite{dijkstra82}).}
\end{tfquot}%
\index{W!Words:!choosing names|(}
We've talked about the significance of using names to symbolize ideas
and objects in the application.  The choosing of names turns out to be
an important part of the design process.

Newcomers tend to overlook the important of names.  ``After all,''
they think, ``the computer doesn't care what names I choose.''

But good names are essential for readability.  Moreover, the mental
exercise of summoning a one-word description bears a synergistic effect
on your perceptions of what the entity should or should not do.

\smallbreak
Here are some rules for choosing good names:
\begin{tip}
Choose names according to ``what,'' not ``how.''
\end{tip}
A definition should hide the complexities of implementation from other
definitions which invoke it.  The name, too, should hide the details
of the procedure, and instead should describe the outward appearance
or net effect.

For instance, the \Forth{} word \forthb{ALLOT} simply increments the
dictionary pointer (called \forthb{DP} or \forthb{H} in most systems).
But the name \forthb{ALLOT} is better than \forth{DP+!} because the
user is thinking of reserving space, not incrementing a pointer.

The '83 Standard adopted the name \forthb{CMOVE>} instead of the
previous name for the same function, \forthb{<CMOVE}.  The operation
makes it possible to copy a region of memory \emph{forward} into
overlapping memory.  It accomplishes this by starting with the last
byte and working \emph{backward}.  In the new name, the forwardness of
the ``what'' supersedes the backwardness of the ``how.''

\begin{tip}
Find the most expressive word.\index{E!Expressive words}
\end{tip}
%Page 164 in first edition.

\begin{tfquot}
A powerful agent is the right word.  Whenever we come upon one of
those intensely right words in a book or a newspaper the resulting
effect is physical as well as spiritual, and electrically prompt
\emph{(\person{Mark Twain}).}

The difference between the right word and the almost-right word is
like the difference between lightning and the lightning bug
\emph{(\person{Mark Twain}).}

Suit the action to the word, the word to the action
\emph{(\person{Shakespeare}, Hamlet, Act~III).}
\end{tfquot}
\person{Henry Laxen},\index{L!Laxen, Henry}
a \Forth{} consultant and author, suggests that the most important
\Forth{} development tool is a good thesaurus \cite{laxen}.%
\index{T!Thesaurus}

Sometimes you'll think of an adequate word for a definition, but it
doesn't feel quite right.  It may be months later before you realize
that you fell short of the mark.  In the Roman numeral example in
\Chap{4}, there's a word that handles the exception case: numbers that
are one-less-than the next symbol's value.  My first choice was
\forth{4-0R-9}.  That's awkward, but it was much later that I thought
of \forth{ALMOST}.

Most fig-\Forth{} systems include the word \forth{VLIST}, which lists
the names of all the words in the current vocabulary.  After many
years someone realized that a nicer name is \forth{WORDS}.  Not only
does \forth{WORDS} sound more pleasant by itself, it also works nicely
with vocabulary names.  For instance:
\begin{Code}
EDITOR WORDS
\end{Code}
or
\begin{Code}
ASSEMBLER WORDS
\end{Code}
On the other hand, \person{Moore}\index{M!Moore, Charles} points out
that inappropriate names can become a simple technique for encryption.%
\index{E!Encryption}
If you need to provide security when you're forced to distribute
source, you can make your code very unreadable by deliberately
choosing misleading names.  Of course, maintenance becomes impossible.

\index{P!Phrases|(}%
\begin{tip}
Choose names that work in phrases.
\end{tip}
Faced with a definition you don't know what to call, think about how the
word will be used in context.  For instance:
%%!!!te this is how it looks like in the original
%%!!!te This really seems to be a description environment
\ifeightyfour
\begin{Code}[fontfamily=cmss,commandchars=\&\{\}]
SHUTTER OPEN
  OPEN is the appropriate name for a word that sets a
  bit in an I/O address identified with the name
  SHUTTER.&medskip
3 BUTTON DOES IGNITION
  DOES is a good choice for a word that vectors the
  address of the function IGNITION into a table of
  functions, so that IGNITION will be executed when
  Button 3 is pushed.&medskip
SAY HELLO
  SAY is the perfect choice for vectoring HELLO into an
  execution variable.  (When I first wrote this example
  for Starting &Forth{}, I called it VERSION. &person{Moore}
  reviewed the manuscript and suggested SAY, which is
  clearly much better.)&medskip
I'M HARRY
  The word I'M seems more natural than LOGON HARRY,
  LOGIN HARRY or SESSION HARRY, as often seen.
\end{Code}
\else
\begin{description}
\item[\forth{SHUTTER OPEN}]~

  \forth{OPEN} is the appropriate name for a word that sets a
  bit in an I/O address identified with the name
  \forth{SHUTTER}.
\item[\forth{3 BUTTON DOES IGNITION}]~

  \forth{DOES} is a good choice for a word that vectors the
  address of the function \forth{IGNITION} into a table of
  functions, so that \forth{IGNITION} will be executed when
  Button 3 is pushed.
\item[\forth{SAY HELLO}]~

  \forth{SAY} is the perfect choice for vectoring \forth{HELLO} into an
  execution variable.  (When I first wrote this example
  for Starting \Forth{}, I called it \forth{VERSION}. \person{Moore}
  reviewed the manuscript and suggested \forth{SAY}, which is
  clearly much better.)
\item[\forth{I'M HARRY}]~

  The word \forth{I'M} seems more natural than \forth{LOGON HARRY},
  \forth{LOGIN HARRY} or \forth{SESSION HARRY}, as often seen.
\end{description}
\fi

\index{M!Moore, Charles|(}%
\begin{interview}
The choice of \forth{I'M} is another invention of \person{Moore}, who
says:

\begin{tfquot}
I detest the word \forth{LOGON}. There is no such word in English.  I
was looking for a word that said, ``I'm \dots{}'' It was a natural.
I just stumbled across it.  Even though it's clumsy with that
apostrophe, it has that sense of rightness.

All these little words are the nicest way of getting the ``Aha!''
reaction.  If you think of the right word, it is \emph{obviously} the
right word.

If you have a wide recall vocabulary, you're in a better position to
come up with the right word.
\end{tfquot}
\end{interview}\index{M!Moore, Charles|)}%
Another of \person{Moore}'s favorite words is \forth{TH}, which he
uses as an array indexing word.  For instance, the phrase
\begin{Code}
5 TH
\end{Code}
returns the address of the ``fifth'' element of the array.%
\index{P!Phrases|)}

\index{S!Spelled-out names|(}%
\begin{tip}
Spell names in full.
\end{tip}
I once saw some \Forth{} code published in a magazine in which the
author seemed hell-bent on purging all vowels from his names,
inventing such eyesores as \forth{DSPL-BFR} for ``display buffer.''
Other writers seem to think that three characters magically says it
all, coining \forth{LEN} for ``length.'' Such practices reflect
thinking from a bygone age.

\index{A!Abbreviations|(}%
\Forth{} words should be fully spelled out.  Feel proud to type every
letter of \forth{INITIALIZE} or \forth{TERMINAL} or \forth{BUFFER}.
These are the words you mean.
%Page 166 in first edition.

The worst problem with abbreviating a word is that you forget just
how you abbreviated it.  Was that \forth{DSPL} or \forth{DSPLY?}

Another problem is that abbreviations hinder readability.
Any programming language is hard enough to read without compounding
the difficulty.

Still, there are exceptions.  Here are a few:

\begin{enumerate}
\item Words that you use extremely frequently in code. \Forth{}
employs a handful of commands that get used over and over, but have
little or no intrinsic meaning:
\begin{Code}
:   ;   @   !   .   ,
\end{Code}
But there are so few of them, and they're used so often, they become
old friends.  I would never want to type, on a regular basis,
\begin{Code}
DEFINE   END-DEFINITION   FETCH   STORE
     PRINT   COMPILE#
\end{Code}
(Interestingly, most of these symbols don't have English counterparts.
We use the phrase ``\emph{colon} definition'' because there's no other
term; we say ``\emph{comma} a number into the dictionary'' because
it's not exactly compiling, and there's no other term.)
\item Words that a terminal operator might use frequently to control
an operation.  These words should be spelled as single letters, as are
line editor commands.
\item Words in which familiar usage implies that they be abbreviated.
\Forth{} assembler mnemonics are typically patterned after the
manufacturer's suggested mnemonics, which are abbreviations (such as
\forth{JMP} and \forth{MOV}).
\end{enumerate}%
\index{S!Spelled-out names|)}
Your names should be pronounceable; otherwise you may regret it when
you try to discuss the program with other people.  If the name is
symbolic, invent a pronunciation (e.g., \forth{>R} is called ``to-r'';
\forth{R>} is called ``r-from'').

\begin{tip}
Favor short words.
\end{tip}
Given the choice between a three-syllable word and a one-syllable word
that means the same thing, choose the shorter.  \forth{BRIGHT} is a
better name than \forth{INTENSE}.  \forth{ENABLE} is a better name
than \forth{ACTIVATE}; \forth{GO}, \forth{RUN}, or \forth{ON} may be
better still.

Shorter names are easier to type.  They save space in the source
screen.  Most important, they make your code crisp and clean.

\index{H!Hyphenated names|(}%
\begin{tip}
Hyphenated names may be a sign of bad factoring.
\end{tip}%
\index{A!Abbreviations|)}
%Page 168 in first edition.

\begin{interview}
\person{Moore}:\index{M!Moore, Charles|(}

\begin{tfquot}
There are diverging programming styles in the \Forth{} community.  One
uses hyphenated words that express in English what the word is doing.
You string these big long words together and you get something that is
quite readable.

But I immediately suspect that the programmer didn't think out the words
carefully enough, that the hyphen should be broken and the words defined
separately.  That isn't always possible, and it isn't always advantageous.
But I suspect a hyphenated word of mixing two concepts.%
\end{tfquot}\index{M!Moore, Charles|)}
\end{interview}
Compare the following two strategies for saying the same thing:
\begin{Code}
ENABLE-LEFT-MOTOR        LEFT MOTOR ON
ENABLE-RIGHT-MOTOR       RIGHT MOTOR ON
DISABLE-LEFT-MOTOR       LEFT MOTOR OFF
DISABLE-RIGHT-MOTOR      RIGHT MOTOR OFF
ENABLE-LEFT-SOLENOID     LEFT SOLENOID ON
ENABLE-RIGHT-SOLENOID    RIGHT SOLENOID ON
DISABLE-LEFT-SOLENOID    LEFT SOLENOID OFF
DISABLE-RIGHT-SOLENOID   RIGHT SOLENOID OFF
\end{Code}
The syntax on the left requires eight dictionary entries; the syntax
on the right requires only six-and some of the words are likely to be
reused in other parts of the application.  If you had a \forth{MIDDLE}
motor and solenoid as well, you'd need only seven words to describe
sixteen combinations.%
\index{H!Hyphenated names|)}

\begin{tip}
Don't bundle numbers into names.
\end{tip}
Watch out for a series of names beginning or ending with numbers, such
as \forth{1CHANNEL}, \forth{2CHANNEL}, \forth{3CHANNEL}, etc.

\index{B!Bundling of names and numbers|(}%
This bundling of names and numbers may be an indication of bad
factoring.  The crime is similar to hyphenation, except that what
should be factored out is a number, not a word.  A better factoring of
the above would be
\begin{Code}
1 CHANNEL
2 CHANNEL
3 CHANNEL
\end{Code}
In this case, the three words were reduced to one.

Often the bundling of names and numbers indicates fuzzy naming.
%Page 168 in first edition.
In the above case, more descriptive names might indicate the purpose
of the channels, as in
\begin{Code}
VOICE , TELEMETRY , GUITAR
\end{Code}
%% Note: AH: the commas be better left out here.
\index{B!Bundling of names and numbers|)}%
We'll amplify on these ideas in the next chapter on ``Factoring.''%
\index{N!Names:!choosing|)}

