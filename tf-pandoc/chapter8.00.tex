%% Thinking Forth
%% Copyright (C) 2004 Leo Brodie
%% Initial transcription by Nils M Holm
%% Based on OCR scans by Steve Fisher
%% 
%% Chapter: Eight - Minimizing Control Structures

%!! horizontal line
%EIGHT
%!! horizontal line

\chapter{Minimizing Control~Structures}\Chapmark{8}%
\index{C!Control structure minimization|(}%

\initial Control structures aren't as important in \Forth{} as they
are in other languages. \Forth{} programmers tend to write very
complex applications in terms of short words, without much emphasis on
\forth{IF THEN} constructs.

There are several techniques for minimizing control structures.
They include:

%!! items should be preceded by bullets
\begin{itemize}
\item computing or calculating
\item hiding conditionals through re-factoring
\item using structured exits
\item vectoring
\item redesigning.
\end{itemize}
In this chapter we'll examine these techniques for simplifying and
eliminating control structures from your code.

