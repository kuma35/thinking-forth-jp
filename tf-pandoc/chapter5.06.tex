\section{Naming Standards: The Science}%
\index{N!Naming conventions|(}

\begin{tip}
Learn and adopt \Forth{}'s naming conventions.
\end{tip}
In the quest for short, yet meaningful names, \Forth{} programmers
have adopted certain naming conventions.  \App{E} includes a list of
the most useful conventions developed over the years.

An example of the power of naming conventions is the use of ``dot''
to mean ``print'' or ``display.'' \Forth{} itself uses
\begin{Code}
.   D.   U.R
\end{Code}
for displaying various types of numbers in various formats.  The
convention extends to application words as well.  If you have a
variable called \forth{DATE,} and you want a word that displays the
date, use the name
\begin{Code}
.DATE
\end{Code}
\index{S!Suffixes|(}%
\index{P!Prefixes|(}%
A caution: The overuse of prefixes and suffixes makes words uglier and
ultimately less readable.  Don't try to describe everything a word
does by its name alone.  After all, a name is a symbol, not a
shorthand for code.  Which is more readable and natural sounding?:

\begin{tfquot}
Oedipus complex
\end{tfquot}
(which bears no intrinsic meaning), or
\begin{tfquot}
subconscious-attachment-to-parent-of-opposite-sex complex
\end{tfquot}
Probably the former, even though it assumes you know the play.

\begin{tip}
Use prefixes and suffices to differentiate between like words rather
than to cram details of meaning into the name itself.
\end{tip}
%Page 169 in first edition.
For instance, the phrase
\begin{Code}
... DONE IF CLOSE THEN ...
\end{Code}
is just as readable as
\begin{Code}
... DONE? IF CLOSE THEN ...
\end{Code}
and cleaner as well.  It is therefore preferable, unless we need an
additional word called \forth{DONE} (as a flag, for instance).

\medbreak
A final tip on naming:\index{W!Words:!choosing names|)}
\begin{tip}
Begin all hex numbers with ``0'' (zero) to avoid potential collisions
with names.
\end{tip}
For example, write \forth{0ADD}, not \forth{ADD}.%
\index{S!Suffixes|)}%
\index{P!Prefixes|)}

By the way, don't expect your \Forth{} system to necessarily conform
to the above conventions.  The conventions are meant to be used in
new applications.

\Forth{} was created and refined over many years by people who used it
as a means to an end.  At that time, it was neither reasonable nor
possible to impose naming standards on a tool that was still growing
and evolving.

Had \Forth{} been designed by committee, we would not love it so.%
\index{N!Naming conventions|)}%
\index{I!Implementation!choosing names|)}

