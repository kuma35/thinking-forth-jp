\section{Budgeting and Scheduling}%
\index{B!Budgeting|(}%
\index{A!Analysis!budgeting|(}%
\index{A!Analysis!scheduling|(}%
\index{S!Scheduling|(}

Another important aspect of the analysis phase is figuring the price
tag.  Again, this process is much more difficult than it would seem.
If you don't know the problem till you solve it, how can you possibly
know how long it will take to solve it?

Careful planning is essential, because things always take longer than
you expect. I have a theory about this, based on the laws of
probability:

%Page 066 in first edition

\wepsfigp{img2-066}{Conventional wisdom reveres complexity.}

%Page 067 in first edition
\begin{tip}
The mean time for making a ``two-hour'' addition to an application is
approximately 12 hours.
\end{tip}

\noindent Imagine the following scenario: You're in the middle of
writing a large application when suddenly it strikes you to add some
relatively simple feature. You think it'll take about two hours, so
without further planning, you just do it. Consider: That's two hours
coding time. The design time you don't count because you perceived the
need---and the design---in a flash of brilliance while working on the
application. So you estimate two hours.

But consider the following possibilities:

\begin{enumerate}
\item Your implementation has a bug. After two hours it doesn't work.
So you spend another two hours recoding. (Total 4.)
\item OR, before you implemented it, you realized your initial design
wouldn't work. You spend two hours redesigning. \emph{These} two hours
count. Plus another two hours coding it. (Total 4.)
\item OR, you implement the first design before you realize the design
wouldn't work. So you redesign (two more hours) and reimplement (two
more). (Total 6.)
\item OR, you implement the first design, code it, find a bug, rewrite
the code, find a design flaw, redesign, recode, find a bug in the new
code, recode again. (Total 10.)
\suspend{enumerate}
You see how the thing snowballs?
\resume{enumerate}
\item Now you have to document your new feature. Add two hours to the
above. (Total 12.)
\item After you've spent anywhere from 2 to 12 hours installing and
debugging your new feature, you suddenly find that element Y of your
application bombs out. Worst yet, you have no idea why. You spend two
hours reading memory dumps trying to divine the reason. Once you do,
you spend as many as 12 additional hours redesigning element Y. (Total
26.) Then you have to document the syntax change you made to element
Y. (Total 27.)
\end{enumerate}

\noindent That's a total of over three man-days. If all these mishaps
befell you at once, you'd call for the men with the little white
coats. It rarely gets that bad, of course, but the odds are decidedly
\emph{against} any project being as easy as you think it will be.

How can you improve your chances of judging time requirements
correctly? Many fine books have been written on this topic, notably
\emph{The Mythical Man-Month} by \person{Frederick P. Brooks}, Jr.
\cite{brooks75}.%
\index{B!Brooks,Fredrick P., Jr.}%
\index{M!Mythical@\emph{Mythical Man-Month, The} Brooks}
I have little to add to this body of knowledge except for some
personal observations.

\begin{enumerate}

\item Don't guess on a total. Break the problem up into the smallest
possible pieces, then estimate the time for each piece. The sum of the
pieces is always greater than what you'd have guessed the total would
be. (The whole appears to be less than the sum of the parts.)

%Page 068 in first edition
\item In itemizing the pieces, separate those you understand well
enough to hazard a guess from those you don't. For the second
category, give the customer a range.

\item A bit of psychology: always give your client some options.
Clients \emph{like} options. If you say, ``This will cost you \$6,000,''
the client will probably respond ``I'd really like to spend \$4,000.''
This puts you in the position of either accepting or going without a job.

But if you say, ``You have a choice: for \$4,000 I'll make it
\emph{walk} through the hoop; for \$6,000 I'll make it \emph{jump}
through the hoop. For \$8,000 I'll make it \emph{dance} through the
hoop waving flags, tossing confetti and singing ``Roll Out the Barrel.''

Most customers opt for jumping through the hoop.
\end{enumerate}

\begin{tip}
Everything takes longer than you think, including thinking.
\end{tip}%
\index{A!Analysis!budgeting|)}%
\index{A!Analysis!scheduling|)}%
\index{B!Budgeting|)}%
\index{S!Scheduling|)}

