\section{Vertical Format vs.\ Horizontal Format}%
\index{V!Vertical format vs. horizontal format|(}%
\index{I!Implementation!vertical format vs. horizontal format|(}
The purpose of commenting is to allow a reader of your code to easily
determine what's going on.  But how much commenting is necessary? To
determine the level of commenting appropriate for your circumstances,
you must ask yourself two questions:

%%!!! this didn't have any bullet points or whatever in the first edition
\begin{itemize}
\expandafter\item\ifeightyfour[]\fi Who will be reading my code?
\expandafter\item\ifeightyfour[]\fi How readable are my definitions?
\end{itemize}%
\index{N!Narrative comments|(}
There are two basic styles of commenting to choose from.  The first
style, often called the ``vertical format,'' includes a step-by-step
description of the process, in the manner of a well-commented assembly
language listing.  These line-by-line comments are called ``narrative
comments.''

%Page 160 in first edition.
\begin{Code}
\ CRC Checksum                                      07/15/83
: ACCUMULATE   ( oldcrc char -- newcrc)
   256 *               \ shift char to hi-order byte
   XOR                 \ & xor into previous crc
   8 0 DO              \ Then for eight repetitions,
       DUP 0< IF       \ if hi-order bit is "1"
          16386 XOR    \ xor it with mask and
          DUP +        \ shift it left one place
          1+           \ set lo-order bit to "1"
              ELSE     \ otherwise, i.e. hi-order bit is "0"
          DUP +        \ shift it left one place
              THEN
       LOOP ;          \ complete the loop
\end{Code}
The other approach does not intersperse narrative comments between
code phrases.  This is called the ``horizontal format.''
\begin{Code}
: ACCUMULATE  ( oldcrc char -- newcrc)
   256 *  XOR  8 0 DO  DUP 0< IF
      16386 XOR  DUP +  1+  ELSE  DUP +  THEN  LOOP ;
\end{Code}
The vertical format is preferred when a large team of programmers are
coding and maintaining the application.  Typically, such a team will
include several junior-level programmers responsible for minor
corrections.  In such an environment, diligent commenting can save a
lot of time and upset.  As \person{Johnson}\index{J!Johnson, Dave} of Moore
Products Co.  says: ``When maintaining code you are usually interested
in just one small section, and the more information written there the
better your chances for a speedy fix.''

Here are several pertinent rules required of the \Forth{} programmers
at Moore Products Co. (I'm paraphrasing):

\begin{enumerate}
\item A vertical format will be used.  Comments will appear to the
right of the source code, but may continue to engulf the next line
totally if needed.
\item There should be more comment characters than source characters.
(The company encourages long descriptive names, greater than ten
characters, and allows the names to be counted as comment characters.)
\item Any conditional structure or application word should appear on a
separate line.  ``Noise words'' can be grouped together.  Indentation
is used to show nested conditionals.
\end{enumerate}
There are some difficulties with this format, however.%
\index{L!Line-by-line commenting|(}
For one thing, line-by-line commenting is time-consuming, even with a
good screen editor.  Productivity can be stifled, especially when
stopping to write the comments breaks your chain of thought.

Also, you must also carefully ensure that the comments are up-to-date.
Very often code is corrected, the revision is tested, the change
%Page 161 in first edition.
works---and the programmer forgets to change the comments.  The more
comments there are, the more likely they are to be wrong.  If they're
wrong, they're worse than useless.

This problem can be alleviated if the project supervisor carefully
reviews code and ensures the accuracy of comments.

Finally, line-by-line commenting can allow a false sense of security.
Don't assume that because each \emph{line} has a comment, the
\emph{application} is well-com\-men\-ted.  Line-by-line commenting doesn't
address the significant aspects of a definition's operation.  What,
for instance, is the thinking behind the checksum algorithm used? Who
knows, from the narrative comments?%
\index{L!Line-by-line commenting|)}%
\index{N!Narrative comments|)}

To properly describe, in prose, the implications of a given procedure
usually requires many paragraphs, not a single phrase.  Such
descriptions properly belong in auxiliary documentation or in the
chapter preamble.

Despite these cautions, many companies find the vertical format
necessary.  Certainly a team that is newly exposed to \Forth{} should
adopt it, as should any very large team.

What about the horizontal format? Perhaps it's an issue of art vs.\@
practicality, but I feel compelled to defend the horizontal format as
equally valid and in some ways superior.

If \Forth{} code is really well-written, there should be nothing
ambiguous about it.  This means that:

\begin{itemize}
\item supporting lexicons have a well-designed syntax
\item stack inputs and outputs are commented
\item the purpose is commented (if it's not clear from the name or
stack comment)
\item definitions are not too long
\item not too many arguments are passed to a single definition via the
stack (see ``The Stylish Stack'' in \Chap{7}).
\end{itemize}
\Forth{} is simply not like other languages, in which line-by-line
commenting is one of the few things you can do to make programs more
readable.

Skillfully written \Forth{} code is like poetry, containing precise
meaning that both programmer and machine can easily read.  Your
\emph{goal} should be to write code that does not need commenting,
even if you choose to comment it.  Design your application so that the
code, not the comments, conveys the meaning.

If you succeed, then you can eliminate the clutter of excessive
commenting, achieving a purity of expression without redundant
explanations.

\wepsfigp{fig5-2}{Wiggins, proud of his commenting technique.}

\begin{tip}
The most-accurate, least-expensive documentation
is self-documenting code.
\end{tip}
%Page 162 in first edition.
Unfortunately, even the best programmers, given the pressure of a
deadline, may write working code that is not easily readable without
comments.  If you are writing for yourself, or for a small group with
whom you can verbally communicate, the horizontal format is ideal.
Otherwise, consider the vertical format.%
\index{V!Vertical format vs. horizontal format|)}%
\index{I!Implementation!vertical format vs. horizontal format|)}

