\section{Example: A Tiny Editor}\program{editor1}

Let's apply decomposition by component to a real problem. It would be
nice to design a large application right here in \Chap{3}, but alas, we
don't have the room and besides, we'd get sidetracked in trying to
understand the application.

Instead, we'll take a component from a large application that has
already been decomposed. We'll design this component by decomposing it
further, into subcomponents.

Imagine that we must create a tiny editor that will allow users to
change the contents of input fields on their terminal screen. For instance,
the screen might look like this:

%!! no bf cmtt font (use Courier?)
\begin{quote}
\textsf{\textbf{Name of Member}\quad \fbox{Justine Time\underline{~}\quad}}
\end{quote}
The editor will provide three modes for users to change the contents of
the input field:

%!! original style: label in italics, no indentation
\begin{description}

\item[Overwrite.] Typing ordinary characters overwrites any characters
that were there before.

\item[Delete.] Pressing the combination of keys ``Ctrl D'' deletes the
character under the cursor and slides the remaining characters
leftwards.

\item[Insert.] Pressing the combination of keys ``Ctrl I'' switches the
editor into ``Insert Mode,'' where subsequently typing ordinary
characters inserts them at the cursor position, sliding the remaining
characters rightwards.

\end{description}
As part of the conceptual model we should also consider the error or
exception-handling; for instance, what is the limit of the field? what
happens in insert mode when characters spill off the right? etc.

That's all the specification we have right now. The rest is up to us.

Let's try to determine what components we'll need. First, the editor
will react to keys that are typed at the keyboard. Therefore we'll
need a keystroke interpreter---some kind of routine that awaits
keystrokes and matches them up with a list of possible operations. The
keystroke interpreter is one component, and its lexicon will consist
of a single word.  Since that word will allow the editing of a field,
let's call the word \forth{EDIT}.

The operations invoked by the keystroke interpreter will comprise a
second lexicon. The definitions in this lexicon will perform the
various functions required. One word might be called \forth{DELETE}, another
\forth{INSERT}, etc. Since each of these commands will be invoked by the
interpreter, each of them will process a single keystroke.

Below these commands should lie a third component, the set of
words that implement the data structure to be edited.

\wepsfiga{fig3-2}{Generalized decomposition of the Tiny Editor problem.}

Finally, we'll need a component to display the field on the video
screen. For the sake of simplicity, let's plan on creating one word only,
\forth{REDISPLAY}, to redisplay the entire field after each key is pressed.

\begin{Code}
: EDITOR  BEGIN  KEY  REVISE  REDISPLAY  ... UNTIL ;
\end{Code}
This approach separates revising the buffer from updating the display.
For now, we'll only concentrate on revising the buffer.

Let's look at each component separately and try to determine the words
each will need. We can begin by considering the events that must occur
within the three most important editing functions: overwriting,
deleting, and inserting. We might draw something like the following on
the back of an old pizza menu (we won't pay much attention to
exception-handling in the present discussion):

\begin{description}
%!!small layout differences between the '84 and '94 edition; using '94 layout
\item[To Overwrite:]~\\[\smallskipamount]
\begin{tabular}{@{}p{2.1in}>{\ttfamily}p{2.1in}}
Store new character into byte pointer to by pointer.

Advance pointer (unless at end of field).
& \parbox[t]{2.1in}{
  F U N \fwbox{\pointto{K}} T I O N A L I T Y\\
  F U N \fwbox{\pointto{C}} T I O N A L I T Y\\
  F U N C \pointto{T} I O N A L I T Y}\\
\end{tabular}

\item[To Delete:]~\\[\smallskipamount]
\begin{tabular}{@{}p{2.1in}>{\ttfamily}p{2.1in}}
Copy leftwards, by one place, the string
beginning one place to the right of the pointer.

Store a ``blank'' into the last
position on the line.
& \parbox[t]{2.1in}{
  F U N C T I O N \pointto{S} \fwbox{A L I T Y}\\
  F U N C T I O N \fwbox{\pointto{A} L I T Y} Y\\
  F U N C T I O N \pointto{A} L I T Y \fwbox{\newlength{\charheight}\settoheight{\charheight}{K}\rule{0cm}{\charheight} }}\\
\end{tabular}

% Note, pointers-and-boxes omitted in this diagram for some reason.
\item[To Insert:]~\\[\smallskipamount]
\begin{tabular}{@{}p{2.1in}>{\ttfamily}p{2.1in}}
Copy rightwards, by one place,
the string beginning at the pointer.
%!!'84 ed: the string beginning one place\\ to the right of the pointer.

Store new character into
byte pointed to by pointer.

Advance pointer (unless at end of field).
& \parbox[t]{2.1in}{
  F U N \fwbox{\pointto{T} I O N A L I T Y}\\
  F U N \pointto{T} \fwbox{T I O N A L I T Y}\\
  F U N \fwbox{\pointto{C}} T I O N A L I T Y\\
  F U N C \pointto{T} I O N A L I T Y} \\

\end{tabular}
\end{description}
We've just developed the algorithms for the problem at hand.

Our next step is to examine these three essential procedures, looking
for useful ``names''---that is procedures or elements which can
either:

\begin{enumerate}
\item possibly be reused, or

\item possibly change
\end{enumerate}
We discover that all three procedures use something called a ``pointer.''
We need two procedures:

\begin{enumerate}
\item to get the pointer (if the pointer itself is relative, this
function will perform some computation).

\item to advance the pointer

%!!next line not part of enumeration (not indented in '94 edition).
\suspend{enumerate}
Wait, three procedures:
\resume{enumerate}
\item to move the pointer backwards
\end{enumerate}
because we will want ``cursor keys'' to move the cursor forward and back
without editing changes.

These three operators will all refer to a physical pointer somewhere
in memory. Where it is kept and how (relative or absolute) should be
hidden within this component.

Let's attempt to rewrite these algorithms in code:

\begin{Code}
: KEY#  ( returns value of key last pressed )  ... ;
: POSITION  ( returns address of character pointed-to)  ;
: FORWARD  ( advance pointer, stopping at last position)  ;
: BACKWARD  ( decrement pointer, stopping at first position)  ;
: OVERWRITE   KEY# POSITION C!  FORWARD ;
: INSERT   SLIDE>  OVERWRITE ;
: DELETE   SLIDE<  BLANK-END ;
\end{Code}
To copy the text leftwards and rightwards, we had to invent two new
names as we went along, \forth{SLIDE<} and \forth{SLIDE>} (pronounced
``slide-backwards'' and ``slide-forwards'' respectively). Both of them
will certainly use \forth{POSITION}, but they also must rely on an element
we've deferred considering: a way to ``know'' the length of the
field. We can tackle that aspect when we get to writing the third
component.  But look at what we found out already: we can describe
``Insert'' as simply ``\forth{SLIDE> OVERWRITE}''.

In other words, ``Insert'' actually \emph{uses} ``Overwrite'' even though
they appear to exist on the same level (at least to a Structured
Programmer).

Instead of probing deeper into the third component, let's lay out
what we know about the first component, the key interpreter. First we
must solve the problem of ``insert mode.'' It turns out that ``insert'' is
not just something that happens when you press a certain key, as
delete is. Instead it is a \emph{different way of interpreting}
some of the possible keystrokes.

For instance in ``overwrite'' mode, an ordinary character gets stored
into the current cursor position; but in ``insert mode'' the remainder
of the line must first be shifted right. And the backspace key works
differently when the editor is in Insert Mode as well.

Since there are two modes, ``inserting'' and ``not-inserting,'' the
keystroke interpreter must associate the keys with two possible sets of
named procedures.

We can write our keystroke interpreter as a decision table (worrying
about the implementation later):

\bigskip
\begin{tabular}{>{\ttfamily}l>{\ttfamily}l>{\ttfamily}l}
\emph{\textrm{Key}} &\emph{\textrm{Not-inserting}}& \emph{\textrm{Inserting}}\\
Ctrl-D        & DELETE              & INSERT-OFF\\
Ctrl-I        & INSERT-ON           & INSERT-OFF\\
backspace      & BACKWARD            & INSERT<	 \\
left-arrow     & BACKWARD            & INSERT-OFF\\
right-arrow    & FORWARD             & INSERT-OFF\\
return         & ESCAPE              & INSERT-OFF\\
any printable  & OVERWRITE           & INSERT    \\
\end{tabular}
\bigskip

\noindent We've placed the possible types of keys in the left column,
what they do normally in the middle column, and what they do in
``insert mode'' in the right column.

To implement what happens when ``backspace'' is pressed while in
Insert Mode, we add a new procedure:

\begin{Code}
: INSERT<   BACKWARD  SLIDE< ;
\end{Code}
(move the cursor backwards on top of the last character typed, then slide
everything to the right leftward, covering the mistake).

This table seems to be the most logical expression of the problem at
the current level. We'll save the implementation for later (\Chap{8}).

Now we'll demonstrate the tremendous value of this approach in
terms of maintainability. We'll throw ourselves a curve---a major change
of plans!

