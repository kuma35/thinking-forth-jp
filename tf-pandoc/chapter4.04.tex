\section{\Forth{} Syntax}%
\index{D!Detailed design!forth@\Forth{} syntax|(}%
\index{F!forth@\Forth{}!syntax|(}%
\index{S!Syntax \Forth{}|(}

At this point in the development cycle you must decide how the words in
your new lexicon will be used in context. In doing so, keep in mind how
the lexicon will be used by subsequent components.

\begin{tip}
In designing a component, the goal is to create a lexicon that will make your
later code readable and easy to maintain.
\end{tip}
Each component should be designed with components that use it in mind.
You must design the syntax of the lexicon so that the words make sense
when they appear in context. Hiding interrelated information within the
component will ensure maintainability, as we've seen.

At the same time, observe \Forth{}'s own syntax. Rather than insisting
on a certain syntax because it seems familiar, you may save
yourself from writing a lot of unnecessary code by choosing a syntax that
\Forth{} can support without any special effort on your part.

Here are some elementary rules of \Forth{}'s natural syntax:

\begin{tip}
Let numbers precede names.
\index{N!Numbers-precede-names syntax rule}
\end{tip}
%!! include cartoon of page 110 here
Words that require a numeric argument will naturally expect to find that
number on the stack. Syntactically speaking, then, the number should
precede the name. For instance, the syntax of the word \forth{SPACES}, which
emits ``$n$'' number of spaces, is

\begin{Code}
20 SPACES
\end{Code}
Sometimes this rule violates the order that our ear is accustomed to
hearing. For instance, the \Forth{} word \forth{+} expects to be preceded
by both arguments, as in

\begin{Code}
3 4 +
\end{Code}
\index{P!Postfix notation|(}
This ordering, in which values precede operators, is called ``postfix.''

\Forth{}, in its magnanimity, won't \emph{insist} upon postfix notation.
You could redefine \forth{+} to expect one number in the input stream,
like this:

\begin{Code}
3 + 4
\end{Code}
by defining it so:
\begin{Code}
: +   BL WORD  NUMBER DROP  + ;
\end{Code}
(where \forthb{WORD} is 79/83 Standard, returning an address, and
\forthb{NUMBER} returns a double-length value as in the 83 Standard
Uncontrolled Reference Words).

Fine. But you wouldn't be able to use this definition inside other colon
definitions or pass it arguments, thereby defeating one of \Forth{}'s
major advantages.

Frequently, ``noun'' type words pass their addresses (or any type of
pointer) as a stack argument to ``verb'' type words. The \Forth{}-like
syntax of
\begin{quote}
%!! sans-serif
{\sf ``noun'' ``verb''}
\end{quote}
\wepsfigxx{img4-110}
will generally prove easiest to implement because of the stack.\medbreak

In some cases this word order\index{W!Words:!ordering} sounds
unnatural.  For instance, suppose we have a file named
\forth{INVENTORY}. One thing we can do with that file is \forth{SHOW}
it; that is, format the information in pretty columns. If
\forth{INVENTORY} passes a pointer to \forth{SHOW}, which acts upon
it, the syntax becomes
\begin{Code}
INVENTORY SHOW
\end{Code}
If your spec demands the English word-order,\index{W!Words:!ordering}
\Forth{} offers ways to
achieve it. But most involve new levels of complexity. Sometimes the
best thing to do is to choose a better name. How about
\begin{Code}
INVENTORY REPORT
\end{Code}
(We've made the ``pointer'' an adjective, and the ``actor'' a noun.)

If the requirements insist on the syntax
\begin{Code}
SHOW INVENTORY
\end{Code}
we have several options. \forth{SHOW} might set a flag and
\forth{INVENTORY} would act according to the flag. Such an approach has
certain disadvantages, especially that \forth{INVENTORY} must be ``smart''
enough to know all the possible actions that might be taken on it. (We'll
treat these problems in Chapters \ref{chapter-7} and \ref{chapter-8}.)%
\index{P!Postfix notation|)}

Or, \forth{SHOW} might look ahead at the next word in the input stream.
We'll discuss this approach in a tip, ``Avoid expectations,'' later in this
chapter.

Or, the recommended approach, \forth{SHOW} might set an ``execution
variable'' that \forth{INVENTORY} will then execute. (We'll discuss vectored
execution in \Chap{7}.)

\begin{tip}
Let text follow names.
\index{T!Text-follows-names syntax rule}
\end{tip}
If the \Forth{} interpreter finds a string of text that is neither a number
nor a predefined word, it will abort with an error message. For this
reason, an undefined string must be preceded by a defined word.

An example is \forth{."} (dot-quote), which precedes the text it will
later print. Another example is \forthb{CREATE} (as well as all defining
words), which precedes the name that is, at the moment, still undefined.

The rule also applies to defined words that you want to refer to, but
not execute in the usual way. An example is \forthb{FORGET}, as in

\begin{Code}
FORGET TASK
\end{Code}
Syntactically, \forthb{FORGET} must precede \forth{TASK} so that
\forth{TASK} doesn't execute.

\begin{tip}
Let definitions consume their arguments.
\end{tip}
This syntax rule is more a convention of good \Forth{} programming
than a preference of \Forth{}.
\index{D!Definitions-consume-arguments syntax rule}

Suppose you're writing the word \forth{LAUNCH}, which requires the
number of a launch pad and fires the appropriate rocket. You want the
definition to look roughly like this:
\begin{Code}
: LAUNCH  ( pad#)  LOAD  AIM  FIRE ;
\end{Code}
Each of the three internal definitions will require the same argument, the
launch pad number. You'll need two \forthb{DUP}s somewhere. The question
is where? If you put them inside \forth{LOAD} and \forth{AIM}, then you
can keep them out of \forth{LAUNCH}, as in the definition above. If you
leave them out of \forth{LOAD} and \forth{AIM}, you'll have to define:
\begin{Code}
: LAUNCH  ( pad#)  DUP LOAD  DUP AIM  FIRE ;
\end{Code}
By convention, the latter version is preferable, because \forth{LOAD} and
\forth{AIM} are cleaner. They do what you expect them to do. Should you
have to define \forth{READY}, you can do it so:
\begin{Code}
: READY  ( pad#)  DUP LOAD  AIM ;
\end{Code}
and not
\begin{Code}
: READY  ( pad#)  LOAD  AIM  DROP ;
\end{Code}

\begin{tip}
Use zero-relative numbering.
\end{tip}
By habit we humans number things starting with one: ``first, second,
third,'' etc. Mathematical models, on the other hand, work more naturally
\index{Z!Zero-relative numbering|(}
when starting with zero. Since computers are numeric processors, software
becomes easier to write when we use zero-relative numbering.

To illustrate, suppose we have a table of eight-byte records. The first
record occupies the first eight bytes of the table. To compute its
starting address, we add ``0'' to \forth{TABLE}. To compute the starting
address of the ``second'' record, we add ``8'' to \forth{TABLE}.

\wepsfiga{fig4-6}{A table of 8-byte records.}

%!! include Figure 4-6 here
\medbreak
It's easy to derive a formula to achieve these results:
%!! should be two columns, right-hand side sans-serif

\bigskip
\begin{tabular}{@{}l@{ }l@{}r}
\sf first record starts at:  &  $\mathsf{0 \times 8} = {}$ & \sf 0  \\
\sf second record starts at: &  $\mathsf{1 \times 8} = {}$ & \sf 8  \\
\sf third record starts at:  &  $\mathsf{2 \times 8} = {}$ & \sf 16 \\
\end{tabular}
\bigskip

\noindent
We can easily write a word which converts a record\# into the address
where that record begins:

\begin{Code}
: RECORD  ( record# -- adr )
   8 *  TABLE + ;
\end{Code}
Thus in computer terms it makes sense to call the ``first record'' the 0th
record.

If your requirements demand that numbering start at one, that's
fine. Use zero-relative numbering throughout your design and then, only
in the ``user lexicons'' (the set of words that the end-user will use)
include the conversion from zero-to one-relative numbering:
\begin{Code}
: ITEM  ( n -- adr)  1- RECORD ;
\end{Code}
\index{Z!Zero-relative numbering|)}

\begin{tip}
Let addresses precede counts.\index{A!Address-precedes-counts syntax rule}
\end{tip}
Again, this is a convention, not a requirement of \Forth{}, but such
conventions are essential for readable code. You'll find examples of this
rule in the words \forthb{TYPE}, \forthb{ERASE}, and \forthb{BLANK}.

\begin{tip}
Let sources precede destinations.
\index{S!Source-precede-destination syntax rule}
\end{tip}
Another convention for readability. For instance, in some systems, the
phrase
\begin{Code}
22 37 COPY
\end{Code}
copies Screen 22 to Screen 37. The syntax of \forth{CMOVE} incorporates both
this convention and the previous convention:
\begin{Code}[commandchars=\&\{\}]
source destination count &poorbf{CMOVE}
\end{Code}

\index{E!Expectations-in-input-stream syntax rule|(}
\begin{tip}
Avoid expectations (in the input stream).
\end{tip}
Generally try to avoid creating words that presume there will be other
words in the input stream.

Suppose your color computer represents blue with the value 1, and
light-blue with 9. You want to define two words: \forth{BLUE} will return
1; \forth{LIGHT} may precede \forth{BLUE} to produce 9.

In \Forth{}, it would be possible to define \forth{BLUE} as a constant, so
that when executed it always returns 1.\program{color1}

\begin{Code}
1 CONSTANT BLUE
\end{Code}
And then define \forth{LIGHT} such that it looks for the next word in the
input stream, executes it, and ``ors'' it with 8 (the logic of this will
become apparent when we visit this example again, later in the book):
\begin{Code}
: LIGHT  ( precedes a color)  ( -- color value)
     ' EXECUTE  8 OR ;
\end{Code}
(in fig-\Forth{}:
\begin{Code}[commandchars=\&\{\}]
: LIGHT [COMPILE] '  CFA EXECUTE  8 OR ;&textrm{)}
\end{Code}
%!! The closing paren at the end of this code is a bit misleading.
%!! At least a leading space and a different font should be used.
\noindent (For novices: The apostrophe in the definition of \forth{LIGHT}
is a \Forth{} word called ``tick.'' Tick is a dictionary-search word; it
takes a name and looks it up in the dictionary, returning the address
where the definition resides. Used in this definition, it will find the
address of the word following \forth{LIGHT}---for instance,
\forth{BLUE}---and pass this address to the word \forthb{EXECUTE}, which
will execute \forth{BLUE}, pushing a one onto the stack.  Having ``sucked
up'' the operation of \forth{BLUE}, \forth{LIGHT} now ``or''s an 8 into
the 1, producing a 9.)

This definition will work when invoked in the input stream, but special
handling is required if we want to let \forth{LIGHT} be invoked within a
colon definition, as in:
\begin{Code}
: EDITING   LIGHT BLUE BORDER ;
\end{Code}
Even in the input stream, the use of \forth{EXECUTE} here will cause a
crash if \forth{LIGHT} is accidentally followed by something other than a
defined word.

The preferred technique, if you're forced to use this particular
syntax, is to have \forth{LIGHT} set a flag, and have \forth{BLUE}
determine whether that flag was set, as we'll see later on.

There will be times when looking ahead in the input stream is desirable,
even necessary. (The proposed \forth{TO} solution is often implemented
this way \cite{rosen82}.)

But generally, avoid expectations. You're setting yourself up for
disappointment.
\index{E!Expectations-in-input-stream syntax rule|)}

\begin{tip}
Let commands perform themselves.
\end{tip}%
\index{C!Compilers|(}%
This rule is a corollary to ``Avoid expectations.'' It's one of
\Forth{}'s philosophical quirks to let words do their own work. Witness
the \Forth{} compiler (the function that compiles colon definitions),
caricatured in \Fig{fig4-7}. It has very few rules:

\wepsfigt{fig4-7}{The traditional compiler vs. the \Forth{} compiler.}

%!! include Figure 4-7 here

%!! this paragraph should be an indented, unordered list
\begin{itemize}
\item Scan for the next word in the input stream and look it up in the
dictionary.

\item If it's an ordinary word, \emph{compile} its address.

\item If it's an ``immediate'' word, \emph{execute} it.

\item If it's not a defined word, try to convert it to a number and
compile it as a literal.

\item If it's not a number, abort with an error message.
\end{itemize}
Nothing is mentioned about compiling-words such as \forthb{IF},
\forthb{ELSE}, \forthb{THEN}, etc. The colon compiler doesn't know about
these words. It merely recognizes certain words as ``immediate'' and
executes them, letting them do their own work. (See \emph{Starting
\Forth{}}, Chapter Eleven, ``How to Control the Colon Compiler.'')

The compiler doesn't even ``look for'' semicolon to know when to stop
compiling. Instead it \emph{executes} semicolon, allowing it to do the
work of ending the definition and shutting off the compiler.

There are two tremendous advantages to this approach. First, the
compiler is so simple it can be written in a few lines of code. Second,
there's no limit on the number of compiling words you can add at any
time, simply by making them immediate. Thus, even \Forth{}'s colon
compiler is extensible!

\index{I!Interpreters|(}
\Forth{}'s text interpreter and \Forth{}'s address interpreter also
adhere to this same rule.

The following tip is perhaps the most important in this chapter:


\begin{tip}
Don't write your own interpreter/compiler when you can use \Forth{}'s.
\end{tip}
One class of applications answers a need for a special purpose
language---a self-contained set of commands for doing one particular
thing. An example is a machine-code assembler. Here you have a large
group of commands, the mnemonics, with which you can describe the
instructions you want assembled. Here again, \Forth{} takes a radical
departure from mainstream philosophy.

Traditional assemblers\index{A!Assemblers} are special-purpose
interpreters---that is, they are complicated programs that scan the
assembly-language listing looking for recognized mnemonics such as
\code{ADD}, \code{SUB}, \code{JMP}, etc., and
assemble machine instructions correspondingly. The \Forth{} assembler,
however, is merely a lexicon of \Forth{} words that themselves assemble
machine instructions.

There are many more examples of the special purpose language,
each specific to individual applications. For instance:

%!! begin ordered list
\begin{enumerate}
\item If you're building an Adventure-type game, you'd want to write a
language that lets you create and describe monsters and rooms, etc. You
might create a defining word called \forth{ROOM} to be used like this:

\begin{Code}
ROOM DUNGEON
\end{Code}
Then create a set of words to describe the room's attributes by building
unseen data structures associated with the room:

\begin{Code}
EAST-OF DRAGON-LAIR
WEST-OF BRIDGE
CONTAINING POT-O-GOLD
etc.
\end{Code}
%!! 'etc.' should not be part of the code, should it?
The commands of this game-building language can simply be \Forth{}
words, with \Forth{} as the interpreter.
% typo: originally, ``WORDS'' was upper case. Doesn't make sense here.

\item If you're working with Programmable Array Logic (PAL) devices,
\index{P!Programmable Array Logic (PAL)}
you'd like a form of notation that lets you describe the behavior of
the output pins in logical terms, based on the states of the input
pins. A PAL programmer was written with wonderful simplicity in
\Forth{} by \person{Michael Stolowitz} \cite{stolowitz82}.
\index{S!Stolowitz, Michael}

\item If you must create a series of user menus to drive your application,
you might want to first develop a menu-compiling language. The words of
this new language allow an application programmer to quickly program the
needed menus---while hiding information about how to draw borders, move
the cursor, etc.
\end{enumerate}
All of these examples can be coded in \Forth{} as lexicons, using the
normal \Forth{} interpreter, without having to write a special-purpose
interpreter or compiler.
\index{C!Compilers|)}

\index{M!Moore, Charles|(}
\begin{interview}
\person{Moore}:
\begin{tfquot}
A simple solution is one that does not obscure the problem with
irrelevancies.  It's conceivable that something about the problem requires
a unique interpreter. But every time you see a unique interpreter, it
implies that there is something particularly awkward about the problem.
And that is almost never the case.

If you write your own interpreter, the interpreter is almost certainly the
most complex, elaborate part of your entire application. You have switched
from solving a problem to writing an interpreter.

I think that programmers like to write interpreters. They like to do these
elaborate difficult things. But there comes a time when the world is going
to have to quit programming keypads and converting numbers to binary,
and start solving problems.
\end{tfquot}
\end{interview}%
\index{M!Moore, Charles|)}%
\index{D!Detailed design!forth@\Forth{} syntax|)}%
\index{F!forth@\Forth{}!syntax|)}%
\index{S!Syntax \Forth{}|)}%
\index{I!Interpreters|)}%
\index{L!Lexicons|)}

