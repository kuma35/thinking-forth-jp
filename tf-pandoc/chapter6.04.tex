\section{The Iterative Approach in Implementation}
\index{I!Iterative approach|(}
\index{F!Factoring!iterative approach|(}
Earlier in the book we discussed the iterative approach, paying
particular attention to its impact on the design phase. Now that we're
talking about implementation, let's see how the approach is actually
used in writing code.

\begin{tip}
Work on only one aspect of a problem at a time.
\end{tip}
Suppose we're entrusted with the job of coding a word to draw or erase
a box at a given x--y coordinate. (This is the same problem we
introduced in the section called ``Compile-Time Factoring.'')

At first we focus our attention on the problem of drawing a
box---never mind erasing it. We might come up with this:\program{boxes3}

\begin{Code}
: LAYER   WIDE  0 DO  ASCII * EMIT  LOOP ;
: BOX   ( upper-left-x  upper-left-y -- )
   HIGH  0 DO  2DUP  I +  XY LAYER  LOOP  2DROP ;
\end{Code}
Having tested this to make sure it works correctly, we turn now to the
problem of using the same code to \emph{un}draw a box. The solution is
simple: instead of hard-coding the \forthb{ASCII *} we'd like to change
the emitted character from an asterisk to a blank. This requires the
addition of a variable, and some readable words for setting the
contents of the variable. So:

\begin{Code}
VARIABLE INK
: DRAW   ASCII *  INK ! ;
: UNDRAW   BL  INK ! ;
: LAYER   WIDTH  0 DO  INK @  EMIT  LOOP ;
\end{Code}
The definition of \forth{BOX}, along with the remainder of the application,
remains the same.

This approach allows the syntax

\begin{Code}
( x y ) DRAW BOX
\end{Code}
or

\begin{Code}
( x y ) UNDRAW BOX
\end{Code}
By switching from an explicit value to a variable that contains a
value, we've added a level of indirection. In this case, we've added
indirection ``backwards,'' adding a new level of complexity to the
definition of \forth{LAYER} without substantially lengthening the definition.

By concentrating on one dimension of the problem at a time, you can
solve each dimension more efficiently. If there's an error in your
thinking, the problem will be easier to see if it's not obscured by yet
another untried, untested aspect of your code.

\begin{tip}
Don't change too much at once.
\end{tip}
While you're editing your application---adding a new feature or fixing
something\hy---it's often tempting to go and fix several other things at
the same time. Our advice: Don't.

Make as few changes as you can each time you edit-compile. Be sure to
test the results of each revision before going on. You'd be amazed how
often you can make three innocent modifications, only to recompile and
have nothing work!

Making changes one at a time ensures that when it stops working, you
know why.

\begin{tip}
Don't try to anticipate ways to factor too early.
\end{tip}%
\index{A!Arrays|(}
Some people wonder why most \Forth{} systems don't include the
definition word \forth{ARRAY}. This rule is the reason.
\begin{interview}
\person{Moore}:\index{M!Moore, Charles|(}
\begin{tfquot}
I often have a class of things called arrays. The simplest array
merely adds a subscript to an address and gives you back an
address. You can define an array by saying
\begin{Code}
CREATE X   100 ALLOT
\end{Code}
then saying
\begin{Code}
X +
\end{Code}
Or you can say
\begin{Code}
: X   X + ;
\end{Code}
One of the problems that's most frustrating for me is knowing whether
it's worth creating a defining word for a particular data structure.
Will I have enough instances to justify it?

I rarely know in advance if I'm going to have more than one array. So
I don't define the word \forth{ARRAY}.

After I discover I need two arrays, the question is marginal.

If I need three then it's clear. Unless they're different. And odds
are they will be different. You may want it to fetch it for you. You
may want a byte array, or a bit array. You may want to do bounds
checking, or store its current length so you can add things to the
end.

I grit my teeth and say, ``Should I make the byte array into a cell
array, just to fit the data structure into the word I already have
available?''

The more complex the problem, the less likely it will be that you'll
find a universally applicable data structure. The number of instances
in which a truly complex data structure has found universal use is
very small. One example of a successful complex data structure is the
\Forth{} dictionary. Very firm structure, great versatility. It's used
everywhere in \Forth{}. But that's rare.

If you choose to define the word \forth{ARRAY}, you've done a
decomposition step. You've factored out the concept of an array from
all the words you'll later back in. And you've gone to another level
of abstraction.

Building levels of abstraction is a dynamic process, not one you can
predict.
\end{tfquot}\index{M!Moore, Charles|)}
\end{interview}%
\index{A!Arrays|)}
\begin{tip}
Today, make it work. Tomorrow, optimize it.
\end{tip}
\begin{interview}
Again \person{Moore}.\index{M!Moore, Charles|(} On the day of this
interview, \person{Moore} had been completing work on the design of a
board-level \Forth{} computer, using commercially available ICs. As
part of his toolkit for designing the board, he created a simulator in
\Forth{}, to test the board's logic:

\begin{tfquot}
This morning I realized I've been mixing the descriptions of the chips
with the placement of the chips on the board. This perfectly
convenient for my purposes at the moment, but when I come up with
another board that I want to use the same chips for, I have arranged
things very badly.

I should have factored it with the descriptions here and the uses
there. I would then have had a chip description language. Okay. At the
time I was doing this I was not interested in that level of
optimization.

Even if the thought had occurred to me then, I probably would have
said, ``All right, I'll do that later,'' then gone right ahead with
what I was doing. Optimization wasn't the most important thing to me
at the time.

Of course I try to factor things well. But if there doesn't seem to be
a good way to do something, I say, ``Let's just make it work.''

My motivation isn't laziness, it's knowing that there are other things
coming down the pike that are going to affect this decision in ways I
can't predict. Trying to optimize this now is foolish. Until I get the
whole picture in front of me, I can't know what the optimum is.
\end{tfquot}\index{M!Moore, Charles|)}
\end{interview}
The observations in this section shouldn't contradict what's been said
before about information hiding and about anticipating elements that
may change. A good programmer continually tries to balance the expense
of building-in changeability against the expense of changing things
later if necessary.

These decisions take experience. But as a general rule:

\begin{tip}
Anticipate things-that-may-change by organizing information, not by
adding complexity. Add complexity only as necessary to make the
current iteration work.
\end{tip}
\index{I!Iterative approach|)}
\index{F!Factoring!iterative approach|)}

\subsection{Summary}
In this chapter we've discussed various techniques and criteria for
factoring. We also examined how the iterative approach applies to the
implementation phase.
\index{F!Factoring|)}

