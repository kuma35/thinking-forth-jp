\section{The Stylish Stack}%
\index{D!Data stacks!concept of|(}

The simplest way for \Forth{} words to pass arguments to each other is
via the stack. The process is ``simple'' because all the work of pushing
and popping values to and from the stack is implicit.

\begin{interview}
\person{Moore}:\index{M!Moore, Charles|(}

\begin{tfquot}
The data stack uses this idea of ``hidden information.'' The arguments
being passed between subroutines are not explicit in the calling sequence.
The same argument might ripple through a whole lot of words quite invisibly,
even below the level of awareness of the programmer, simply because it
doesn't have to be referred to explicitly.
\end{tfquot}\index{M!Moore, Charles|)}
\end{interview}
One important result of this approach: Arguments are unnamed. They
reside on the stack, not in named variables. This effect is one of the
reasons for \Forth{}'s elegance. At the same time it's one of the reasons
badly written \Forth{} code can be unreadable. Let's explore this
paradox.

The invention of the stack is analogous to that of pronouns in
English. Consider the passage:

%!! indent paragraph
\begin{tfquot}
Take this gift, wrap it in tissue paper and put it in a box.
\end{tfquot}
Notice the word ``gift'' is mentioned only once. The gift is referred to
henceforth as ``it.''

The informality of the ``it'' construct makes English more readable
(provided the reference is unambiguous). So with the stack, the implicit
passing of arguments makes code more readable. We emphasize the
\emph{processes}, not the \emph{passing of arguments} to the processes.

Our analogy to pronouns suggests why bad \Forth{} can be so unreadable.
The spoken language gets confusing when too many things are
referred to with pronouns.

\begin{tfquot}
Take off the wrapping and open the box. Remove the gift and throw it
away.
\end{tfquot}
The problem with this passage is that we're using ``it'' to refer to too
many things at once. There are two solutions to this error. The easiest
solution is to supply a real name instead of ``it'':

\begin{tfquot}
Remove the wrapping and open the box. Take out the gift and throw
\emph{the box} away.
\end{tfquot}
Or we can introduce the words ``former'' and ``latter.'' But the best
solution is to redesign the passage:

\begin{tfquot}
Remove the wrapping and open the present. Throw away the box.
\end{tfquot}
So in \Forth{} we have analogous observations:

\begin{tip}
Simplify code by using the stack. But don't stack too deeply within any
single definition. Redesign, or, as a last resort, use a named variable.
\end{tip}
Some newcomers to \Forth{} view the stack the way a gymnast views a
trampoline: as a fun place to bounce around on. But the stack is meant for
data-passing, not acrobatics.

\index{D!Data stacks!depth of|(}%
So how deep is ``too deep?'' Generally, three elements on the stack is
the most you can manage within a single definition. (In double-length
arithmetic, each ``element'' occupies two stack positions but is logically
treated as a single element by operators such as \forthb{2DUP}, \forthb{2OVER},
etc.)

In your ordinary lexicon of stack operators, \forthb{ROT} is the only one
that gives you access to the third stack item. Aside from \forthb{PICK} and
\forthb{ROLL} (which we'll comment on soon), there's no easy way to get at
anything below that.

To stretch our analogy to the limit, perhaps three elements on the
stack corresponds to the three English pronouns ``this,'' ``that,'' and
``t'other.''%
\index{D!Data stacks!concept of|)}

\subsection{Redesign}

Let's witness a case where a wrong-headed approach leads to a messy
stack problem. Suppose we're trying to write the definition of \forth{+THRU}
(see \Chap{5}, ``Listing Organization'' section, ``Relative Loading''
subsection). We've decided that our loop body will be

\begin{Code}
...  DO  I LOAD  LOOP ;
\end{Code}
that is, we'll put \forthb{LOAD} in a loop, then arrange for the index and
limit to correspond to the absolute screens being loaded.

On the stack initially we have:

\begin{Code}
lo hi
\end{Code}
where ``lo'' and ``hi'' are the \emph{offsets} from \forthb{BLK}.

We need to permute them for \forthb{DO}, like this:

\begin{Code}
hi+1+blk lo+blk
\end{Code}
Our biggest problem is adding the value of \forth{BLK} to both offsets.

We've already taken a wrong turn but we don't know it yet. So let's
proceed. We try:

\begin{Code}
lo hi
               BLK @
lo hi blk
               SWAP
lo blk hi
               OVER
lo blk hi blk
               +
lo blk hi+blk
               1+
lo blk hi+blk+1
               ROT ROT
hi+blk+1 lo blk
               +
hi+blk+1 lo+blk
\end{Code}
We made it, but what a mess!

If we're gluttons for punishment, we might make two more stabs at
it arriving at:

\begin{Code}
BLK @  DUP ROT + 1+  ROT ROT +
\end{Code}
and

\begin{Code}
BLK @  ROT OVER +  ROT ROT + 1+  SWAP
\end{Code}
All three sequences do the same thing, but the code seems to be getting
blurrier, not better.

With experience we learn to recognize the combination \forth{ROT ROT} as
a danger sign: the stack is too crowded. Without having to work out the
alternates, we recognize the problem: once we make two copies of ``blk,''
we have four elements on the stack.

\medbreak
At this point, the first resort is usually the return stack:

\begin{Code}
BLK @  DUP >R  + 1+  SWAP R> +
\end{Code}
(See ``The Stylish Return Stack,'' coming up next.) Here we've \forthb{DUP}ed
``blk,'' saving one copy on the return stack and adding the other copy to
``hi.''

Admittedly an improvement. But readable?

Next we think, ``Maybe we need a named variable.'' Of course, we
have one already: \forthb{BLK}. So we try:

\begin{Code}
BLK @  + 1+  SWAP BLK @ +
\end{Code}
Now it's more readable, but it's still rather long, and redundant too.
\forthb{BLK @ +} appears twice.

``\forthb{BLK @ +}''? That sounds familiar. Finally our neurons connect.

We look back at the source for \forth{+LOAD} just defined:

\begin{Code}
: +LOAD  ( offset -- )  BLK @ +  LOAD ;
\end{Code}
This word, \forth{+LOAD}, should be doing the work. All we have to write is:

\begin{Code}
: +THRU  ( lo hi )  1+ SWAP  DO  I +LOAD  LOOP ;
\end{Code}
We haven't created a more efficient version here, because the work of
\forthb{BLK @ +} will be done on every pass of the loop. But we have
created a cleaner, conceptually simpler, and more readable piece of code.
In this case, the inefficiency is unnoticeable because it only occurs as
each block is loaded.

Redesigning, or rethinking the problem, was the path we should
have taken as soon as things got ugly.%
\index{D!Data stacks!depth of|)}

\subsection{Local Variables}%
\index{L!Local variables|(}%
\index{D!Data stacks!local variables and,|(}%
\index{D!Data stacks!data@vs. data structures|(}%
\index{D!Data structures:!data@vs. data stacks|(}
\index{V!Variables:!local|(}%

Most of the time problems can be arranged so that only a few arguments
are needed on the stack at any one time. Occasionally, however, there's
nothing you can do.

Here's an example of a worst case. Assume you have a word called
\forth{LINE} which draws a line between any two points, specified as
coordinates in this order:

\begin{Code}
( x1 y1 x2 y2)
\end{Code}
where $x_1,y_1$ represent the $x,y$ coordinates for the one end-point, and
$x_2,y_2$ represent the opposite end-point.

Now you have to write a box-drawing word called \forth{[BOX]} which
takes four arguments in this order:

\begin{Code}
( x1 y1 x2 y2)
\end{Code}
where x1 y1 represent the $x,y$ coordinates for the upper left-hand corner
of the box, and x2 y2 represent the lower right-hand corner coordinates.
Not only do you have four elements on the stack, they each have to
be referred to more than once as you draw lines from point to point.

Although we're using the stack to get the four arguments, the algorithm
for drawing a box doesn't lend itself to the nature of the stack. If you're
in a hurry, it would probably be best to take the easy way out:

\begin{Code}
VARIABLE TOP         ( y coordinates top of box)
VARIABLE LEFT        ( x     "       left side)
VARIABLE BOTTOM      ( y     "       bottom)
VARIABLE RIGHT       ( x     "       right side)
: [BOX]   ( x1 y1 x2 y2)   BOTTOM !  RIGHT !  TOP !  LEFT !
   LEFT @ TOP @  RIGHT @ TOP @  LINE
   RIGHT @ TOP @  RIGHT @ BOTTOM @  LINE
   RIGHT @ BOTTOM @  LEFT @ BOTTOM @  LINE
   LEFT @ BOTTOM @  LEFT @ TOP @  LINE ;
\end{Code}
What we've done is create four named variables, one for each coordinate.
The first thing \forth{[BOX]} does is fill these variables with the
arguments from the stack. Then the four lines are drawn, referencing the
variables.  Variables such as these that are used only within a definition
(or in some cases, within a lexicon) are called ``local variables.''

I've been guilty many times of playing hotshot, trying to do as
much as possible on the stack rather than define a local variable. There
are three reasons to avoid this cockiness.

First, it's a pain to code that way. Second, the result is unreadable.
Third, all your work becomes useless when a design change becomes
necessary, and the order of two arguments changes on the stack. The
\forthb{DUP}s, \forthb{OVER}s and \forthb{ROT}s weren't really solving the
problem, just jockeying things into position.

With this third reason in mind, I recommend the following:

\begin{tip}
Especially in the design phase, keep on the stack only the arguments you're
using immediately. Create local variables for any others. (If necessary,
eliminate the variables during the optimization phase.)
\end{tip}
Fourth, if the definition is extremely time-critical, those tricky stack
manipulators, (e.g., \forthb{ROT ROT}) can really eat up clock cycles.
Direct access to variables is faster.

If it's \emph{really} time-critical, you may need to convert to assembler
anyway. In this case, all your stack problems fly out the door, because
all your data will be referenced either in registers or indirectly through
registers. Luckily, the definitions with the messiest stack arguments are
often the ones written in code. Our \forth{[BOX]} primitive is a case in
point.  \forthb{CMOVE>} is another.

The approach we took with \forth{[BOX]} certainly beats spending half an
hour juggling items on the stack, but it is by no means the best solution.
What's nasty about it is the expense of creating four named variables,
headers and all, solely for use within this one routine.

(If you're target compiling an application that will not require headers
in the dictionary, the only loss will be the 8 bytes in RAM for the
variables. In \Forth{} systems of the future, headers may be separated
into other pages of memory anyway; again the loss will be only 8 bytes.)
Let me repeat: This example represents a worst-case situation, and occurs
rarely in most \Forth{} applications. If words are well-factored, then
each word is designed to do very little. Words that do little generally
require few arguments.

In this case, though, we are dealing with two points each represented
by two coordinates.

Can we change the design? First, \forth{LINE} may be \emph{too} primitive a
primitive. It requires four arguments because it can draw lines between
any two points, diagonally, if necessary.

In drawing our box, we may only need perfectly vertical and horizontal
lines. In this case we can write the more powerful, but less specific,
words \forth{VERTICAL} and \forth{HORIZONTAL} to draw these lines. Each
requires only \emph{three} arguments: the starting position's x and y, and
the length. This factoring of function simplifies the definition of
\forth{[BOX].}

Or we might discover that this syntax feels more natural to the
user:

\begin{Code}
10 10 ORIGIN! 30 30 BOX
\end{Code}
where \forth{ORIGIN!} sets a two-element pointer to the ``origin,'' the
place where the box will start (the upper left-hand corner). Then
``\forth{30 30 BOX}'' draws a box 30 units high and 30 units wide,
relative to the origin.

This approach reduces the number of stack arguments to \forth{BOX} as
part of the design.

\begin{tip}
When determining which arguments to handle via data structures rather
than via the stack, choose the arguments that are the more permanent or
that represent a current state.
\end{tip}%
\index{L!Local variables|)}%
\index{V!Variables:!local|)}%
\index{D!Data stacks!local variables and,|)}%
\index{D!Data stacks!data@vs. data structures|)}%
\index{D!Data structures:!data@vs. data stacks|)}

\subsection{On PICK and ROLL}%
\index{D!Data stacks!PICK and ROLL|(}%
\index{P!PICK|(}%
\index{R!ROLL|(}%

Some folks like the words \forthb{PICK} and \forthb{ROLL}. They use these
words to access elements from any level on the stack. We don't recommend
them.  For one thing, \forthb{PICK} and \forthb{ROLL} encourage the
programmer to think of the stack as an array, which it is not. If you have
so many elements on the stack that you need \forthb{PICK} and
\forthb{ROLL}, those elements should be in an array instead.

Second, they encourage the programmer to refer to arguments that
have been left on the stack by higher-level, calling definitions without
being explicitly \emph{passed} as arguments. This makes the definition
dependent on other definitions. That's unstructured---and dangerous.

Finally, the position of an element on the stack depends on what's
above it, and the number of things above it can change constantly. For
instance, if you have an address at the fourth stack position down, you can
write

\begin{Code}
4 PICK @
\end{Code}
to fetch its contents. But you must write

\begin{Code}
( n) 5 PICK !
\end{Code}
because with ``$n$'' on the stack, the address is now in the fifth position.
Code like this is hard to read and harder to modify.%
\index{D!Data stacks!PICK and ROLL|)}%
\index{P!PICK|)}%
\index{R!ROLL|)}%

\subsection{Make Stack Drawings}%
\index{D!Data stacks!drawings|(}

When you do have a cumbersome stack situation to solve, it's best to work
it out with paper and pencil. Some people even make up forms, such as the
one in \Fig{fig7-1}. Done formally like this (instead of on the back of your
phone bill), stack commentaries serve as nice auxiliary documentation.

\subsection{Stack Tips}

\begin{tip}
Make sure that stack effects balance out under all possible control flows.
\end{tip}
In the stack commentary for \forthb{CMOVE>} in \Fig{fig7-1}, the inner
brace represents the contents of the \forthb{DO }\forthb{LOOP}. The stack
depth upon exiting the loop is the same as upon entering it: one element.
Within the outer braces, the stack result of the \forthb{IF} clause is the
same as that of the \forthb{ELSE} clause: one element left over. (What
that leftover element represents doesn't matter, as symbolized by the
``x'' next to \forthb{THEN}.)

\wepsfiga{fig7-1}{Example of a stack commentary.}

%!! include \Fig{fig7-1} here

\begin{tip}
When doing two things with the same number, perform the function that
will go underneath first.
\end{tip}
For example:

\begin{Code}
: COUNT  ( a -- a+1 # )  DUP C@  SWAP 1+  SWAP ;
\end{Code}
(where you first get the count) is more efficiently written:

\begin{Code}
: COUNT  ( a -- a+1 # )  DUP 1+  SWAP C@ ;
\end{Code}
(where you first compute the address).
\goodbreak

\begin{tip}
Where possible, keep the number of return arguments the same in all
possible cases.
\end{tip}%
\index{E!Error-code|(}
You'll often find a definition which does some job and, if something goes
wrong, returns an error-code identifying the problem. Here's one way
the stack interface might be designed:

\begin{Code}
( -- error-code f | -- t)
\end{Code}
If the flag is true, the operation was successful. If the flag is false,
it was unsuccessful and there's another value on the stack to indicate the
nature of the error.

You'll find stack manipulation easier, though, if you redesign the interface
to look like this:

\begin{Code}
( -- error-code | O=no-error)
\end{Code}
One value serves both as a flag and (in case of an error) the error code.
Note that reverse-logic is used; non-zero indicates an error. You can use
any values for the error codes except zero.%
\index{E!Error-code|)}%
\index{D!Data stacks!drawings|)}

