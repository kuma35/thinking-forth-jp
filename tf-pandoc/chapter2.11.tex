\section{Reviewing the Conceptual Model}%
\index{A!Analysis!conceptual model|(}%
\index{C!Conceptual model|(}

The final box on our iterative analytic wheel is labeled ``Show Model
to Customer.'' With the tools we've outlined in this chapter, this job
should be easy to do.

In documenting the requirements specification, remember that specs are
like snowmen. They may be frozen now, but they shift, slip, and melt
away when the heat is on. Whether you choose data-flow diagrams or
straight \Forth{} pseudocode, prepare yourself for the great thaw by
remembering to apply the concepts of limited redundancy.

Show the documented conceptual model to the customer. When the
customer is finally satisfied, you're ready for the next big step: the
design!%
\index{A!Analysis!conceptual model|)}%
\index{C!Conceptual model|)}

\begin{references}{9}
\bibitem{harris81} \person{Kim Harris}, ``The \Forth{} Philosophy,''
  \emph{Dr.\@ Dobb's Journal,} Vol. 6, Iss. 9, No. 59 (Sept. 81),
  pp. 6-11.
\bibitem{weinberg80} \person{Victor Weinberg}, \emph{Structured Analysis,}
  Englewood Cliffs, N.J.: Prentice-Hall, Inc., 1980.
\bibitem{stuart80} \person{LaFarr Stuart}, ``LaFORTH,''
  \emph{1980 FORML Proceedings,} p. 78.
\bibitem{brooks75} \person{Frederick P. Brooks}, Jr., \emph{The Mythical
  Man-Month,} Reading, Massachusetts, Addison-Wesley, 1975.
\end{references}

% end of chapter two
