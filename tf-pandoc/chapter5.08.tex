\section{Summary}
Maintainability requires readability.  In this chapter we've enumerated
various ways to make a source listing more readable.  We've assumed a
%Page 171 in first edition.
policy of making our code as self-documenting as possible.  Techniques
include listing organization, spacing and indenting, commenting, name
choices, and special words that enhance clarity.

We've mentioned only briefly auxiliary documentation, which includes
all documentation apart from the listing itself.  We won't discuss
auxiliary documentation further in this volume, but it remains an
integral part of the software development process.%
\index{I!Implementation|)}

\begin{references}{9}
\bibitem{stevenson81} \person{Gregory Stevenson}, ``Documentation Priorities,''
\emph{1981 FORML Conference Proceedings,} p. 401.
\bibitem{lee81} \person{Joanne Lee}, ``Quality Assurance in a \Forth{}
Environment,'' (Appendix A), \emph{1981 FORML Proceedings,} p. 363.
\bibitem{dijkstra82} \person{Edsger W. Dijkstra}, \emph{Selected Writings on
Computing: A Personal Perspective,} New York, Springer Verlag, Inc.,
1982.
\bibitem{laxen} \person{Henry Laxen}, ``Choosing Names,'' \emph{\Forth{} Dimensions,}
vol. 4, no.\ 4, \Forth{} Interest Group.
\end{references}

%Page 172 in first edition.
