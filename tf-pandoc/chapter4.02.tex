\section{Interview with a Software Inventor}

\index{B!Burgess, Donald A.|(}
\begin{interview}
\person{Donald A. Burgess}, owner and president of Scientek Instrumentation,
Inc.:

%!! begin indented paragraph
\begin{tfquot}
I have a few techniques I've found useful over the years in designing
anything, to keep myself flexible.
My first rule is, ``Nothing is impossible.''
My second rule is, ``Don't forget, the object is to make a buck.''

First examine the problem, laying out two or three approaches on paper.
Then try the most appealing one, to see if it works. Carry it through. Then
deliberately go all the way back to the beginning, and start over.

Starting over has two values. First, it gives you a fresh approach. You
either gravitate back to the way you started, or the way you started
gravitates toward the new way.

Second, the new approach may show all kinds of powerful possibilities. Now
you have a benchmark. You can look at both approaches and compare the
advantages of both. You're in a better position to judge.

Getting stuck comes from trying too hard to follow a single approach.
Remember to say, ``I want this kumquat crusher to be different. Let's
reject the traditional design as not interesting. Let's try some crazy
ideas.''

The best thing is to start drawing pictures. I draw little men. That keeps
it from looking like ``data'' and interfering with my thinking process. The
human mind works exceptionally well with analogies. Putting things in
context keeps you from getting stuck within the confines of any language,
even \Forth{}.

When I want to focus my concentration, I draw on little pieces of paper.
When I want to think in broad strokes, to capture the overall flow, I draw
on great big pieces of paper. These are some of the crazy tricks I use to keep
from getting stagnant.

When I program in \Forth{}, I spend a day just dreaming, kicking around
ideas. Usually before I start typing, I sketch it out in general terms. No
code, just talk. Notes to myself.

Then I start with the last line of code first. I describe what I would like
to do, as close to English as I can. Then I use the editor to slide this
definition towards the bottom of the screen, and begin coding the internal
words. Then I realize that's a lousy way to do it. Maybe I split my top word
into two and transfer one of them to an earlier block so I can use it earlier.
I run the hardware if I have it; otherwise I simulate it.

\Forth{} requires self-discipline. You have to stop diddling with the
keyboard. \Forth{} is so willing to do what I tell it to, I'll tell it to
do all kinds of ridiculous things that have nothing to do with where I'm
trying to go. At those times I have to get away from the keyboard.

\Forth{} lets you play. That's fine, chances are you'll get some ideas. As
long as you keep yourself from playing as a habit. Your head is a whole lot
better than the computer for inventing things.
\end{tfquot}
\end{interview}%
\index{B!Burgess, Donald A.|)}

