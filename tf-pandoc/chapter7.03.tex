\section{The Problem With Variables}%
\index{V!Variables:!problem with|(}

Although we handle data of immediate interest on the stack, we depend
on much information tucked away in variables, ready for recurring access.
A piece of code can change the contents of a variable without
necessarily having to know anything about how that data will be used,
who will use it, or when and if it will be used. Another piece of code can
fetch the contents of a variable and use it without knowing where that
value came from.

For every word that pushes a value onto the stack, another word
must consume that value. The stack gives us point-to-point communication,
like the post office.

Variables, on the other hand, can be set by any command and accessed
any number of times---or not at all---by any command. Variables
are available for anyone who cares to look---like graffiti.

Thus variables can be used to reflect the current state of affairs.

Using currentness can simplify problems. In the Roman numeral example
of \Chap{4}, we used the variable \forth{COLUMN\#} to represent the
current decimal-place; the words \forth{ONER}, \forth{FIVER}, and
\forth{TENER} depended on this information to determine which type of
symbol to display. We didn't have to specify both descriptions every
time, as in \forth{TENS ONER}, \forth{TENS FIVER}, etc.

On the other hand, currentness adds a new level of complexity. To make
something current we must first define a variable or some type of data
structure. We also must remember to initialize it, if there's any chance
that part of our code will refer to it before another part has had a
chance to set it.

A more serious problem with variables is that they are not ``reentrant.''
On a multi-tasked \Forth{} system, each task which requires local
variables must have its own copies. \Forth{}'s \forthb{USER} variables
serve this purpose. (See \emph{Starting \Forth{}}, Chapter Nine,
``\Forth{} Geography.'')

Even within a single task, a definition that refers to a variable is
harder to test, verify, and reuse in a different situation than one in
which arguments are passed via the stack.

Suppose we are implementing a word-processor editor. We need a routine
that calculates the number of characters between the current cursor
position and the previous carriage-return/line-feed sequence. So we write
a word that employs a \forthb{DO }\forthb{LOOP} starting at the current
position (\forth{CURSOR @}) and ending at the zeroth position, searching
for the line feed character.

Once the loop has found the character sequence, we subtract its
relative address from our current cursor position

\begin{Code}
its-position CURSOR @  SWAP -
\end{Code}
to determine the distance between them.

Our word's stack effect is:

\begin{Code}
( -- distance-to-previous-cr/lf)
\end{Code}
But in later coding we find we need a similar word to compute the distance
from an arbitrary character---\emph{not} the current cursor position---to
the first previous line-feed character. We end up factoring out the
``\forth{CURSOR @}'' and allowing the starting address to be passed as an
argument on the stack, resulting in:

\begin{Code}
( starting-position -- distance-to-previous-cr/lf)
\end{Code}
By factoring-out the reference to the variable, we made the definition
more useful.

\begin{tip}
Unless it involves cluttering up the stack to the point of unreadability,
try to pass arguments via the stack rather than pulling them out of
variables.
\end{tip}

\begin{interview}
\index{K!Kogge, Peter|(}
\person{Kogge}:
\begin{tfquot}
Most of the modularity of \Forth{} comes from designing and treating
\Forth{} words as ``functions'' in the mathematical sense. In my
experience a \Forth{} programmer usually tries quite hard to avoid
defining any but the most essential global variables (I have a friend who
has the sign ``Help stamp out variables'' above his desk), and tries to
write words with what is called ``referential transparency,'' i.e., given
the same stack inputs a word will always give the same stack outputs
regardless of the more global context in which it is executed.

In fact this property is exactly what we use when we test words in
isolation.  Words that do not have this property are significantly harder
to test. In a sense a ``named variable'' whose value changes frequently is
the next worst thing to the now ``forbidden'' GOTO.
\end{tfquot}
\index{K!Kogge, Peter|)}
\end{interview}

\wepsfigp{img7-211}{``Shot from a cannon on a fast-moving train,
hurtling between the blades of a windmill, and expecting to grab a
trapeze dangling from a hot-air balloon\dots{} I told you Ace, there were
too many variables!''}

\noindent Earlier we suggested the use of local variables especially
during the design phase, to eliminate stack traffic. It's important to
note that in doing so, the variables were referred to only within the one
definition.  In our example, \forth{[BOX]} receives four arguments from
the stack and immediately loads them into local variables for its own use.
The four variables are not referred to outside of this definition, and the
word behaves safely as a function.

Programmers unaccustomed to a language in which data can be passed
implicitly don't always utilize the stack as fully as they should.
\person{Michael Ham}\index{H!Ham, Michael}
suggests the reason may be that beginning \Forth{} users
don't trust the stack \cite{ham83}. He admits to initially feeling
safer about storing values into variables than leaving them on the
stack. ``No telling \emph{what} might happen with all that thrashing
about on the stack,'' he felt.

It took some time for him to appreciate that ``if words keep properly
to themselves, using the stack only for their expected input and output
and cleaning up after themselves, they can be looked upon as sealed
systems \dots{} I could put the count on the stack at the beginning of the
loop, go through the complete routine for each group, and at the end the
count would emerge, back on top of the stack, not a hair out of place.''%
\index{V!Variables:!problem with|)}

