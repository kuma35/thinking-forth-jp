\section{Data Structures}%
\index{D!Data structures:!operators}%

Still another defining word is \forthb{CONSTANT},\index{C!CONSTANT} which
is used like this:
\begin{Code}
17 CONSTANT SEVENTEEN
\end{Code}
The new word \forth{SEVENTEEN} can now be used in place of the actual
number 17.

{\sloppy
The defining word \forth{VARIABLE}\index{V!VARIABLE} creates a location
for temporary data. \forth{VARIABLE} is used like this:
\begin{Code}
VARIABLE BANANAS
\end{Code}
This reserves a location which is identified by the name \forth{BANANAS}.}

Fetching the contents of this location is the job of the word \forthb{@}
(pronounced ``fetch'').  For instance,
\begin{Code}
BANANAS @
\end{Code}
fetches the contents of the variable \forth{BANANAS}.  Its counterpart is
\forthb{!} (pronounced ``store''), which stores a value into the location,
as in:
\begin{Code}
100 BANANAS !
\end{Code}
\Forth{} also provides a word to increment the current value by the given
value; for instance, the phrase
\begin{Code}
2 BANANAS +!
\end{Code}
increments the count by two, making it 102.

\Forth{} provides many other
data structure operators\index{D!Data structures:!operators}, but more
importantly, it provides the tools necessary for the programmer to
create any type of data structure needed for the application.%
\index{D!Dictionary:!defined|)}%

