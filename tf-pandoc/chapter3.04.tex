\section{Designing and Maintaining a Traditional Application}

Let's pretend we haven't studied the Tiny Editor problem yet, and
we're back with a minimal set of specs. We'll also start with our
initial assumption, that we can refresh the display by retyping the
entire field after each keystroke.

According to the dictum of top-down design, let's take the
widest-angle view possible and examine the problem.  \Fig{fig3-3}
depicts the program in its simplest terms. Here we've realized that
the editor is actually a loop which keeps getting keystrokes and
performing some editing function, until the user presses the return
key.

\wepsfigb{fig3-3}{The traditional approach: view from the top.}
Inside the loop we have three modules: getting a character from the
keyboard, editing the data, and finally refreshing the display to match
the data.

Clearly most of the work will go on inside ``Process a Keystroke.''

Applying the notion of successive refinement, \fig{fig3-4} shows the
editor problem redrawn with ``Process a Keystroke'' expanded. We find it
takes several attempts before we arrive at this configuration. Designing
this level forces us to consider many things at once that we had deferred
till later in the previous try.

\wepsfigt{fig3-4}{A structure for ``Process a Keystroke.''}

For instance, we must determine all the keys that might be pressed.
More significantly, we must consider the problem of ``insert mode.'' This
realization forces us to invent a flag called \forth{INSERT-MODE} which gets
toggled by the ``Ctrl I'' key. It's used within several of the structural
lines to determine how to process a type of key.

A second flag, called \forth{ESCAPE}, seems to provide a nice structured
way of escaping the editor loop if the user presses the return key while
not in insert mode.

Having finished the diagram, we're bothered by the multiple tests
for Insert Mode. Could we test for Insert Mode once, at the beginning?
Following this notion, we draw yet another chart (\fig{fig3-5}).

As you can see, this turns out even more awkward than the first
figure. Now we're testing for each key twice. It's interesting though,
how the two structures are totally different, yet functionally
equivalent. It's enough to make one wonder whether the control
structure is terribly relevant to the problem.

\wepsfigt{fig3-5}{Another structure for ``Process a Keystroke.''}

Having decided on the first structure, we've finally arrived at the
most important modules---the ones that do the work of overwriting,
inserting, and deleting. Take another look at our expansion of
``Process a Character'' in \fig{fig3-4}. Let's consider just one of the
seven possible execution paths, the one that happens if a printable
character is pressed.

In \fig{fig3-6}(a) we see the original structural path for a printable
character.

Once we figure out the algorithms for overwriting and inserting
characters, we might refine it as shown in \fig{fig3-6}(b). But look at
that embarrassing redundancy of code (circled portions). Most
competent structured programmers would recognize that this redundancy
is unnecessary, and change the structure as shown in \fig{fig3-6}(c).
Not too bad so far, right?

\subsection{Change in Plan}

Okay, everyone, now act surprised. We've just been told that this
application won't run on a memory-mapped display. What does this
change do to our design structure?

\wepsfigt{fig3-6}{The same section, ``refined'' and ``optimized.''}

Well, for one thing it destroys ``Refresh Display'' as a separate
module. The function of ``Refresh Display'' is now scattered among the
various structural lines inside ``Process a Keystroke.'' The structure of
our entire application has changed. It's easy to see how we might have
spent weeks doing top-down design only to find we'd been barking down
the wrong tree.

What happens when we try to change the program? Let's look again
at the path for any printable character.

\fig{fig3-7} (a) shows what happens to our first-pass design when we
add refresh. Part (b) shows our ``optimized'' design with the refresh
modules expanded. Notice that we're now testing the Insert flag twice
within this single leg of the outer loop.

But worse, there's a bug in this design. Can you find it?

In both cases, overwriting and inserting, the pointer is incremented
\emph{before} the refresh. In the case of overwrite, we're displaying
the new character in the wrong position. In the case of insert, we're
typing the remainder of the line but not the new character.

Granted, this is an easy problem to fix. We need only move the refresh
modules up before ``Increment Pointer.'' The point here is: How did we
miss it? By getting preoccupied with control flow structure, a
superficial element of program design.

\wepsfigt{fig3-7}{Adding refresh.}

In contrast, in our design by components the correct solution fell
out naturally because we ``used'' the refresh component inside the editing
component. Also we used \forth{OVERWRITE} inside \forth{INSERT}.

By decomposing our application into components which use one another,
we achieved not only \emph{elegance} but a more direct path to
\emph{correctness}.\program{editor2}

