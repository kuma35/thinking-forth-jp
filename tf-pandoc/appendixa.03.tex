\section{The Stack}
\index{D!Data stacks!defined|(}
In \Forth{}, variables and arrays are used for saving values that may be
required by many other routines and/or at unpredictable times.  They are
\emph{not} used for the local passing of data between the definitions.
For this, \Forth{} employs a much simpler mechanism: the data stack.

When you type a number, it goes on the stack.  When you invoke a word
which has numeric input, it will take it from the stack.  Thus the phrase
\begin{Code}
17 SPACES
\end{Code}
will display seventeen blanks on the current output device.  ``17'' pushes
the binary value 17 onto the stack; the word \forthb{SPACES} consumes it.

A constant also pushes its value onto the stack; thus the phrase:
\begin{Code}
SEVENTEEN SPACES
\end{Code}
has the same effect.

The stack operates on a ``last-in, first-out'' (LIFO) basis.  This means
that data can be passed between words in an orderly, modular way,
consistent with the nesting of colon definitions.

For instance, a definition called \forth{GRID} might invoke the phrase
\forth{17 SPACES}.  This temporary activity on the stack will be
transparent to any other definition that invokes \forth{GRID} because the
value placed on the stack is removed before the definition of \forth{GRID}
ends.  The calling definition might have placed some numbers of its own on
the stack prior to calling \forth{GRID}.  These will remain on the stack,
unharmed, until \forth{GRID} has been executed and the calling definition
continues.
\index{D!Data stacks!defined|)}

