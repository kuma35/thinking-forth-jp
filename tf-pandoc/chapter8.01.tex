\section{What's So Bad about Control Structures?}%
\index{C!Control structure minimization!reasons for|(}

Before we begin reeling off our list of tips, let's pause to examine why
conditionals should be avoided in the first place.

The use of conditional structures adds complexity to your code. The
more complex your code is, the harder it will be for you to read and to
maintain. The more parts a machine has, the greater are its chances of
breaking down. And the harder it is for someone to fix.

%!! horizontal line
\begin{interview}
\person{Moore}\index{M!Moore, Charles|(} tells this story:
\begin{tfquot}
I recently went back to a company we had done some work for several years
ago. They called me in because their program is now five years old, and
it's gotten very complicated. They've had programmers going in and
patching things, adding state variables and conditionals. Every statement
that I recall being a simple thing five years ago, now has gotten very
complicated.  ``If this, else if this, else if this'' \dots{} and then the
simple thing.

Reading that statement now, it's impossible for me to figure out what it's
doing and why. I'd have to remember what each variable indicated, why it
was relevant in this case, and then what was happening as a consequence of
it---or not happening.

It started innocently. They had a special case they needed to worry about.
To handle that special case, they put a conditional in one place. Then they
discovered that they also needed one here, and here. And then a few more.
Each incremental step only added a little confusion to the program. Since
they were the programmers, they were right on top of it.

The net result was disastrous. In the end they had half a dozen flags.
Test this one, reset it, set that one, and so on. As a result of this
condition, you knew you had other conditions coming up you had to look out
for. They created the logical equivalent of spaghetti code in spite of the
opportunity for a structured program.

The complexity went far beyond what they had ever intended. But they'd
committed themselves to going down this path, and they missed the simple
solution that would have made it all unnecessary---having two words
instead of one. You either say \forth{GO} or you say \forth{PRETEND}.

In most applications there are remarkably few times when you need to test
the condition. For instance in a video game, you don't really say ``If he
presses Button A, then do this; if he presses Button B, then do something
else.'' You don't go through that kind of logic.

If he presses the button, you do something. What you do is associated with
the button, not with the logic.

Conditionals aren't bad in themselves---they are an essential construct. But
a program with a lot of conditionals is clumsy and unreadable. All you can
do is question each one. Every conditional should cause you to ask, ``What
am I doing wrong?''

What you're trying to do with the conditional can be done in a different
way. The long-term consequences of the different way are preferable to the
long-term consequences of the conditional.
\end{tfquot}\index{M!Moore, Charles|)}
\end{interview}%
\index{C!Control structure minimization!reasons for|)}%
Before we introduce some detailed techniques, let's look at three
approaches to the use of conditionals in a particular example.
\Fig{fig8-1}, \Fig{fig8-2}, and \Fig{fig8-3} illustrate three versions of
a design for an automatic teller machine.\program{teller1}

\begin{figure*}[tttt]
\verbfig{fig8-1}{The structured approach.}
\begin{center}
\small\begin{BVerbatim}[commandchars=\&\{\},baselinestretch=0.85]
&underline{AUTOMATIC-TELLER}

IF card is valid DO
   IF card owner is valid DO
      IF request withdrawal DO
         IF authorization code is valid DO
            query for amount
            IF request &(&le&) current balance DO
               IF withdrawal &(&le&) available cash DO
                  vend currency
                  debit account
               ELSE
                  message
                  terminate session
            ELSE
               message
               terminate session
         ELSE
            message
            terminate session
      ELSE
         IF authorization code is valid DO
            query for amount
            accept envelope through hatch
            credit account
         ELSE
            message
            terminate session
   ELSE
      eat card
ELSE
   message
END
\end{BVerbatim}
\end{center}
\end{figure*}
The first example comes straight out of the School for Structured
Programmers. The logic of the application depends on the correct nesting
of \forth{IF} statements.

Easy to read? Tell me under what condition the user's card gets eaten. To
answer, you have to either count \forth{ELSE}s from the bottom and match
them with the same number of \forth{IF}s from the top, or use a
straightedge.

\begin{figure*}[tttt]
\verbfig{fig8-2}{Nesting conditionals within named procedures.}
\small\begin{center}
\begin{BVerbatim}[commandchars=\&\{\},baselinestretch=0.85]
&underline{AUTOMATIC-TELLER}

PROCEDURE READ-CARD
     IF  card is readable  THEN  CHECK-OWNER
          ELSE  eject card  END

PROCEDURE CHECK-OWNER
     IF  owner is valid  THEN  CHECK-CODE
          ELSE  eat card  END

PROCEDURE CHECK-CODE
     IF  code entered matches owner  THEN  TRANSACT
          ELSE message, terminate session  END

PROCEDURE TRANSACT
     IF requests withdrawal  THEN  WITHDRAW
          ELSE  DEPOSIT END

PROCEDURE WITHDRAW
     Query
     If  request &(&le&) current balance  THEN  DISBURSE  END

PROCEDURE DISBURSE
     IF disbursement &(&le&) available cash  THEN
           vend currency
           debit account
         ELSE  message  END

PROCEDURE DEPOSIT
     accept envelope
     credit account
\end{BVerbatim}
\end{center}
\end{figure*}
The second version, \Fig{fig8-2}, shows the improvement that using many
small, named procedures can have on readability. The user's card is eaten
if the owner is not valid.

But even with this improvement, the design of each word depends completely
on the \emph{sequence} in which the tests must be performed. The
supposedly ``highest'' level procedure is burdened with eliminating the
worst-case, most trivial kind of event. And each test becomes responsible
for invoking the next test.

\begin{figure*}[tttt]
\verbfig{fig8-3}{Refactoring and/or eliminating conditionals.}
\begin{center}
\small\begin{BVerbatim}[commandchars=\&\{\},baselinestretch=0.85]
&underline{AUTOMATIC-TELLER}

: RUN
     READ-CARD  CHECK-OWNER  CHECK-CODE  TRANSACT  ;

: READ-CARD
     valid code sequence NOT readable  IF  eject card  QUIT
        THEN ;

: CHECK-OWNER
     owner is NOT valid  IF  eat card  QUIT  THEN ;

: CHECK-CODE
     code entered MISmatches owner's code  IF  message  QUIT
        THEN ;

: READ-BUTTON ( -- adr-of-button's-function)
     ( device-dependent primitive) ;

: TRANSACT
     READ-BUTTON  EXECUTE ;

1 BUTTON WITHDRAW

2 BUTTON DEPOSIT

: WITHDRAW
     Query
     request &(&le&) current balance  IF  DISBURSE  THEN ;

: DISBURSE
     disbursement &(&le&) available cash  IF
            vend currency
            debit account
          ELSE  message  THEN  ;

: DEPOSIT
     accept envelope
     credit account ;
\end{BVerbatim}
\end{center}
\end{figure*}

The third version comes closest to the promise of \Forth{}. The highest
level word expresses exactly what's happening conceptually, showing only
the main path. Each of the subordinate words has its own error exit, not
cluttering the reading of the main word. One test does not have to invoke
the next test.

Also \code{TRANSACT} is designed around the fact that the user will make
requests by pressing buttons on a keypad. No conditions are necessary. One
button will initiate a withdrawal, another a deposit. This approach
readily accommodates design changes later, such as the addition of a
feature to transfer funds. (And this approach does not thereby become
dependent on hardware. Details of the interface to the keypad may be
hidden within the keypad lexicon, \code{READ-BUTTON} and \code{BUTTON}.)

Of course, \Forth{} will allow you to take any of the three approaches.
Which do you prefer?\program{teller2}

