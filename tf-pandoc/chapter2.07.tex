\section{Defining the Rules}%
\index{A!Analysis!rule definition|(}%
\index{C!Conceptual model!rule definition|(}%
\index{R!Rule definition|(}%
\program{phone1}

Most of your efforts at defining a problem will center on describing
the interface.%
\index{I!Interface definition}
Some applications will also require that you define the set of
application rules.

All programming involves rules. Usually these rules are so simple it
hardly matters how you express them: ``If someone pushes the button,
ring the bell.''

Some applications, however, involve rules so complicated that they
can't be expressed in a few sentences of English. A few formal techniques
can come in handy to help you understand and document these more
complicated rules.

Here's an example. Our requirements call for a system to compute the
charges on long-distance phone calls. Here's the customer's
explanation of its rate structure. (I made this up; I have no idea how
the phone company actually computes their rates except that they
overcharge.)

\begin{tfquot}
All charges are computed by the minute, according to distance in
hundreds of miles, plus a flat charge. The flat charge for direct dial
calls during weekdays between 8 A.M. and 5 P.M. is .30 for the first
minute, and .20 for each additional minute; in addition, each minute
is charged .12 per 100 miles. The flat charge for direct calls during
weekdays between 5 P.M. and 11 P.M. is .22 for the first minute, and
.15 for each additional minute; the distance rate per minute is .10
per 100 miles. The flat charge for direct calls late during weekdays
between 11 P.M. or anytime on Saturday, Sundays, or holidays is .12
for the first minute, and .09 for each additional minute; the distance
rate per minute is .06 per 100 miles. If the call requires assistance
from the operator, the flat charge increases by .90, regardless of the hour.
\end{tfquot}
This description is written in plain old English, and it's quite a
mouthful.  It's hard to follow and, like an attic cluttered with
accumulated belongings, it may even hide a few bugs.

%Page 053 in first edition
In building a conceptual model for this system, we must describe the
rate structure in an unambiguous, useful way. The first step towards
cleaning up the clutter involves factoring out irrelevant pieces of
information---that is, applying the rules of limited redundancy. We
can improve this statement a lot by splitting it into two statements.
First there's the time-of-day rule:%
\index{A!Analysis!conceptual model|)}

\begin{tfquot}
Calls during weekdays between 8 A.M. and 5 P.M. are charged at ``full'' rate.
Calls during weekdays between 5 P.M. and 11 P.M. are charged at ``lower''
rate. Calls placed during weekdays between 11 P.M. or anytime on Saturday,
Sundays, or holidays are charged at the ``lowest'' rate.
\end{tfquot}
Then there's the rate structure itself, which should be described in
terms of ``first-minute rate,'' ``additional minute rate,'' ``distance
rate,'' and ``operator-assistance rate.''

\begin{tip}
Factor the fruit. (Don't confuse apples with oranges.)
\end{tip}
These prose statements are still difficult to read, however. System
analysts use several techniques to simplify these statements:
structured English, decision trees, and decision tables. Let's study
each of these techniques and evaluate their usefulness in the \Forth{}
environment.

\subsection{Structured English}%
\index{A!Analysis!Structured English|(}%
\index{S!Structured English|(}

Structured English is a sort of structured pseudocode in which our rate
statement would read something like this:

%Page 053/054 in first edition
\begin{Code}[baselinestretch=0.95]
IF full rate
   IF direct-dial
      IF first-minute
	 .30 + .12/100miles
      ELSE ( add'l- minute)
	 .20 + .12/100miles
      ENDIF
   ELSE ( operator )
      IF first-minute
	 1.20 + .12/100miles
      ELSE ( add'l- minute)
	 .20 + .12/100miles
      ENDIF
   ENDIF
ELSE  ( not-full-rate)
   IF lower-rate
      IF direct-dial
	 IF first-minute
	    .22 + .10/100miles
	 ELSE ( add'l- minute)
	    .15 + .10/100miles
	 END IF
      ELSE ( operator)
	 IF first-minute
	    1.12 + .10/100miles
	 ELSE ( add'l- minute)
	    .15 + .10/100miles
	 ENDIF
      ENDIF
   ELSE ( lowest-rate)
      IF direct-dial
	 IF first-minute
	    .12 + .06/100miles
	 ELSE ( add'l- minute)
	    .09 + .O6/100miles
	 ENDIF
      ELSE ( operator)
	 IF first-minute
	    1.02 + .O6/100miles
	 ELSE ( add'l- minute)
	    .09 + .06/100miles
	 ENDIF
      ENDIF
   ENDIF
ENDIF
\end{Code}
This is just plain awkward. It's hard to read, harder to maintain, and
hardest to write. And for all that, it's worthless at implementation
time. I don't even want to talk about it anymore.%
\index{A!Analysis!Structured English|)}%
\index{S!Structured English|)}

\subsection{The Decision Tree}%
\index{A!Analysis!decision tree|(}%
\index{D!Decision tree|(}

\wepsfigt{fig2-4}{Example of a decision tree.}

\Fig{fig2-4} illustrates the telephone rate rules by means of a
decision tree.  The decision tree is the easiest method of any to
``follow down'' to determine the result of certain conditions. For
this reason, it may be the best representation to show the customer.

Unfortunately, the decision tree is difficult to ``follow up,'' to
determine which conditions produce certain results. This difficulty
inhibits seeing ways to simplify the problem. The tree obscures the
fact that additional minutes cost the same, whether the operator
assists or not. You can't see the facts for the tree.%
\index{A!Analysis!decision tree|)}%
\index{D!Decision tree|)}

%Page 055 in first edition

% Figure 2-4 was here

\subsection{The Decision Table}%
\index{A!Analysis!decision table|(}%
\index{D!Decision table|(}

The decision table, described next, provides the most usable graphic
representation of compound rules for the programmer, and possibly for
the customer as well. \Fig{fig2-5} shows our rate structure rules in
decision-table form.

\wepsfigt{fig2-5}{The decision table.}

In \Fig{fig2-5} there are three dimensions: the rate discount, whether
an operator intervenes, and initial minute vs. additional minute.

Drawing problems with more than two dimensions gets a little tricky.
As you can see, these additional dimensions can be depicted on
%Page 056 in first edition
paper as subdimensions within an outer dimension. All of the
subdimension's conditions appear within every condition of the outer
dimension.  In software, any number of dimensions can be easily
handled, as we'll see.

All the techniques we've described force you to analyze which
conditions apply to which dimensions. In factoring these dimensions,
two rules apply:

First, all the elements of each dimension must be mutually exclusive.
You don't put ``first minute'' in the same dimension as ``direct
dial,'' because they are not mutually exclusive.

Second, all possibilities must be accounted for within each dimension.
If there were another rate for calls made between 2 A.M. to 2:05 A.M.,
the table would have to be enlarged.

But our decision tables have other advantages all to themselves.  The
decision table not only reads well to the client but actually benefits
the implementor in several ways:

\begin{description}
\item[Transferability to actual code.] This is particularly true in
\Forth{}, where decision tables are easy to implement in a form very
similar to the drawing.

\item[Ability to trace the logic upwards.] Find a condition and see what
factors produced it.

\item[Clearer graphic representation.] Decision tables serve as a better
tool for understanding, both for the implementor and the analyst.
\end{description}

Unlike decision trees, these decision tables group the \emph{results}
together in a graphically meaningful way. Visualization of ideas helps in
understanding problems, particularly those problems that are too
complex to perceive in a linear way.

For instance, \Fig{fig2-5} clearly shows that the charge for
additional minutes does not depend on whether an operator assisted or not.
With this new understanding we can draw a simplified table, as shown
in \Fig{fig2-6}.

\wepsfigt{fig2-6}{A simplified decision table.}

%Page 057 in first edition

It's easy to get so enamored of one's analytic tools that one forgets
about the problem. The analyst must do more than carry out all
possibilities of a problem to the nth degree, as I have seen authors
of books on structured analysis recommend. That approach only
increases the amount of available detail. The problem solver must also
try to simplify the problem.

\begin{tip}
You don't understand a problem until you can simplify it.
\end{tip}

\noindent If the goal of analysis is not only understanding, but
simplification, then perhaps we've got more work to do.

Our revised decision table (\Fig{fig2-6}) shows that the per-mile
charge depends only on whether the rate is full, lower, or lowest. In
other words, it's subject to only one of the three dimensions shown in
the table.  What happens if we split this table into two tables, as in
\Fig{fig2-7}?

\wepsfiga{fig2-7}{The sectional decision table.}

Now we're getting the answer through a combination of table look-up
and calculation. The formula for the per-minute charge can be
expressed as a pseudo\Forth{} definition:

\begin{Code}
: PER-MINUTE-CHARGE ( -- per-minute-charge)
        CONNECT-CHARGE  MILEAGE-CHARGE  + ;
\end{Code}
The ``\forth{+}'' now appears once in the definition,
not nine times in the table.

Taking the principle of calculation one step further, we note (or
remember from the original problem statement) that operator assistance
merely adds a one-time charge of .90 to the total charge. In this
sense, the operator charge is not a function of any of the three
dimensions. It's more
%Page 058 in first edition
appropriately expressed as a ``logical calculation''; that is, a
function that combines logic with arithmetic:

\begin{Code}
: ?ASSISTANCE
   ( direct-dial-charge -- total-charge)
   OPERATOR? IF .90 + THEN ;
\end{Code}
(But remember, this charge applies only to the first minute.)

\wepsfigb{fig2-8}{The decision table without operator involvement depicted.}

This leaves us with the simplified table shown in \Fig{fig2-8}, and an
increased reliance on expressing calculations. Now we're getting
somewhere.

Let's go back to our definition of \forth{PER-MINUTE-CHARGE}:
\begin{Code}
: PER-MINUTE-CHARGE ( -- per-minute-charge)
   CONNECT-CHARGE  MILEAGE-CHARGE  + ;
\end{Code}
Let's get more specific about the rules for the connection charge and for
the mileage charge.

The connection charge depends on whether the minute is the first or
an additional minute. Since there are two kinds of per-minute charges,
perhaps it will be easiest to rewrite \forth{PER-MINUTE-CHARGE} as two
different words.

Let's assume we will build a component that will fetch the appropriate
rates from the table. The word \forth{1MINUTE} will get the rate for
the first minute; \forth{+MINUTES} will get the rate for each
additional minute.  Both of these words will depend on the time of day
to determine whether to use the full, lower, or lowest rates.

Now we can define the pair of words to replace \forth{PER-MINUTE-CHARGE}:

%Page 059 in first edition

\begin{Code}
: FIRST  ( -- charge)
  1MINUTE  ?ASSISTANCE   MILEAGE-CHARGE + ;
: PER-ADDITIONAL  ( -- charge)
   +MINUTES  MILEAGE-CHARGE + ;
\end{Code}
What is the rule for the mileage charge? Very simple. It is the rate
(per hundred miles) times the number of miles (in hundreds). Let's
assume we can define the word \forth{MILEAGE-RATE}, which will fetch
the mileage rate from the table:

\begin{Code}
: MILEAGE-CHARGE  ( -- charge)
   #MILES @  MILEAGE-RATE * ;
\end{Code}
Finally, if we know the total number of minutes for a call, we can now
calculate the total direct-dial charge:

\begin{Code}
: TOTAL   ( -- total-charge)
   FIRST                        ( first minute rate)
   ( #minutes) 1-               ( additional minutes)
      PER-ADDITIONAL *          ( times the rate)
   +  ;                         ( added together)
\end{Code}
We've expressed the rules to this particular problem through a
combination of simple tables and logical calculations.

(Some final notes on this example: We've written something very close
to a running \Forth{} application. But it is only pseudocode. We've
avoided stack manipulations by assuming that values will somehow be on
the stack where the comments indicate. Also, we've used hyphenated
names because they might be more readable for the customer. Short
names are preferred in real code---see \Chap{5}.)\program{phone2}

We'll unveil the finished code for this example in \Chap{8}.%
\index{A!Analysis!decision table|)}%
\index{A!Analysis!rule definition|)}%
\index{C!Conceptual model!rule definition|)}%
\index{D!Decision table|)}%
\index{R!Rule definition|)}

