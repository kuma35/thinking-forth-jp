\section{Calculations vs. Data Structures vs. Logic}%
\index{C!Calculations|(}%
\index{D!Data structures:!calculation@vs. calculation vs. logic|(}%
\index{D!Detailed design!calculations|(}%
\index{D!Detailed design!data structures|(}%
\index{D!Detailed design!logic|(}%
\index{L!Logic|(}

We've stated before that the best solution to a problem is the simplest
adequate one; for any problem we should strive for the simplest
approach.

Suppose we must write code to fulfill this specification:

%!! sans-serif
\begin{Code}[fontfamily=cmss]
if the input argument is 1, the output is 10
if the input argument is 2, the output is 12
if the input argument is 3, the output is 14
\end{Code}
There are three approaches we could take:

\begin{description}
\item[Calculation]~
\begin{Code}
( n)  1-  2*  10 +
\end{Code}
\medbreak
\item[Data Structure]~
\begin{Code}
CREATE TABLE  10 C,  12 C,  14 C,
( n)  1- TABLE + C@
\end{Code}
\medbreak
\item[Logic]~
\begin{Code}
( n)  CASE
         1 OF 10 ENDOF
         2 OF 12 ENDOF
         3 OF 14 ENDOF  ENDCASE
\end{Code}
\end{description}
In this problem, calculation is simplest. Assuming it is also adequate
(speed is not critical), calculation is best.

The problem of converting angles to sines and cosines can be implemented
more simply (at least in terms of lines of code and object size)
by calculating the answers than by using a data structure. But for many
applications requiring trig, it's faster to look up the answer in a table
stored in memory. In this case, the simplest \emph{adequate} solution is
using the data structure.

In \Chap{2} we introduced the telephone rate problem. In that
problem the rates appeared to be arbitrary, so we designed a data
structure:

\bigskip
%!! this should be a table; see pg 119
\begin{tabular}{lccc}
          & \emph{Full Rate} & \emph{Lower Rate} & \emph{Lowest Rate} \\ \hline
First Min.  &       .30        &        .22        &          .12 \\ \hline
Add'1 Mins. &       .12        &        .10        &          .06 \\ \hline
\end{tabular}\program{phone3}
\bigskip

\noindent Using a data structure was simpler than trying to invent a
formula by which these values could be calculated. And the formula might
prove wrong later. In this case, table-driven code is easier to maintain.

In \Chap{3} we designed a keystroke interpreter for our Tiny
Editor using a decision table:\program{editor3}

%!! this should be a table; see pg 119
%%typo: The original has here ``Cntrl''. Usually, it's Ctrl, and other chapters
%%use that name, too.
\medskip
\begin{tabular}{lll}
\emph{Key}         & \emph{Not-Inserting} & \emph{Inserting} \\
\texttt{Ctrl-D}    & \texttt{DELETE}      & \texttt{INSERT-OFF} \\
\texttt{Ctrl-I}    & \texttt{INSERT-ON}   & \texttt{INSERT-OFF} \\
\texttt{backspace} & \texttt{BACKWARD}    & \texttt{INSERT<} \\
etc. & &
\end{tabular}
\medskip

\medbreak\noindent
We could have achieved this same result with logic:
\begin{Code}
CASE
   CTRL-D     OF  'INSERTING @  IF
      INSERT-OFF   ELSE DELETE     THEN   ENDOF
   CTRL-I     OF  'INSERTING @  IF
      INSERT-OFF   ELSE INSERT-ON  THEN   ENDOF
   BACKSPACE  OF  'INSERTING @  IF
      INSERT<      ELSE BACKWARD   THEN   ENDOF
ENDCASE
\end{Code}
but the logic is more confusing. And the use of logic to express such a
multi-condition algorithm gets even more convoluted when a table was
not used in the original design.

The use of logic becomes advisable when the result is not calculable,
or when the decision is not complicated enough to warrant a decision
table. \Chap{8} is devoted to the issue of minimizing the use of logic
in your programs.

\begin{tip}
In choosing which approach to apply towards solving a problem, give
preference in the following order:
\medskip
%!! there's certainly some better way to do ordered lists in TeX
\begin{enumerate}
\item calculation (except when speed counts)
\item data structures
\item logic
\end{enumerate}
\end{tip}
Of course, one nice feature of modular languages such as \Forth{} is that
the actual implementation of a component---whether it uses calculation,
data structures, or logic---doesn't have to be visible to the rest of the
application.
\index{C!Calculations|)}%
\index{D!Data structures:!calculation@vs. calculation vs. logic|)}%
\index{D!Detailed design!calculations|)}%
\index{D!Detailed design!data structures|)}%
\index{D!Detailed design!logic|)}%
\index{L!Logic|)}

