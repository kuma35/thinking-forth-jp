\section{Defining the Interfaces}%
\index{A!Analysis!interface definition|(}%
\index{C!Conceptual model!interface definition|(}%
\index{I!Interface definition|(}

\begin{tip}
First, and most importantly, the conceptual model should describe the
system's interfaces.\index{I!Interface definition}
\end{tip}%
\begin{interview}
\index{T!Teleska, John|(}
\person{Teleska}:

\begin{tfquot}
The ``spec'' basically deals with WHAT. In its most glorious form, it
describes what the system would look like to the user---you might call it
the user's manual. I find I write more notes on the human
interaction---what it
%Page 049 in first edition
will look like on the outside---than on the part that gets the job
done. For instance, I'll include a whole error-action listing to show
what happens when a particular error occurs. Oddly, this is the part
that takes the most time to implement anyway.

I'm currently working on a solid-state industrial washing-machine
timer. In this case, the user interface is not that complex. What is
complex is the interface to the washing machine, for which I must
depend on the customer and the documentation they can provide.

The significant interface is whatever is the arms and legs of the
product. I don't make the distinction between hardware and software at
this early stage. They can be interchanged in the implementation.

The process of designing hardware and the process of designing
software are analogous. The way I design hardware is to treat it as a
black box. The front panel is input and output. You can do the same
with software.

I use any techniques, diagrams, etc., to show the customer what the
inputs and outputs look like, using his description of what the
product has to do.  But in parallel, in my own mind, I'm imagining how
it will be implemented.  I'm evaluating whether I can do this
efficiently. So to me it's not a black box, it's a gray box. The
designer must be able to see inside the black boxes.

When I design a system that's got different modules, I try to make the
coupling as rational and as little as possible. But there's always
give and take, since you're compromising the ideal.

\index{D!Data-flow diagrams|(}
For the document itself, I use DFDs {[}data-flow diagrams, which we'll
discuss later{]}, and any other kind of representation that I can show
to my client. I show them as many diagrams as I can to clarify my
understanding.  I don't generally use these once it comes to
implementation. The prose must be complete, even without reference to
the diagrams.
\end{tfquot}%
\index{T!Teleska, John|)}
\end{interview}%
\index{E!Error handling|(}
\begin{tip}
Decide on error- and exception-handling early as part of defining the
interface.
\end{tip}

\noindent It's true that when coding for oneself, a programmer can
often concentrate first on making the code run correctly under
\emph{normal} conditions, then worry about error-handling later. When
working for someone else, however, error-handling should be worked out
ahead of time. This is an area often overlooked by the beginning
programmer.

The reason it's so important to decide on error-handling at this stage
is the wide divergence in how errors can be treated. An error might be:

\begin{itemize}
\item ignored
\item made to set a flag indicating that an error occurred, while
processing continues
\item made to halt the application immediately
\item designed to initiate procedures to correct the problem and keep
the program running.
\end{itemize}
%
%Page 050 in first edition
%
There's room for a serious communications gap if the degree of
complexity required in the error-handling is not nailed down early.
Obviously, the choice bears tremendous impact on the design and
implementation of the application.%
\index{E!Error handling|)}

\begin{tip}
Develop the conceptual model by imagining the data traveling through and
being acted upon by the parts of the model.
\end{tip}%
\index{S!Structured analysis|(}
A discipline called \emph{structured analysis} \cite{weinberg80}
offers some techniques for describing interfaces in ways that your
clients will easily understand.  One of these techniques is called the
``data-flow diagram'' (DFD), which \person{Teleska} mentioned.

\wepsfiga{fig2-3}{A data-flow diagram.}

A data-flow diagram, such as the one depicted in \Fig{fig2-3},
emphasizes what happens to items of data as they travel through the
system.  The circles represent ``transforms,'' functions that act upon
information.  The arrows represent the inputs and outputs of the
transforms.

The diagram depicts a frozen moment of the system in action. It
ignores initialization, looping structures, and other details of
programming that relate to time.

Three benefits are claimed for using DFDs:

First, they speak in simple, direct terms to the customer. If your
%Page 051 in first edition
customer agrees with the contents of your data-flow diagram, you know
you understand the problem.

Second, they let you think in terms of the logical ``whats,'' without
getting caught up in the procedural ``hows,'' which is consistent with
the philosophy of hiding information as we discussed in the last chapter.

Third, they focus your attention on the interfaces to the system and
between modules.

\Forth{} programmers, however, rarely use DFDs except for the customer's
benefit. \Forth{} encourages you to think in terms of the conceptual
model, and \Forth{}'s implicit use of a data stack makes the passing of
data among modules so simple it can usually be taken for granted.
This is because \Forth{}, used properly, approaches a functional language.%
\index{D!Data-flow diagrams|)}%
\index{S!Structured analysis|)}%
\index{P!Pseudocode|(}

For anyone with a few days' familiarity with \Forth{}, simple definitions
convey at least as much meaning as the diagrams:
\begin{Code}
: REQUEST  ( quantity part# -- )
   ON-HAND?  IF  TRANSFER  ELSE  REORDER  THEN ;
: REORDER   AUTHORIZATION?  IF  P.O.  THEN ;
: P.O.   BOOKKEEPING COPY   RECEIVING COPY
   VENDOR MAIL-COPY ;
\end{Code}
This is \Forth{} pseudocode. No effort has been made to determine what
values are actually passed on the stack, because that is an
implementation detail. The stack comment for \forth{REQUEST} is used
only to indicate the two items of data needed to initiate the process.

(If I were designing this application, I'd suggest that the user
interface be a word called \forth{NEED}, which has this syntax:

\begin{Code}
NEED 50 AXLES
\end{Code}

\noindent \forth{NEED} converts the quantity into a numeric value on
the stack, translates the string \forth{AXLES} into a part number,
also on the stack, then calls \forth{REQUEST}. Such a command should
be defined only at the outer-most level.)%
\begin{interview}
\index{J!Johnson, Dave|(}
\person{Johnson} of Moore Products Co. has a few words on \Forth{}
pseudocode:
\begin{tfquot}
IBM uses a rigorously documented PDL (program design language). We use
a PDL here as well, although we call it FDL, for \Forth{} design
language. It's probably worthwhile having all those standards, but
once you're familiar with \Forth{}, \Forth{} itself can be a design
language. You just have to leave out the so-called ``noise'' words:
\forth{C@}, \forth{DUP}, \forth{OVER}, etc., and show only the basic
flow. Most \Forth{} people probably do that informally. We do it
purposefully.
\end{tfquot}%
\index{J!Johnson, Dave|)}%
\end{interview}%
\index{P!Pseudocode|)}
%Page 052 in first edition
\begin{interview*}
During one of our interviews I asked \person{Moore}\index{M!Moore,
Charles|(} if he used diagrams of any sort to plan out the conceptual
model, or did he code straight into \Forth{}? His reply:

\begin{tfquot}
The conceptual model \emph{is} \Forth{}. Over the years I've learned
to think that way.
\end{tfquot}
Can everyone learn to think that way?

\begin{tfquot}
I've got an unfair advantage. I codified my programming style and other
people have adopted it. I was surprised that this happened. And I feel at a
lovely advantage because it is my style that others are learning to emulate.
Can they learn to think like I think? I imagine so. It's just a matter of
practice, and I've had more practice.
\end{tfquot}\index{M!Moore, Charles|)}
\end{interview*}%
\index{A!Analysis!interface definition|)}%
\index{C!Conceptual model!interface definition|)}%
\index{I!Interface definition|)}

