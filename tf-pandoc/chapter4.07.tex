\section{Solving a Problem: Computing Roman Numerals}%
\index{D!Detailed design!demonstration of|(}%
\index{R!Roman numeral computation problem|(}%
\program{roman1}

In this section we'll attempt to demonstrate the process of designing a
lexicon. Rather than merely present the problem and its solution, I'm
hoping we can crack this problem together. (I kept a record of my
thought processes as I solved this problem originally.) You'll see
elements of the problem-solving guidelines previously given, but you'll
also see them being applied in a seemingly haphazard order---just as they
would be in reality.

Here goes: The problem is to write a definition that consumes a
number on the stack and displays it as a Roman numeral.

This problem most likely represents a component of a larger system. We'll
probably end up defining several words in the course of solving this
problem, including data structures. But this particular lexicon will
include only one name, \forth{ROMAN}, and it will take its argument from
the stack. (Other words will be internal to the component.)

Having thus decided on the external syntax, we can now proceed to
devise the algorithms\index{A!Algorithms} and data structures.

We'll follow the scientific method---we'll observe reality, model a
solution, test it against reality, modify the solution, and so on. We'll
begin by recalling what we know about Roman numerals.

Actually, we don't remember any formal rules about Roman
numerals. But if you give us a number, we can make a Roman numeral
out of it. We know how to do it---but we can't yet state the procedure as
an algorithm.

So, let's look at the first ten Roman numerals:

%!! paragraph should be sans-serif + indented
\medskip
{\sf\begin{tabular}{r}
I \\
II \\
III \\
IV \\
V \\
VI \\
VII \\
VIII \\
IX \\
X \\
\end{tabular}}
\medskip

\noindent We make a few observations. First, there's the idea of a
tally, where we represent a number by making that many marks (3 =
III). On the other hand, special symbols are used to represent groups
(5 = V). In fact, it seems we can't have more than three I's in a row
before we use a larger symbol.

Second, there's a symmetry around five. There's a symbol for five (V), and
a symbol for ten (X). The pattern I, II, III repeats in the second half,
but with a preceding V.

One-less-than-five is written IV, and one-less-than-ten is written IX.
It seems that putting an ``I'' in front of a larger-value symbol is like
saying ``one-less-than\dots{}''

These are vague, hazy observations. But that's alright. We don't
have the whole picture yet.

Let's study what happens above ten:

%!! paragraph should be sans-serif + indented
\medskip
{\sf\begin{tabular}{r}
XI \\
XII \\
XIII \\
XIV \\
XV \\
XVI \\
XVII \\
XVIII \\
XIX \\
XX \\
\end{tabular}}

\medskip
\noindent This is exactly the pattern as before, with an extra ``X''
in front. So there's a repeating cycle of ten, as well.

If we look at the twenties, they're the same, with two ``X''s; the
thirties with three ``X''s. In fact, the number of ``X'' is the same as the
number in the tens column of the original decimal number.

This seems like an important observation: we can decompose our decimal
number into decimal digits, and treat each digit separately. For instance,
37 can be written as

%!! sans-serif + indented
\begin{quote}
{\sf XXX (thirty)}
\end{quote}

\noindent followed by

%!! sans-serif + indented
\begin{quote}
{\sf VII (seven)}
\end{quote}

\noindent It may be premature, but we can already see a method by
which \Forth{} will let us decompose a number into decimal digits---with
modulo division by ten. For instance, if we say

\begin{Code}
37 10 /MOD
\end{Code}
we'll get a 7 and a 3 on the stack (the three---being the quotient---is on
top.)

But these observations raise a question: What about below ten, where there
is no ten's place? Is this a special case? Well, if we consider that
each ``X'' represents ten, then the absence of ``X'' represents zero.  So
it's \emph{not} a special case. Our algorithm works, even for numbers less
than ten.

Let's continue our observations, paying special attention to the
cycles of ten. We notice that forty is ``XL.'' This is analogous to 4 being
``IV,'' only shifted by the value of ten. The ``X'' before the ``L'' says
``ten-less-than-fifty.'' Similarly,

%!! paragraph should be sans-serif + indented
\bigskip
{\sf\begin{tabular}{rrcrr}
   L &  (50) &   is analogous to &     V & (5) \\
  LX &  (60) &          "        &    VI & (6) \\
 LXX &  (70) &          "        &   VII & (7) \\
LXXX &  (80) &          "        &  VIII & (8) \\
  XC &  (90) &          "        &    IX & (9) \\
   C & (100) &          "        &     X & (10) \\
\end{tabular}}
\bigskip

\noindent Apparently the same patterns apply for any decimal digit---only
the symbols themselves change. Anyway, it's clear now that we're dealing
with an essentially decimal system.

If pressed to do so, we could even build a model for a system to display
Roman numerals from 1 to 99, using a combination of algorithm and
data structure.

\subsection{Data Structure}

%!! see table on pg 123
\bigskip
\begin{tabular}{lcrclcr}
\multicolumn{3}{c}{\underline{\emph{Ones' Table}}} &
\hbox to 3em{\hss} & \multicolumn{3}{c}{\underline{\emph{Tens' Table}}} \\
& 0 &          & &    & 0 & \\
& 1 &       I  & &    & 1 &      X \\
& 2 &      II  & &    & 2 &     XX \\
& 3 &     III  & &    & 3 &    XXX \\
& 4 &      IV  & &    & 4 &     XL \\
& 5 &       V  & &    & 5 &      L \\
& 6 &      VI  & &    & 6 &     LX \\
& 7 &     VII  & &    & 7 &    LXX \\
& 8 &    VIII  & &    & 8 &   LXXX \\
& 9 &      IX  & &    & 9 &     XC \\
\end{tabular}
\bigskip

\subsection{Algorithm}

Divide $n$ by 10. The quotient is the tens' column digit; the remainder is
the ones' column digit. Look up the ten's digit in the tens' table and
print the corresponding symbol pattern. Look up the ones' digit in the
one's table and print that corresponding symbol pattern.

For example, if the number is 72, the quotient is 7, the remainder is 2. 7
in the tens' table corresponds to ``LXX,'' so print that. 2 in the ones'
column corresponds to ``II,'' so print that. The result:

%!! sans-serif + indented
\begin{quote}
{\sf LXXII}
\end{quote}
We've just constructed a model that works for numbers from one to 99.
Any higher number would require a hundreds' table as well, along with
an initial division by 100.

The logical model just described might be satisfactory, as long as it
does the job. But somehow it doesn't seem we've fully solved the problem.
We avoided figuring out how to produce the basic pattern by storing
all possible combinations in a series of tables. Earlier in this chapter we
observed that calculating an answer, if it's possible, can be easier than
using a data structure.

Since this section deals with devising algorithms\index{A!Algorithms},
let's go all the way.
Let's look for a general algorithm for producing any digit, using
only the elementary set of symbols. Our data structure should contain
only this much information:

%!! sans-serif + indented
\bigskip
{\sf\begin{tabular}{c@{\hspace{1em}}c}
I & V \\
X & L \\
C & D \\
M & \\
\end{tabular}}
\bigskip

\noindent In listing the symbols, we've also \emph{organized} them in a
way that seems right. The symbols in the left column are all multiples of
ten; the symbols in the right column are multiples of five. Furthermore,
the symbols in each row have ten times the value of the symbols directly
above them.

Another difference, the symbols in the first column can all be combined in
multiples, as ``XXXIII.'' But you can't have multiples of any of the
right-column symbols, such as VVV. Is this observation useful? Who knows?

Let's call the symbols in the left column \forth{ONERS} and in the right
column \forth{FIVERS}. The \forth{ONERS} represent the values 1, 10, 100,
and 1,000; that is, the value of one in every possible decimal place. The
\forth{FIVERS} represent 5, 50, and 500; that is, the value of five in
every possible decimal place.

Using these terms, instead of the symbols themselves, we should be able to
express the algorithm\index{A!Algorithms} for producing any digit. (We've
factored out the actual symbols from the \emph{kind} of symbols.) For
instance, we can state the following preliminary algorithm:

%!! indent paragraph
\begin{tfquot}
For any digit, print as many \forth{ONERS} as necessary to add up to the
value.
\end{tfquot}
Thus, for 300 we get ``CCC,'' for 20 we get ``XX'' for one we get ``I.''
And for 321 we get ``CCCXXI.''

This algorithm works until the digit is 4. Now we'll have to expand
our algorithm to cover this exception:

\begin{tfquot}
Print as many \forth{ONERS} as necessary to add up to the value, but if
the digit is 4, print a \forth{ONER} then a \forth{FIVER}.
%!! vertical half space
Hence, 40 is ``XL''; 4 is ``IV.''
\end{tfquot}
This new rule works until the digit is 5. As we noticed before, digits of
five and above begin with a \forth{FIVER} symbol. So we expand our rule
again:

\begin{tfquot}
If the digit is 5 or more, begin with a \forth{FIVER} and subtract five
from the value; otherwise do nothing. Then print as many \forth{ONERS} as
necessary to add up to the value. But if the digit is 4, print only a
\forth{ONER} and a \forth{FIVER}.
\end{tfquot}
This rule works until the digit is 9. In this case, we must print a
\forth{ONER} preceding a---what? A \forth{ONER} from the next higher
decimal place (the next row below). Let's call this a \forth{TENER}. Our
complete model, then is:

\begin{tfquot}
If the digit is 5 or more, begin with a \forth{FIVER} and subtract five
from the value; otherwise do nothing. Then, print as many \forth{ONERS} as
necessary to add up to the value. But if the digit is 4, print only a
\forth{ONER} and a \forth{FIVER}, or if it's 9, print only a \forth{ONER}
and a \forth{TENER}.
\end{tfquot}
\index{A!Algorithms|(}%
We now have an English-language version of our algorithm.  But we still
have some steps to go before we can run it on our computer.

In particular, we have to be more specific about the exceptions. We
can't just say,

\begin{tfquot}
Do a, b, and c. \emph{But} in such and such a case, do something different.
\end{tfquot}
because the computer will do a, b, and c before it knows any better.
\medbreak

Instead, we have to check whether the exceptions apply \emph{before} we
do anything else.

\begin{tip}
In devising an algorithm, consider exceptions last. In writing code, handle
exceptions first.
\end{tip}
This tells us something about the general structure of our digit-producing
word. It will have to begin with a test for the 4/9 exceptions. In
either of those cases, it will respond accordingly. If neither exception
applies, it will follow the ``normal'' algorithm. Using pseudocode, then:

\begin{Code}
: DIGIT  ( n )  4-OR-9? IF  special cases
   ELSE  normal case  THEN ;
\end{Code}
An experienced \Forth{} programmer would not actually write out this
pseudocode, but would more likely form a mental image of the structure
for eliminating the special cases. A less experienced programmer might
find it helpful to capture the structure in a diagram, or in code as we've
done here.
\index{A!Algorithms|)}

In \Forth{} we try to minimize our dependence on logic. But in this case
we need the conditional \forthb{IF} because we have an exception we need
to eliminate. Still, we've minimized the complexity of the control
structure by limiting the number of \forthb{IF }\forthb{THEN}s in this
definition to one.

Yes, we still have to distinguish between the 4-case and the 9-case,
but we've deferred that structural dimension to lower-level
definitions---the test for 4-or-9 and the ``special case'' code.

What our structure really says is that either the 4-exception or the
9-exception must prohibit execution of the normal case. It's not enough
merely to test for each exception, as in this version:

\begin{Code}
: DIGIT  ( n )  4-CASE? IF  ONER FIVER  THEN
                9-CASE? IF  ONER TENER  THEN
                normal case... ;
\end{Code}
because the normal case is never excluded. (There's no way to put an
\forthb{ELSE} just before the normal case, because \forthb{ELSE} must
appear between \forthb{IF} and \forthb{THEN}.)

If we insist on handling the 4-exception and the 9-exception separately,
we could arrange for each exception to pass an additional flag, indicating
that the exception occurred. If either of these flags is true, then we can
exclude the normal case:

%!! warning: \textbf{} in verbatim block
\begin{Code}[commandchars=\&\{\}]
: DIGIT  ( n )  4-CASE? &poorbf{DUP} IF  ONER FIVER  THEN
                9-CASE? &poorbf{DUP} IF  ONER TENER  THEN
                OR  NOT IF normal case THEN ;
\end{Code}
But this approach needlessly complicates the definition by adding new
control structures. We'll leave it like it was.

Now we have a general idea of the structure of our main definition.

We stated, ``If the digit is 5 or more, begin with a \forth{FIVER} and
subtract five from the value; otherwise do nothing. Then, print as many
\forth{ONERS} as necessary to add up to the value.''

A direct translation of these rules into \Forth{} would look like this:

\begin{Code}
( n)  DUP  4 > IF  FIVER 5 -  THEN  ONERS
\end{Code}
This is technically correct, but if we're familiar with the technique of
modulo division, we'll see this as a natural situation for modulo division
by 5. If we divide the number by five, the quotient will be zero (false)
when the number is less than five, and one (true) when it's between 5 and
9. We can use it as the boolean flag to tell whether we want the leading
\forth{FIVER}:

\begin{Code}
( n )  5 / IF FIVER THEN ...
\end{Code}
The quotient/flag becomes the argument to \forth{IF}.

Furthermore, the remainder of modulo 5 division is always a number between
0 and 4, which means that (except for our exception) we can use the
remainder directly as the argument to \forth{ONERS}. We revise our phrase
to

\begin{Code}[commandchars=\&\{\}]
( n )  5 &poorbf{/MOD} IF FIVER THEN  &poorbf{ONERS}
\end{Code}
Getting back to that exception, we now see that we can test for both 4 and
9 with a single test---namely, if the remainder is 4. This suggests that
we can do our 5 \forthb{/MOD} first, then test for the exception.
Something like this:

%!! warning: \textbf{} in verbatim block
\begin{Code}[commandchars=\&\{\}]
: DIGIT  ( n )
     5 /MOD  &poorbf{OVER 4 =  IF  special case  ELSE}
     IF FIVER THEN  ONERS  THEN ;
\end{Code}
(Notice that we \forthb{OVER}ed the remainder so that we could compare it with
4 without consuming it.)

So it turns out we \emph{do} have a doubly-nested \forthb{IF THEN}
construct after all. But it seems justified because the \forthb{IF THEN}
is handling the special case. The other is such a short phrase,
``\forth{IF FIVER THEN},'' it's hardly worth making into a separate
definition. You could though. (But we won't.)

Let's focus on the code for the special case. To state its algorithm: ``If
the digit is four, print a \forth{ONER} and a \forth{FIVER}. If the digit
is nine, print a \forth{ONER} and a \forth{TENER}.''

We can assume that the digit will be one or the other, or else we'd never
be executing this definition. The question is, how do we tell which one?

Again, we can use the quotient of division by five. If the quotient is
zero, the digit must have been four; otherwise it was nine. So we'll play
the same trick and use the quotient as a boolean flag. We'll write:

\begin{Code}
: ALMOST  ( quotient )
     IF  ONER TENER  ELSE  ONER FIVER  THEN ;
\end{Code}
In retrospect, we notice that we're printing a \forth{ONER} either way. We
can simplify the definition to:

\begin{Code}
: ALMOST  ( quotient )
     ONER  IF TENER ELSE FIVER THEN ;
\end{Code}
We've assumed that we have a quotient on the stack to use. Let's go back
to our definition of \forth{DIGIT} and make sure that we do, in fact:

%!! warning: \textbf{} in verbatim block
\begin{Code}[commandchars=\&\{\}]
: DIGIT  ( n )
     5 /MOD  OVER 4 =  IF  &poorbf{ALMOST}  ELSE
     IF FIVER THEN  ONERS  THEN ;
\end{Code}
It turns out that we have not only a quotient, but a remainder underneath
as well. We're keeping both on the stack in the event we execute the
\forthb{ELSE} clause. The word \forthb{ALMOST}, however, only needs the
quotient. So, for symmetry, we must \forthb{DROP} the remainder like this:

%!! warning: \textbf{} in verbatim block
\begin{Code}[commandchars=\&\{\}]
: DIGIT  ( n )
     5 /MOD  OVER 4 =  IF  ALMOST  &poorbf{DROP}  ELSE
     IF FIVER THEN  ONERS  THEN ;
\end{Code}
There we have the complete, coded definition for producing a single digit
of a Roman numeral. If we were desperate to try it out before writing the
needed auxiliary definitions, we could very quickly define a lexicon of
words to print one group of symbols, say the \forth{ONES} row:

%!! there is a small mistake in the original book:
%!! : TENER   ." X"; <-- missing space
%!! Fixed that.
\begin{Code}
: ONER    ." I" ;
: FIVER   ." V" ;
: TENER   ." X" ;
: ONERS  ( # of oners -- )
     ?DUP IF 0 DO  ONER  LOOP  THEN ;
\end{Code}
before loading our definitions of \forth{ALMOST} and \forth{DIGIT}.

But we're not that desperate. No, we're anxious to move on to the
problem of defining the words \forth{ONER}, \forth{FIVER}, and \forth{TENER} so that their
symbols depend on which decimal digit we're formatting.

Let's go back to the symbol table we drew earlier:

%!! sans-serif + indented, heading in italics
\bigskip{\sf
\begin{tabular}{rcc}
          & \it \forth{ONER}s & \it \forth{FIVER}s \\
     ones & I  &  V \\
     tens & X  &  L \\
 hundreds & C  &  D \\
thousands & M  & \\
\end{tabular}}
\bigskip

\noindent
We've observed that we also need a ``\forth{TENER}''---which is the
\forth{ONER} in the next row below. It's as if the table should really
be written:

%!! Here comes above table again, but this time with
%!! some circles and lines in it. I have no clue how
%!! to render this. See pg 128.
\bigskip
{\sf\begin{tabular}{rccc}
          & \it \forth{ONER}s & \it \forth{FIVER}s & \it \forth{TENER}s \\
     ones & \circlenode[linecolor=white]{I2}{\boxto{}I}  &  V & \circlenode{X1}{\boxto{}X} \\
     tens & \circlenode{X2}{\boxto{}X}  &  L & \circlenode{C1}{\boxto{}C} \\
 hundreds & \circlenode{C2}{\boxto{}C}  &  D & \circlenode{M1}{\boxto{}M} \\
thousands & \circlenode{M2}{\boxto{}M}  & \\
\end{tabular}
\ncline{X1}{X2}
\ncline{C1}{C2}
\ncline{M1}{M2}}
\bigskip

\noindent But that seems redundant. Can we avoid it? Perhaps if we try
a different model, perhaps a linear table, like this:

\bigskip
{\sf\begin{tabular}{rc}
ones      & I \\
          & V \\
tens      & X \\
          & L \\
hundreds  & C \\
          & D \\
thousands & M \\
\end{tabular}}
\bigskip

\noindent Now we can imagine that each column name (``ones,'' ``tens,''
etc.) points to the \forth{ONER} of that column. From there we can also
get each column's \forth{FIVER} by reaching down one slot below the
current \forth{ONER}, and the \forth{TENER} by reaching down two slots.

It's like building an arm with three hands. We can attach it to the
\forth{ONES} column, as in \Fig{fig4-8}a, or we can attach it to the tens'
column, as in \Fig{fig4-8}b, or to any power of ten.

\wepsfigt{fig4-8}{A mechanical representation: accessing the data
structure.}

%!! include Figure 4-8 here

An experienced \Forth{} programmer is not likely to imagine arms,
hands, or things like that. But there must be a strong mental image---the
stuff of right-brain thinking---before there's any attempt to construct the
model with code.

Beginners who are learning to think in this right-brain way might
find the following tip helpful:

\begin{tip}
If you have trouble thinking about a conceptual model,
\index{C!Conceptual model}
visualize it---or draw it---as a mechanical device.
\end{tip}
Our table is simply an array of characters. Since a character requires
only a byte, let's make each ``slot'' one byte. We'll call the table
\forth{ROMANS}:

\begin{Code}
CREATE ROMANS    ( ones)  ASCII I  C,   ASCII V  C,
                 ( tens)  ASCII X  C,   ASCII L  C,
             ( hundreds)  ASCII C  C,   ASCII D  C,
            ( thousands)  ASCII M  C,
\end{Code}
Note: This use of \forthb{ASCII} requires that \forthb{ASCII} be
``\forthb{STATE}-dependent'' (see \App{C}). If the word \forthb{ASCII}
is not defined in your system, or if it is not state-dependent, use:

\begin{Code}
CREATE ROMANS  73 C,  86 C,  88 C,  76 C,
   67 C,  68 C,  77 C,
\end{Code}
We can select a particular symbol from the table by applying two
different offsets at the same time. One dimension represents the decimal
place: ones, tens, hundreds, etc. This dimension is made ``current,'' that
is, its state stays the same until we change it.

The other dimension represents the kind of symbol we want---\forth{ONER},
\forth{FIVER}, \forth{TENER}---within the current decimal column. This
dimension is incidental, that is, we'll specify which symbol we want each
time.

Let's start by implementing the ``current'' dimension. We need
some way to point to the current decimal column. Let's create a variable
called \forth{COLUMN\#} (pronounced ``column-number'') and have it contain an
offset into the table:

\begin{Code}
VARIABLE COLUMN#  ( current offset)
: ONES        O COLUMN# ! ;
: TENS        2 COLUMN# ! ;
: HUNDREDS    4 COLUMN# ! ;
: THOUSANDS   6 COLUMN# ! ;
\end{Code}
Now we can find our way to any ``arm position'' by adding the contents of
\forth{COLUMN\#} to the beginning address of the table, given by
\forth{ROMANS}:

\begin{Code}
: COLUMN  ( -- adr-of-column)  ROMANS  COLUMN# @  + ;
\end{Code}
Let's see if we can implement one of the words to display a symbol. We'll
start with \forth{ONER}.

The thing we want to do in \forth{ONER} is \forthb{EMIT} a character.

\begin{Code}
: ONER                   EMIT ;
\end{Code}
Working backward, \forthb{EMIT} requires the ASCII character on the stack.
How do we get it there? With \forthb{C@}.

\begin{Code}
: ONER                C@ EMIT ;
\end{Code}
\forthb{C@} requires the \emph{address} of the slot that contains the
symbol we want. How do we get that address?

The \forth{ONER} is the first ``hand'' on the movable arm---the position
that \forth{COLUMN} is already pointing to. So, the address we want is
simply the address returned by \forth{COLUMN}:

\begin{Code}
: ONER   COLUMN       C@ EMIT ;
\end{Code}
Now let's write \forth{FIVER}. It computes the same slot address, then
adds one to get the next slot, before fetching the symbol and emitting it:

\begin{Code}
: FIVER  COLUMN 1+    C@ EMIT ;
\end{Code}
And \forth{TENER} is:

\begin{Code}
: TENER  COLUMN 2+    C@ EMIT ;
\end{Code}
These three definitions are redundant. Since the only difference between
them is the incidental offset, we can factor the incidental offset out
from the rest of the definitions:

\begin{Code}
: .SYMBOL  ( offset)  COLUMN +  C@ EMIT ;
\end{Code}
Now we can define:

\begin{Code}
: ONER    O .SYMBOL ;
: FIVER   1 .SYMBOL ;
: TENER   2 .SYMBOL ;
\end{Code}
All that remains for us to do now is to decompose our complete decimal
number into a series of decimal digits. Based on the observations we've
already made, this should be easy. \Fig{fig4-9} shows our completed
listing.

Voila! From problem, to conceptual model, to code.

Note: this solution is not optimal. The present volume does not address
the optimization phase.

One more thought: Depending on who uses this application, we may
want to add error-checking. Fact is, the highest symbol we know is M; the
highest value we can represent is 3,999, or MMMCMXCIX.

\goodbreak
We might redefine ROMAN as follows:

%!! I have added an IF...THEN around ABORT" Too large".
%!! Without it, it does not seem to make sense, does it?
%!! (Pg 131)
\begin{Code}
: ROMAN  ( n)
   DUP  3999 >  ABORT" Too large"  ROMAN ;
\end{Code}

\begin{interview}
\person{Moore}:\index{M!Moore, Charles|(}

\begin{tfquot}
There's a definite sense of rightness when you've done it right. It may be
that feeling that distinguishes \Forth{} from other languages, where you
never feel you've really done well. In \Forth{}, it's the ``Aha!''
reaction. You want to run off and tell somebody.

Of course, nobody will appreciate it like you do.
\end{tfquot}\index{M!Moore, Charles|)}
\end{interview}

\begin{figure*}[tttt]
\caption{Roman numerals, solved.}
\labelfig{fig4-9}
\vspace{1ex}
\setcounter{screen}{20}
\begin{Screen}
\ Roman numerals                                         8/18/83
CREATE ROMANS    ( ones)  ASCII I  C,   ASCII V  C,
                 ( tens)  ASCII X  C,   ASCII L  C,
             ( hundreds)  ASCII C  C,   ASCII D  C,
            ( thousands)  ASCII M  C,
VARIABLE COLUMN#  ( current_offset)
: ONES       O COLUMN# ! ;
: TENS       2 COLUMN# ! ;
: HUNDREDS   4 COLUMN# ! ;
: THOUSANDS  6 COLUMN# ! ;

: COLUMN ( -- address-of-column)  ROMANS  COLUMN# @  + ;

\end{Screen}

\begin{Screen}
\ Roman numerals cont'd                                  8/18/83
: .SYMBOL  ( offset -- )  COLUMN +  C@ EMIT ;
: ONER    O .SYMBOL ;
: FIVER   1 .SYMBOL ;
: TENER   2 .SYMBOL ;

: ONERS  ( #-of-oners -- )
   ?DUP  IF  O DO  ONER  LOOP  THEN ;
: ALMOST  ( quotient-of-5/ -- )
   ONER  IF  TENER  ELSE  FIVER  THEN ;
: DIGIT  ( digit -- )
   5 /MOD  OVER  4 = IF  ALMOST  DROP  ELSE  IF FIVER THEN
   ONERS THEN ;

\end{Screen}

\begin{Screen}
\ Roman numerals cont'd                                  8/18/83
: ROMAN  ( number --)  1000 /MOD  THOUSANDS DIGIT
                        100 /MOD   HUNDREDS DIGIT
                         10 /MOD       TENS DIGIT
                                       ONES DIGIT  ;

\end{Screen}
\vspace{1ex}
\label{fig-fig4-9}
\end{figure*}%
\index{R!Roman numeral computation problem|)}%
\program{roman2}

