\section{Hide From Whom?}%
\index{F!forth@\Forth{}!information-hiding|(}%
\index{I!Information-hiding|(}
Because modern mainstream languages give a slightly different meaning
to the phrase ``information-hiding,'' we should clarify. From what, or
whom are we hiding information?

The newest traditional languages (such as Modula 2) bend over
backwards to ensure that modules hide internal routines and data
structures from other modules. The goal is to achieve module
independence (a minimum coupling). The fear seems to be that modules
strive to attack each other like alien antibodies. Or else, that evil
bands of marauding modules are out to clobber the precious family data
structures.

This is \emph{not} what we're concerned about. The purpose of hiding
information, as we mean it, is simply to minimize the effects of a
possible design-change by localizing things that might change within
each component.

\Forth{} programmers generally prefer to keep the program under their own
control and not to employ any techniques to physically hide data
structures. (Nevertheless a brilliantly simple technique for adding
Modula-type modules to \Forth{} has been implemented, in only three lines
of code, by \person{Dewey Val Shorre} \cite{shorre71}.)%
\index{I!Information-hiding|)}%
\index{F!forth@\Forth{}!information-hiding|)}


