\section{The Nine Phases of the Programming Cycle}%

As we've seen, \Forth{} integrates aspects of design with aspects of
implementation and maintenance. As a result, the notion of a ``typical
development cycle'' makes as much sense as a ``typical noise.''

But any approach is better than no approach, and indeed, some
approaches have worked out better than others. Here is a development
cycle that represents an ``average'' of the most successful approaches
used in software projects:%
\index{P!Programming cycle:!phases|(}
\begin{description}
\item[Analysis] \hfill\index{A!Analysis}
    \begin{enumerate}
    \item Discover the Requirements and Constraints
    \item Build a Conceptual Model of the Solution
    \item Estimate Cost/Schedule/Performance
    \end{enumerate}
\item[Engineering] \hfill
    \begin{enumerate}
    \setcounter{enumi}{3}
    \item Preliminary Design
    \item Detailed Design
    \item Implementation
    \end{enumerate}
\item[Usage] \hfill
    \begin{enumerate}
    \setcounter{enumi}{6}
    \item Optimization
    \item Validation and Debugging
    \item Maintenance
    \end{enumerate}
\end{description}
In this book we'll treat the first six stages of the cycle,
focusing on analysis, design, and implementation.

In a \Forth{} project the phases occur on several levels. Looking at a
project from the widest perspective, each of these steps could take a
month or more.  One step follows the next, like seasons.

%Page 039 in first edition

But \Forth{} programmers also apply these same phases toward
defining each word. The cycle then repeats on the order of minutes.%
\index{P!Programming cycle:!phases|)}

Developing an application with this rapid repetition of the
programming cycle is known as using the ``Iterative Approach.''

