\section{The Value of Planning}%
\index{P!Planning:!value of|(}%
\index{P!Programming cycle:!value of planning|(}

In the nine phases at the start of this chapter we listed five steps
\emph{before} ``implementation.'' Yet in \Chap{1} we saw that an
overindulgence in planning is both difficult and pointless.

Clearly you can't undertake a significant software
project---regardless of the language---without some degree of planning.
Exactly what degree is appropriate?%
\index{J!Johnson, Dave|(}
\begin{interview}
\noindent More than one \Forth{} programmer has expressed high regard for
\person{Dave Johnson}'s meticulous approach to planning. \person{Johnson}
is supervisor at Moore Products Co. in Springhouse, Pennsylvania. The firm
specializes in industrial instrumentation and process control
applications. Dave has been using \Forth{} since 1978.

He describes his approach:
\begin{tfquot}
Compared with many others that use \Forth{}, I suppose we take a more
formal approach. I learned this the hard way, though. My lack of
discipline in the early years has come back to haunt me.

We use two tools to come up with new products: a functional specification
and a design specification. Our department of Sales \& Applications comes
up with the functional specification, through customer contact.

Once we've agreed on what we're going to do, the functional
specification is turned over to our department. At that point we work
through a design, and come up with the design specification.

Up to this point our approach is no different from programming in any
language. But with \Forth{}, we go about designing somewhat
differently.  With \Forth{} you don't have to work 95\% through your
design before you can start coding, but rather 60\% before you can get
into the iterative process.

%Page 041 in first edition

A typical project would be to add a functional enhancement to one of
our products. For example, we have an intelligent terminal with disk
drives, and we need certain protocols for communicating with another
device. The project to design the protocols, come up with displays,
provide the operator interfaces, etc. may take several months. The
functional specification takes a month; the design specification takes
a month; coding takes three months; integration and testing take
another month.

This is the typical cycle. One project took almost two years, but six
or seven months is reasonable.

When we started with \Forth{} five years ago, it wasn't like that. When I
received a functional specification, I just started coding. I used a
cross between top-down and bottom-up, generally defining a structure,
and as I needed it, some of the lower level, and then returning with
more structure.

The reason for that approach was the tremendous pressure to show
something to management. We wound up never writing down what we were
doing. Three years later we would go back and try to modify the code,
without any documentation. \Forth{} became a disadvantage because it
allowed us to go in too early. It was fun to make the lights flash and
disk drives hum. But we didn't go through the nitty-gritty design
work. As I said, our ``free spirits'' have come back to haunt us.

Now for the new programmers, we have an established requirement: a
thorough design spec that defines in detail all the high-level \Forth{}
words---the tasks that your project is going to do. No more reading a
few pages of the functional specification, answering that, reading a
few more, answering that, etc.

No living programmer likes to document. By ensuring the design ahead
of time, we're able to look back several years later and remember what
we did.

I should mention that during the design phase there is some amount of
coding done to test out certain ideas. But this code may not be part
of the finished product. The idea is to map out your design.
\end{tfquot}%
\index{J!Johnson, Dave|)}
\end{interview}
\person{Johnson} advises us to complete the design specification
before starting to code, with the exception of needed preliminary
tests. The next interview backs up this point, and adds some
additional reasons.%
\begin{interview}
\index{T!Teleska, John|(}
\noindent \person{John Teleska} has been an independent software
consultant since 1976, specializing in custom applications for
academic research environments.  He enjoys providing research tools
``right at the edge of what technology is able to do.''
\person{Teleska} works in Rochester, New York:

\begin{tfquot}
I see the software development process as having two phases. The first is
making sure I understand what the problem is. The second is
implementation, including debugging, verification, etc.

My goal in Phase One is an operational specification. I start with a
problem description, and as I proceed it becomes the operational
specification. My understanding of the problem metamorphoses into a
solution. The better the understanding, the more complete the
solution. I look for closure; a sense of having no more questions that
aren't answered in print.

I've found that on each project I've been putting more time into Phase
One, much to the initial dismay of many of my clients. The limiting
factor is how
%Page 042 in first edition
much I can convince the client it's necessary to spend that time up
front.  Customers generally don't know the specifications for the job
they want done. And they don't have the capital---or don't feel they
do---to spend on good specs. Part of my job is to convince them it
will end up costing more time and money not to.

Some of Phase One is spent on feasibility studies. Writing the spec
unearths uncertainties. I try to be as uncertain about uncertainties
as possible. For instance, they may want to collect 200,000 samples a
second to a certain accuracy. I first need to find out if it's even
possible with the equipment they've got. In this case I've got to test
its feasibility by writing a patch of code.

Another reason for the spec is to cover myself. In case the
application performs to the spec but doesn't fully satisfy the
customer, it's the customer's responsibility. If the customer wants
more, we'll have to renegotiate. But I see it as the designer's
responsibility to do whatever is necessary to generate an operational
specification that will do the job to the customer's satisfaction.

I think there are consultants who bow to client pressure and limit the
time they spend on specs, for fear of losing the job. But in these
situations nobody ends up happy.
\end{tfquot}
\index{T!Teleska, John|)}
\end{interview}%
We'll return to the \person{Teleska} interview momentarily.%
\index{P!Planning:!value of|)}%
\index{P!Programming cycle:!value of planning|)}

