\section{The Iterative Approach}%
\index{I!Iterative approach|(}%
\index{P!Programming cycle:!iterative approach|(}

The iterative approach was explained eloquently by
\person{Kim Harris} \cite{harris81}. \index{H!Harris, Kim|(}
He begins by describing the scientific method:

\begin{tfquot}
\dots{} a never-ending cycle of discovery and refinement. It first
studies a natural system and gathers observations about its behavior.
Then the observations are modeled to produce a theory about the
natural system. Next, analysis tools are applied to the model, which
produces predictions about the real system's behavior. Experiments
are devised to compare actual behavior to the predicted behavior. The
natural system is again studied, and the model is revised.

\wepsfigb{fig2-1}{The iterative approach to the
software development cycle, from ``The \Forth{} Philosophy,''
by \person{Kim Harris}, \emph{Dr.\@ Dobb's Journal.}}%
\index{H!Harris, Kim|)}

The \emph{goal} of the method is to produce a model which accurately
predicts all observable behavior of the natural system.
\end{tfquot}
\person{Harris} then applies the scientific method to the software
development cycle, illustrated in \Fig{fig2-1}:

\begin{enumerate}
\item A problem is analyzed to determine what functions are required
in the solution.
\item Decisions are made about how to achieve those functions with
the available resources.
\item A program is written which attempts to implement the design.
\item The program is tested to determine if the functions were
implemented correctly.
\end{enumerate}
%Page 040 in first edition
Mr. \person{Harris} adds:

\begin{tfquot}
Software development in \Forth{} seeks first to find the simplest
solution to a given problem. This is done by implementing selected
parts of the problem separately and by ignoring as many constraints as
possible. Then one or a few constraints are imposed and the program is
modified.
\end{tfquot}
An excellent testimonial to the development/testing model of design is
evolution. From protozoa to tadpoles to people, each species along the
way has consisted of functional, living beings. The Creator does not
appear to be a top-down designer.

\begin{tip}
Start simple. Get it running. Learn what you're trying to do. Add
complexity gradually, as needed to fit the requirements and
constraints. Don't be afraid to restart from scratch.
\end{tip}%
\index{I!Iterative approach|)}%
\index{P!Programming cycle:!iterative approach|)}

