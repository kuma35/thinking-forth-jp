\section{The Language of Performance}%
\index{F!forth@\Forth{}!performance of|(}%
\index{P!Performance|(}
Although performance is not the main topic of this book, the newcomer
to \Forth{} should be reassured that its advantages aren't purely
philosophical.  Overall, \Forth{} outdoes all other high-level languages
in speed, capability and compactness.


\subsection{Speed}%
\index{F!forth@\Forth{}!speed of|(}%
\index{S!Speed|(}
Although \Forth{} is an interpretive language, it executes compiled code.
Therefore it runs about ten times faster than interpretive BASIC.

\Forth{} is optimized for the execution of words by means of a technique
known as ``threaded code'' \cite{bell72}, \cite{dewar},
\cite{kogge82}. The penalty for modularizing into very small pieces of
code is relatively slight.

It does not run as fast as assembler code because the inner
interpreter (which interprets the list of addresses that comprise each
colon definition) may consume up to 50\% of the run time of primitive
words, depending on the processor.

But in large applications, \Forth{} comes very close to the speed of
assembler. Here are three reasons:

First and foremost, \Forth{} is simple. \Forth{}'s use of a data stack%
\index{D!Data stacks}
greatly reduces the performance cost of passing arguments from word to
word. In most languages, passing arguments between modules is one of
the main reasons that the use of subroutines inhibits performance.

Second, \Forth{} allows you to define words either in high-level or in
machine language. Either way, no special calling sequence is needed.
You can write a new definition in high level and, having verified that
it is correct, rewrite it in assembler without changing any of the
code that uses it. In a typical application, perhaps 20\% of the code
will be running 80\% of the time. Only the most often used,
time-critical routines need to be machine coded. The \Forth{} system
itself is largely implemented in machine-code definitions, so you'll
have few application words that need to be coded in assembler.

\wepsfigp{img1-033}{The best top-down designs of mice and young men.}

Third, \Forth{} applications tend to be better designed than those
written entirely in assembler. \Forth{} programmers take advantage of the
language's prototyping capabilities and try out several algorithms
before settling on the one best suited for their needs. Because \Forth{}
encourages change, it can also be called the language of optimization.

\Forth{} doesn't guarantee fast applications. It does give the programmer
a creative environment in which to design fast applications.%
\index{F!forth@\Forth{}!speed of|)}%
\index{S!Speed|)}

\subsection{Capability}
\Forth{}\index{C!Capability}\index{F!forth@\Forth{}!capability of}
can do anything any other language can do---usually easier.%
\index{F!forth@\Forth{}!advantages of}

At the low end, nearly all \Forth{} systems include assemblers%
\index{A!Assemblers}. These
support control-structure operators for writing conditionals and loops
using structured programming techniques. They usually allow you to
write interrupts---you can even write interrupt code in high level if
desired.

Some \Forth{} systems are multitasked, allowing you to add as many
foreground or background tasks as you want.

\Forth{} can be written to run on top of any operating system such as
RT-11, CP/M, or MS-DOS---or, for those who prefer it, \Forth{} can be
written as a self-sufficient operating system including its own
terminal drivers and disk drivers.

With a \Forth{} cross-compiler%
\index{C!Cross-compilers} or
target compiler,%
\index{C!Compilers}%
\index{T!Target compilers}
you can use \Forth{} to recreate new \Forth{} systems, for the same computer
or for different computers. Since \Forth{} is written in \Forth{}, you have
the otherwise unthinkable opportunity to rewrite the operating system
according to the needs of your application. Or you can transport
streamlined versions of applications over to embedded systems. 


\subsection{Size}%
\index{F!forth@\Forth{}!size of|(}
There are two considerations here: the size of the root \Forth{} system,
and the size of compiled \Forth{} applications.

The \Forth{} nucleus is very flexible. In an embedded application, the
part of \Forth{} you need to run your application can fit in as little
as 1K. In a full development environment, a multitasked \Forth{} system
including interpreter, compiler, assembler,%
\index{A!Assemblers}
editor, operating system, and all other support utilities averages
16K. This leaves plenty of room for applications. (And some \Forth{}s on
the newer processors handle 32-bit addressing, allowing unimaginably
large programs.)

Similarly, \Forth{} compiled applications tend to be very small---usually
smaller than equivalent assembly language programs. The reason, again,
is threaded code. Each reference to a previously defined word, no
matter how powerful, uses only two bytes.

One of the most exciting new territories for \Forth{} is the production
of \Forth{} chips such as the
Rockwell R65F11 \Forth{}-based microprocessor%
\index{R!Rockwell R65F11 \Forth{}-based microprocessor}
\cite{dumse}. The chip includes not only hardware features but also
the run-time portions of the \Forth{} language and operating system for
dedicated applications. Only \Forth{}'s architecture and compactness make
\Forth{}-based micros possible.%
\index{F!forth@\Forth{}!size of|)}%
\index{F!forth@\Forth{}!performance of|)}%
\index{P!Performance|)}


