\section{More Tips for Readability}

Here are some final suggestions to make your code more readable.
(Definitions appear in \App{C}.)

One constant that pays for itself in most applications is
\forth{BL}\index{B!Blank space (BL)}
(the ASCII value for ``blank-space'').

The word \forthb{ASCII} is used primarily within colon definitions to
free you from having to know the literal value of an ASCII character.
For instance, instead of writing:
\begin{Code}
: (    41 WORD  DROP ;  IMMEDIATE
\end{Code}
where 41 is the ASCII representation for right-parenthesis, you can
write
\begin{Code}
: (    ASCII ) WORD  DROP ;  IMMEDIATE
\end{Code}
\index{T!TRUE|(}%
\index{F!FALSE|(}%
A pair of words that can make dealing with booleans more readable are
\forthb{TRUE} and \forthb{FALSE}.  With these additions you can write
phrases such as

%Page 170 in first edition.
\begin{Code}
TRUE 'STAMP? !
\end{Code}
to set a flag or
\begin{Code}
FALSE 'STAMP? !
\end{Code}
to clear it.

(I once used \forthb{T} and \forthb{F}, but the words are
needed so rarely I now heed the injunction against abbreviations.)

As part of your application (not necessarily part of your \Forth{}
system), you can take this idea a step further and define:
\begin{Code}
: ON   ( a)  TRUE SWAP ! ;
: OFF   ( a)  FALSE SWAP ! ;
\end{Code}
\index{T!TRUE|)}\index{F!FALSE|)}\index{O!ON}\index{O!OFF}%
These words allow you to write:
\begin{Code}
'STAMP? ON
\end{Code}
or
\begin{Code}
'STAMP? OFF
\end{Code}
Other names for these definitions include \forth{SET} and
\forth{RESET}, although \forth{SET} and \forth{RESET} most commonly
use bit masks to manipulate individual bits.

An often-used word is \forthb{WITHIN},\index{W!WITHIN} which determines
whether a given value lies within two other values.  The syntax is:
\begin{Code}
n  lo hi WITHIN
\end{Code}
where ``n'' is the value to be tested and ``lo'' and ``hi'' represent
the range. \forthb{WITHIN} returns true if ``n'' is
\emph{greater-than or equal-to} ``lo'' and \emph{less-than}
``hi.'' This use of the non-inclusive upper limit parallels the syntax
of \forthb{DO }\forthb{LOOP}s.

\person{Moore}\index{M!Moore, Charles} recommends the word \forthb{UNDER+}.
It's useful for adding a value to the number just under the top stack
item, instead of to the top stack item.  It could be implemented in
high level as:
\begin{Code}
: UNDER+  ( a b c -- a+c b )  ROT +  SWAP ;
\end{Code}

