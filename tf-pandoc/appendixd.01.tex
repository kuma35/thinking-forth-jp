\section{\Chap{3}}
\newcounter{exercise}
\begin{enumerate}
\item The answer depends on whether you believe that other components will
need to ``know the numeric code associated with each key.  Usually this
would \emph{not} be the case.  The simpler, more compact form is therefore
preferable.  Also in the first version, to add a new key would require a
change in two places.
\item The problem with the words \forth{RAM-ALLOT} and \forth{THERE} are
that they are \emph{time-dependent}:  we must execute them in a particular
order.  Our solution then will be to devise an interface to the RAM
allocation pointer that is not dependent on order; the way to do this is
to have a \emph{single} word which does both functions transparently.

Our word's syntax will be
\begin{Code}
: RAM-ALLOT   ( #bytes-to-allot -- starting-adr) 
    ... ;
\end{Code}
This syntax will remain the same whether we define it to allocate growing 
upward:
\begin{Code}
: RAM-ALLOT  ( #bytes-to-allot -- starting-adr)
    >RAM @  DUP ROT +  >RAM ! ;
\end{Code}
or to allocate growing downward:
\begin{Code}
: RAM-ALLOT  ( #bytes-to-allot -- starting-adr)
    >RAM @  SWAP -  DUP >RAM ! ;
\end{Code}
\end{enumerate}

