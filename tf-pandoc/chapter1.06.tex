\section{Hiding the Construction of Data Structures}%
\index{D!Data structures:!hiding construction of|(}
We've noted two inventions of \Forth{} that make possible the methodology
we've described---implicit calls and implicit data passing%
\index{D!Data passing, implicit}. A third
feature allows the data structures within a component to be described
in terms of previously-defined components. This feature is direct
access memory.

Suppose we define a variable called \forth{APPLES}, like this:\program{apples1}
\begin{Code}
VARIABLE APPLES
\end{Code}
We can store a number into this variable to indicate how many apples
we currently have:
\begin{Code}
20 APPLES !
\end{Code}
We can display the contents of the variable:
\begin{Code}[commandchars=\&\{\}]
APPLES ? &underline{20 ok}
\end{Code}
We can up the count by one:
\begin{Code}
1 APPLES +!
\end{Code}
(The newcomer can study the mechanics of these phrases in Appendix A.)

The word \forth{APPLES} has but one function: to put on the stack the
\emph{address} of the memory location where the tally of apples is
kept. The tally can be thought of as a ``thing,'' while the words that
set the tally, read the tally, or increment the tally can be
considered as ``actions.''

\Forth{} conveniently separates ``things'' from ``actions'' by allowing
addresses of data structures to be passed on the stack and providing
the ``fetch'' and ``store'' commands.

We've discussed the importance of designing around things that may
change. Suppose we've written a lot of code using this variable
\forth{APPLES}.  And now, at the eleventh hour, we discover that we
must keep track of two different kinds of apples, red and green!

We needn't wring our hands, but rather remember the function of
\forth{APPLES}: to provide an address. If we need two separate
tallies, \forth{APPLES} can supply two different addresses depending
on which kind of apple we're currently talking about. So we define a
more complicated version of
\forth{APPLES} as follows:

\begin{Code}
VARIABLE COLOR  ( pointer to current tally)
VARIABLE REDS  ( tally of red apples)
VARIABLE GREENS  ( tally of green apples)
: RED  ( set apple-type to RED)  REDS COLOR ! ;
: GREEN  ( set apple-type to GREEN)  GREENS COLOR ! ;
: APPLES  (  -- adr of current apple tally)  COLOR @ ;
\end{Code}

\wepsfiga{fig1-10}{Changing the indirect pointer.}

\noindent Here we've redefined \forth{APPLES.} Now it fetches the contents
of a variable called \forth{COLOR.} \forth{COLOR} is a pointer, either to
the variable \forth{REDS} or to the variable \forth{GREENS.} These two
variables are the real tallies.

If we first say \forth{RED,} then we can use \forth{APPLES} to refer
to red apples.  If we say \forth{GREEN,} we can use it to refer to
green apples (\Fig{fig1-10}).

We didn't need to change the syntax of any existing code that uses
\forth{APPLES.} We can still say

\begin{Code}
20 APPLES !
\end{Code}
and

\begin{Code}
1 APPLES +!
\end{Code}
Look again at what we did. We changed the definition of \forth{APPLES}
from that of a variable to a colon definition, without affecting its
usage.  \Forth{} allows us to hide the details of how \forth{APPLES} is
defined from the code that uses it. What appears to be ``thing'' (a
variable) to the original code is actually defined as an ``action'' (a
colon definition) within the component.

\Forth{} encourages the use of abstract data types by allowing data
structures to be defined in terms of lower level components. Only
\Forth{}, which eliminates the CALLs from procedures, which allows
addresses and data to be implicitly passed via the stack, and which
provides direct access to memory locations with \forth{@} and
\forth{!}, can offer this level of information-hiding.

\Forth{} pays little attention to whether something is a data structure
or an algorithm. This indifference allows us programmers incredible
freedom in creating the parts of speech we need to describe our
applications.

I tend to think of any word which returns an address, such as
\forth{APPLES,} as a ``noun,'' regardless of how it's defined. A word
that performs an obvious action is a ``verb.''

Words such as \forth{RED} and \forth{GREEN} in our example can only be
called ``adjectives'' since they modify the function of
\forth{APPLES.} The phrase

\begin{Code}
RED APPLES ?
\end{Code}
is different from

\begin{Code}
GREEN APPLES ?
\end{Code}
\Forth{} words can also serve as adverbs and prepositions. There's little
value in trying to determine what part of speech a particular word is,
since \Forth{} doesn't care anyway. We need only enjoy the ease of
describing an application in natural terms.%
\program{apples2}%
\index{D!Data structures:!hiding construction of|)}


