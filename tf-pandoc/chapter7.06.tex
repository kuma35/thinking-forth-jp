\section{Application Stacks}

In the last section we examined some ways to save and restore a single
previous value. Some applications require \emph{several} values to be
saved and restored. You may often find the best solution to this problem
in defining your own stack.

Here is the code for a user stack including very simple error checking
(an error clears the stack):

\begin{Code}
CREATE STACK  12 ALLOT  \  { 2tos-pointer | 10stack [5 cells] }
HERE CONSTANT STACK>
: INIT-STACK   STACK STACK ! ;   INIT-STACK
: ?BAD  ( ?)   IF ." STACK ERROR "  INIT-STACK  ABORT  THEN ;
: PUSH  ( n)   2 STACK +!  STACK @  DUP  STACK> = ?BAD  ! ;
: POP  ( -- n)  STACK @ @  -2 STACK +!  STACK @ STACK < ?BAD ;
\end{Code}
The word \forth{PUSH} takes a value from off of your data stack and
``pushes'' it onto this new stack. \forth{POP} is the opposite,
``popping'' a value from off the new stack, and onto \Forth{}'s data
stack.

In a real application you might want to change the names \forth{PUSH}
and \forth{POP} to better match their conceptual purposes.

