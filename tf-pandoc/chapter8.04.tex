\section{Summary}

The use of logic and conditionals as a significant structural element in
programming leads to overly-complicated, difficult-to-maintain, and
inefficient code. In this chapter we've discussed several ways to minimize,
optimize or eliminate unnecessary conditional structures.

As a final note, \Forth{}'s downplaying of conditionals is not shared
by most contemporary languages. In fact, the Japanese are basing their
fifth-generation computer project on a language called PROLOG---for
PROgramming in LOGic---in which one programs entirely in logic. It will
be interesting to see the battle-lines forming as we ponder the question:

\begin{tfquot}
To \forthb{IF} or not to \forthb{IF}
\end{tfquot}
In this book we've covered the first six steps of the software development
cycle, exploring both the philosophical questions of designing software
and practical considerations of implementing robust, efficient, readable
software.

We have not discussed optimization, validation, debugging, documenting,
project management, \Forth{} development tools, assembler
definitions, uses and abuses of recursion, developing multiprogrammed
applications, or target compilation.

But that's another story.

\begin{references}{9}
\bibitem{eaker} \person{Charles Eaker}, ``Just in Case,'' \emph{\Forth{}
Dimensions} II/3, p. 37.
\end{references}

