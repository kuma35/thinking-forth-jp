\section{How to Eliminate Control Structures}

In this section we'll study numerous techniques for simplifying or
avoiding conditionals. Most of them will produce code that is more
readable, more maintainable, and more efficient. Some of the techniques
produce code that is more efficient, but not always as readable.
Remember, therefore: Not all of the tips will be applicable in all
situations.

\subsection{Using the Dictionary}%
\index{C!Control structure minimization!dictionary|(}%
\index{D!Dictionary:!control structure minimization with|(}

\begin{tip}
Give each function its own definition.
\end{tip}
By using the \Forth{} dictionary properly, we're not actually eliminating
conditionals; we're merely factoring them out from our application code.
The \Forth{} dictionary is a giant string case statement. The match and
execute functions are hidden within the \Forth{} system.

\begin{interview}
\person{Moore}:\index{M!Moore, Charles|(}
\begin{tfquot}
In my accounting package, if you receive a check from somebody, you type
the amount, the check number, the word \forth{FROM}, and the person's
name:

\begin{Code}
200.00 127 FROM ALLIED
\end{Code}
The word \forth{FROM} takes care of that situation. If you want to bill
someone, you type the amount, the invoice number, the word \forth{BILL}
and the person's name:

\begin{Code}
1000.00 280 BILL TECHNITECH
\end{Code}
\end{tfquot}\index{M!Moore, Charles|)}
\dots{} One word for each situation. The dictionary is making the decision.\medskip
\end{interview}
This notion pervades \Forth{} itself. To add a pair of single-length
numbers we use the command \forth{+}. To add a pair of double-length
numbers we use the command \forth{D+}. A less efficient, more complex
approach would be a single command that somehow ``knows'' which type of
numbers are being added.

\Forth{} is efficient because all these words---\forth{FROM} and
\forth{BILL} and \forth{+} and \forth{D+}---can be implemented without any
need for testing and branching.

\index{D!Dumb words|(}
\begin{tip}
Use dumb words.
\end{tip}
This isn't advice for TV writers. It's another instance of using the
dictionary. A ``dumb'' word is one that is not state-dependent, but
instead, has the same behavior all the time (``referentially
transparent'').

A dumb word is unambiguous, and therefore, more trustworthy.

A few common \Forth{} words have been the source of controversy recently
over this issue. One such word is \forth{."} which prints a string.  In
its simplest form, it's allowed only inside a colon definition:

\begin{Code}
: TEST   ." THIS IS A STRING " ;
\end{Code}
Actually, this version of the word does \emph{not} print a string. It
\emph{compiles} a string, along with the address of another definition
that does the printing at run time.

This is the dumb version of the word. If you use it outside a colon
definition, it will uselessly compile the string, not at all what a
beginner might expect.

To solve this problem, the FIG model added a test inside \forth{."} that
determined whether the system was currently compiling or interpreting.  In
the first case, \forth{."} would compile the string and the address of the
primitives; in the second case it would \forth{TYPE} it.

\forth{."} became two completely different words housed together in one
definition with an \forth{IF ELSE THEN} structure. The flag that indicates
whether \Forth{} is compiling or interpreting is called \forth{STATE}.
Since the \forth{."} depends on \forth{STATE}, it is said to be
``\forth{STATE}-dependent,'' literally.

The command \emph{appeared} to behave the same inside and outside a colon
definition. This duplicity proved useful in afternoon introductions to
\Forth{}, but the serious student soon learned there's more to it than
that.

Suppose a student wants to write a new word called \forthb{B."} (for
``bright-dot-quote'') to display a string in bright characters on her
display, to be used like this:

\begin{Code}
." INSERT DISK IN "  B." LEFT "  ." DRIVE "
\end{Code}
She might expect to define \forth{B."} as

\begin{Code}
: B."   BRIGHT  ."  NORMAL ;
\end{Code}
that is, change the video mode to bright, print the string, then reset the
mode to normal.

She tries it. Immediately the illusion is destroyed; the deception is
revealed; the definition won't work.

To solve her problem, the programmer will have to study the definition of
\forth{(.")} in her own system. I'm not going to get sidetracked here with
explaining how \forth{(.")} works---my point is that smartness isn't all
it appears to be.

Incidentally, there's a different syntactical approach to our student's
problem, one that does not require having two separate words, \forth{."}
and \forth{B."} to print strings. Change the system's \forth{(.")} so that
it always sets the mode to normal after typing, even though it will
already be normal most of the time. With this syntax, the programmer need
merely precede the emphasized string with the simple word \forth{BRIGHT}.

\begin{Code}
." INSERT DISK IN "  BRIGHT ." LEFT "  ." DRIVE "
\end{Code}
The '83 Standard now specifies a dumb \forth{."} and, for those cases
where an interpretive version is wanted, the new word \forth{.(} has been
added. Happily, in this new standard we're using the dictionary to make a
decision by having two separate words.

\index{S!STATE|(}
The word \forth{'} (tick) has a similar history. It was
\forthb{STATE}-dependent in fig-\Forth{}, and is now dumb in the '83
Standard. Tick shares with dot-quote the characteristic that a programmer
might want to reuse either of these words in a higher-level definition and
have them behave in the same way they do normally.%
\index{D!Dumb words|)}

\begin{tip}
Words should not depend on \forthb{STATE} if a programmer might ever want
to invoke them from within a higher-level definition and expect them to
behave as they do interpretively.
\end{tip}
\forthb{ASCII}\index{A!ASCII} works well as a \forth{STATE}-dependent
word, and so does \forthb{MAKE}.\index{M!MAKE} (See \App{C}.)%
\index{S!STATE|)}%
\index{C!Control structure minimization!dictionary|)}%
\index{D!Dictionary:!control structure minimization with|)}

\subsection{Nesting and Combining Conditionals}%
\index{N!Nesting conditionals|(}%
\index{C!Conditionals, nesting and combining|(}%
\index{C!Control structure minimization!nesting and combining conditionals|(}
\begin{tip}
Don't test for something that has already been excluded.
\end{tip}
Take this example, please:

\begin{Code}
: PROCESS-KEY
   KEY  DUP  LEFT-ARROW  =  IF CURSOR-LEFT  THEN
        DUP  RIGHT-ARROW =  IF CURSOR-RIGHT THEN
        DUP  UP-ARROW    =  IF CURSOR-UP    THEN
             DOWN-ARROW  =  IF CURSOR-DOWN  THEN ;
\end{Code}
This version is inefficient because all four tests must be made regardless
of the outcome of any of them. If the key pressed was the left-arrow key,
there's no need to check if it was some other key.

Instead, you can nest the conditionals, like this:

\begin{Code}
: PROCESS-KEY
   KEY  DUP  LEFT-ARROW  =  IF CURSOR-LEFT  ELSE
        DUP  RIGHT-ARROW =  IF CURSOR-RIGHT ELSE
        DUP  UP-ARROW    =  IF CURSOR-UP    ELSE
                               CURSOR-DOWN
           THEN THEN THEN  DROP ;
\end{Code}
\begin{tip}
Combine booleans of similar weight.
\end{tip}
Many instances of doubly-nested \forthb{IF }\forthb{THEN} structures can
be simplified by combining the flags with logical operators before making
the decision.
Here's a doubly-nested test:
\begin{Code}
: ?PLAY   SATURDAY? IF  WORK FINISHED? IF
     GO PARTY  THEN  THEN ;
\end{Code}
The above code uses nested \forthb{IF}s to make sure that it's both
Saturday and the chores are done before it boogies on down. Instead,
let's combine the conditions logically and make a single decision:

\begin{Code}
: ?PLAY   SATURDAY?  WORK FINISHED? AND  IF
   GO PARTY  THEN ;
\end{Code}
It's simpler and more readable.

The logical ``or'' situation, when implemented with
\forthb{IF }\forthb{THEN}s, is even clumsier:

\begin{Code}
: ?RISE    PHONE RINGS? IF  UP GET  THEN
     ALARM-CLOCK RINGS?  IF UP GET THEN ;
\end{Code}
This is much more elegantly written as

\begin{Code}
: ?RISE  PHONE RINGS?  ALARM RINGS? OR  IF  UP GET THEN ;
\end{Code}
One exception to this rule arises when the speed penalty for checking
some of the conditions is too great.

We might write

\begin{Code}
: ?CHOW-MEIN   BEAN-SPROUTS?  CHOW-MEIN RECIPE?  AND IF
   CHOW-MEIN PREPARE  THEN ;
\end{Code}
But suppose it's going to take us a long time to hunt through our recipe
file to see if there's anything on chow mein. Obviously there's no point in
undertaking the search if we have no bean sprouts in the fridge. It would
be more efficient to write

\begin{Code}
: ?CHOW-MEIN   BEAN-SPROUTS? IF  CHOW-MEIN RECIPE? IF
   CHOW-MEIN PREPARE THEN   THEN ;
\end{Code}
We don't bother looking for the recipe if there are no sprouts.

Another exception arises if any term is probably not true. By
eliminating such a condition first, you avoid having to try the other
conditions.

\begin{tip}
When multiple conditions have dissimilar weights (in likelihood or
calculation time) nest conditionals with the term that is least likely
to be true or easiest to calculate on the outside.
\end{tip}
Trying to improve performance in this way is more difficult with the
\forth{OR} construct. For instance, in the definition

\begin{Code}
: ?RISE  PHONE RINGS?  ALARM RINGS? OR  IF  UP GET THEN ;
\end{Code}
we're testing for the phone and the alarm, even though only one of them
needs to ring for us to get up. Now suppose it were much more difficult to
determine that the alarm clock was ringing. We could write

\begin{Code}
: ?RISE   PHONE RINGS? IF  UP GET  ELSE
     ALARM-CLOCK RINGS?  IF UP GET THEN THEN  ;
\end{Code}
If the first condition is true, we don't waste time evaluating the second.
We have to get up to answer the phone anyway.

The repetition of \forth{UP GET} is ugly---not nearly as readable as the
solution using \forth{OR}---but in some cases desirable.%
\index{C!Conditionals, nesting and combining|)}%
\index{C!Control structure minimization!nesting and combining conditionals|)}
\index{N!Nesting conditionals|)}%

\subsection{Choosing Control Structures}%
\index{C!Control structures:!choosing|(}
\begin{tip}
The most elegant code is that which most closely matches the problem.
Choose the control structure that most closely matches the control-flow
problem.
\end{tip}%
\index{C!Control structures:!choosing|)}

\subsubsection{Case Statements}%
\index{C!Case statements|(}%
\index{C!Control structure minimization!case statements|(}

A particular class of problem involves selecting one of several possible
paths of execution according to a numeric argument. For instance, we
want the word \forth{.SUIT} to take a number representing a suit of playing
cards, 0 through 3, and display the name of the suit. We might define this
word using nested \forthb{IF }\forthb{ELSE }\forthb{THEN}s, like this:

\begin{Code}
: .SUIT ( suit -- )
  DUP  O=  IF ." HEARTS "   ELSE
  DUP  1 = IF ." SPADES "   ELSE
  DUP  2 = IF ." DIAMONDS " ELSE
              ." CLUBS "
  THEN THEN THEN  DROP ;
\end{Code}
We can solve this problem more elegantly by using a ``case statement.''

Here's the same definition, rewritten using the ``\person{Eaker} case
statement'' format, named after Dr.\@ \person{Charles E. Eaker}, the
gentleman who proposed it \cite{eaker}.

\begin{Code}
: .SUIT ( suit -- )
  CASE
  O OF   ." HEARTS "    ENDOF
  1 OF   ." SPADES "    ENDOF
  2 OF   ." DIAMONDS "  ENDOF
  3 OF   ." CLUBS "     ENDOF     ENDCASE ;
\end{Code}
The case statement's value lies exclusively in its readability and
writeability. There's no efficiency improvement either in object memory or
in execution speed. In fact, the case statement compiles much the same
code as the nested \forthb{IF }\forthb{THEN} statements. A case statement
is a good example of compile-time factoring.

Should all \Forth{} systems include such a case statement? That's a matter
of controversy. The problem is twofold. First, the instances in which a
case statement is actually needed are rare---rare enough to question its
value. If there are only a few cases, a nested \forthb{IF }\forthb{ELSE
}\forthb{THEN} construct will work as well, though perhaps not as
readably. If there are many cases, a decision table is more flexible.

Second, many case-like problems are not quite appropriate for the case
structure. The \person{Eaker} case statement assumes that you're testing
for equality against a number on the stack. In the instance of
\forth{.SUIT}, we have contiguous integers from zero to three. It's more
efficient to use the integer to calculate an offset and directly jump to
the right code.

In the case of our Tiny Editor, later in this chapter, we have not one, but
two, dimensions of possibilities. The case statement doesn't match that
problem either.

Personally, I consider the case statement an elegant solution to a
misguided problem: attempting an algorithmic expression of what is
more aptly described in a decision table.

A case statement ought to be part of the application when useful,
but not part of the system.%
\index{C!Case statements|)}%
\index{C!Control structure minimization!case statements|)}

\subsubsection{Looping Structures}%
\index{C!Control structure minimization!looping structures|(}%
\index{L!Loops|(}
The right looping structure can eliminate extra conditionals.
\begin{interview}
\person{Moore}:\index{M!Moore, Charles|(}
\begin{tfquot}
Many times conditionals are used to get out of loops. That particular use
can be avoided by having loops with multiple exit points.

This is a live topic, because of the multiple \forthb{WHILE} construct
which is in poly\Forth{} but hasn't percolated up to \Forth{} '83. It's a
simple way of defining multiple \forthb{WHILE}s in the same
\forthb{REPEAT}.

Also \person{Dean Sanderson}\index{S!Sanderson, Dean} [of \Forth{}, Inc.]
has invented a new construct that introduces two exit points to a
\forthb{DO }\forthb{LOOP}.\index{D!DO LOOP} Given that construction you'll
have fewer tests. Very often I leave a truth value on the stack, and if
I'm leaving a loop early, I change the truth value to remind myself that I
left the loop early. Then later I'll have an \forthb{IF} to see whether I
left the loop early, and it's just clumsy.

Once you've made a decision, you shouldn't have to make it again. With the
proper looping constructs you won't need to remember where you came from,
so more conditionals will go away.

This is not completely popular because it's rather unstructured. Or perhaps
it is elaborately structured. The value is that you get simpler programs.
And it costs nothing.
\end{tfquot}\index{M!Moore, Charles|)}
\end{interview}
Indeed, this is a live topic. As of this writing it's too early to make
any specific proposals for new loop constructs. Check your system's
documentation to see what it offers in the way of exotic looping
structures.  Or, depending on the needs of your application, consider
adding your own conditional constructs. It's not that hard in \Forth{}.

I'm not even sure whether this use of multiple exits doesn't violate the
doctrine of structured programming. In a
\forthb{BEGIN }\forthb{WHILE }\forthb{REPEAT}
loop with multiple \forthb{WHILE}s, all the exits bring you to a common
``continue'' point: the \forthb{REPEAT}. But with \person{Sanderson}'s
\index{S!Sanderson, Dean} construct, you
can exit the loop by jumping \emph{past} the end of the loop, continuing
at an \forthb{ELSE}. There are two possible ``continue'' points.

This is ``less structured,'' if we can be permitted to say that. And
yet the definition will always conclude at its semicolon and return to the
word that invoked it. In that sense it is well-structured; the module has
one entry point and one exit point.

When you want to execute special code only if you did \emph{not} leave the
loop prematurely, this approach seems the most natural structure to use.
(We'll see an example of this in a later section, ``Using Structured
Exits.'')
%
\index{C!Counts vs. terminators|(}
\index{T!Terminators vs. counts|(}
\begin{tip}
Favor counts over terminators.
\end{tip}
\Forth{} handles strings by saving the length of the string in the first
byte. This makes it easier to type, move, or do practically anything with
the string. With the address and count on the stack, the definition of
\forthb{TYPE} can be coded:

\begin{Code}
: TYPE  ( a #)  OVER + SWAP DO  I C@ EMIT  LOOP ;
\end{Code}
(Although \forthb{TYPE} really ought to be written in machine code.)

This definition uses no overt conditional. \forthb{LOOP} actually hides the
conditional since each loop checks the index and returns to \forthb{DO} if it
has not yet reached the limit.

If a delimiter were used, let's say ASCII null (zero), the definition
would have to be written:

\begin{Code}
: TYPE  ( a)  BEGIN DUP C@  ?DUP WHILE  EMIT  1+
   REPEAT  DROP ;
\end{Code}
An extra test is needed on each pass of the loop. (\forthb{WHILE} is a
conditional operator.)

Optimization note: The use of \forthb{?DUP} in this solution is expensive
in terms of time because it contains an extra decision itself. A faster
definition would be:

\begin{Code}
: TYPE  ( a)  BEGIN DUP C@  DUP WHILE EMIT 1+
    REPEAT  2DROP ;
\end{Code}
The '83 Standard applied this principle to \forthb{INTERPRET}
\index{I!INTERPRET}
which now accepts a count rather than looking for a terminator.

The flip side of this coin is certain data structures in which it's
easiest to \emph{link} the structures together. Each record points to the
next (or previous) record. The last (or first) record in the chain can be
indicated with a zero in its link field.

If you have a link field, you have to fetch it anyway. You might as
well test for zero. You don't need to keep a counter of how many records
there are. If you decrement a counter to decide whether to terminate,
you're making more work for yourself. (This is the technique used to
implement \Forth{}'s dictionary as a linked list.)%
\index{C!Counts vs. terminators|)}%
\index{C!Control structure minimization!looping structures|)}%
\index{L!Loops|)}%
\index{T!Terminators vs. counts|)}%

\subsubsection{Calculating Results}%
\index{C!Control structure minimization!calculating results|(}%
\index{C!Calculations|(}%

\begin{tip}
Don't decide, calculate.
\end{tip}
Many times conditional control structures are applied mistakenly to
situations in which the difference in outcome results from a difference in
numbers. If numbers are involved, we can calculate them. (In Chapter
Four see the section called ``Calculations vs. Data Structures vs. Logic.'')
%
\index{B!Booleans, as hybrid values|(}
\begin{tip}
Use booleans as hybrid values.
\end{tip}
This is a fascinating corollary to the previous tip, ``Don't decide,
calculate.'' The idea is that booleans, which the computer represents as
numbers, can efficiently be used to effect numeric decisions. Here's one
example, found in many \Forth{} systems:

\begin{Code}
: S>D  ( n -- d)  \ sign extend s to d
     DUP O<  IF -1  ELSE  O THEN ;
\end{Code}
(The purpose of this definition is to convert a single-length number to
double-length. A double-length number is represented as two 16-bit
values on the stack, the high-order part on top. Converting a positive
integer to double-length merely means adding a zero onto the stack, to
represent its high-order part. But converting a negative integer to
double-length requires ``sign extension;'' that is, the high-order part
should be all ones.)

The above definition tests whether the single-length number is negative.
If so, it pushes a negative one onto the stack; otherwise a zero.  But
notice that the outcome is merely arithmetic; there's no change in
process. We can take advantage of this fact by using the boolean itself:

\begin{Code}
: S>D  ( n -- d)  \ sign extend s to d
     DUP  O< ;
\end{Code}
This version pushes a zero or negative one onto the stack without a
moment's (in)decision.

\medbreak
(In pre-1983 systems, the definition would be:

\begin{Code}
: S>D  ( n -- d)  \ sign extend s to d
     DUP  O< NEGATE ;
\end{Code}
See \App{C}.)%
\index{B!Booleans, as hybrid values|)}\medbreak

\index{H!Hybrid values|(}%
We can do even more with ``hybrid values'':
%
\index{A!AND|(}
\begin{tip}
To effect a decision with a numeric outcome, use \forthb{AND}.
\end{tip}
In the case of a decision that produces either zero or a non-zero ``$n$,''
the traditional phrase

\begin{Code}
( ? ) IF  n  ELSE  O  THEN
\end{Code}
is equivalent to the simpler statement

\begin{Code}
( ? )  n AND
\end{Code}
Again, the secret is that ``true'' is represented by $-1$ (all ones) in
'83 \Forth{} systems. \forthb{AND}ing ``$n$'' with the flag will either
produce ``$n$'' (all bits intact) or ``$0$'' (all bits cleared).

To restate with an example:

\begin{Code}
( ? )  IF  200  ELSE  O  THEN
\end{Code}
is the same as

\begin{Code}
( ? )  200 AND
\end{Code}
Take a look at this example:

\begin{Code}
n  a b <  IF  45 +  THEN
\end{Code}
This phrase either adds 45 to ``$n$'' or doesn't, depending on the
relative sizes of ``$a$'' and ``$b$.'' Since ``adding 45 or not'' is the
same as ``adding 45 or adding 0,'' the difference between the two outcomes
is purely numeric.  We can rid ourselves of a decision, and simply
compute:

\begin{Code}
n  a b <  45 AND  +
\end{Code}
\begin{interview}
\person{Moore}:\index{M!Moore, Charles|(}
\begin{tfquot}
The ``\forth{45 AND}'' is faster than the \forth{IF}, and certainly more
graceful. It's simpler. If you form a habit of looking for instances where
you're calculating this value from that value, then usually by doing
arithmetic on the logic you get the same result more cleanly.

I don't know what you call this. It has no terminology; it's merely doing
arithmetic with truth values. But it's perfectly valid, and someday boolean
algebra and arithmetic expressions will accommodate it.

In books you often see a lot of piece-wise linear approximations that fail to
express things clearly. For instance the expression

\begin{Code}[commandchars=\&\{\}]
x = O for t < O
x = 1 for t &(&ge&) O
\end{Code}
This would be equivalent to
\begin{Code}
t  O<  1 AND
\end{Code}
as a single expression, not a piece-wise expression.
\end{tfquot}\index{M!Moore, Charles|)}
\end{interview}%%
\index{A!AND|)}

I call these flags ``hybrid values'' because they are booleans (truth
values) being applied as data (numeric values). Also, I don't know what
else to call them.%
\index{H!Hybrid values|)}

We can eliminate numeric \forth{ELSE} clauses as well (where both results
are non-zero), by factoring out the difference between the two results.
For instance,

\begin{Code}
: STEPPERS  'TESTING? @  IF 150 ELSE 151  THEN  LOAD ;
\end{Code}
can be simplified to

\begin{Code}
: STEPPERS   150  'TESTING? @  1 AND +  LOAD ;
\end{Code}
This approach works here because conceptually we want to either load
Screen 150, or if testing, the next screen past it.%
\index{C!Calculations|)}%
\index{C!Control structure minimization!calculating results|)}

