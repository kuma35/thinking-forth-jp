\section{Sharing Components}%
\index{C!Components:!sharing|(}%
\index{S!Sharing components|(}
\begin{tip}
It's legal to use a component for an additional purpose besides its
intended one, provided:
\medskip
\begin{enumerate}
\item All uses of the component are mutually exclusive
\item Each interrupting use of the component restores the component to
   its previous state when finished.
\end{enumerate}
\medskip\noindent
Otherwise you need an additional component or level of complexity.
\end{tip}%
\index{D!Data stacks!return|(}
We've seen a simple example of this principle with the return stack. The
return stack is a component of the \Forth{} system designed to hold return
addresses, and thereby serve as an indication of where you've been and
where you're going. To use the return stack as a holder for temporary
values is possible, and in many cases desirable. Problems occur when one
of the above restrictions is ignored.%
\index{D!Data stacks!return|)}

In my text formatter the output can go invisible. This feature has
two purposes: 
\begin{enumerate}
\item for looking ahead to see whether something will fit, and
\item for formatting the table of contents (the entire document is
formatted and page numbers are calculated without anything actually being
displayed).
\end{enumerate}

It was tempting to think that once having added the ability to make
the output invisible, I could use this feature to serve both purposes.
Unfortunately, the two purposes are not mutually exclusive.

Let's see what would happen if I tried to violate this rule. Imagine
that the word \forth{DISPLAY} does the output, and it's smart enough
to know whether to be visible or invisible. The words \forth{VISIBLE}
and \forth{INVISIBLE} set the state respectively.

My code for looking ahead will first execute \forth{INVISIBLE,} then
test-format the upcoming text to determine its length, and finally
execute \forth{VISIBLE} to restore things to the normal state.

This works fine.

Later I add the table-of-contents feature. First the code executes
\forth{IN\-VI\-SI\-BLE}, then runs through the document determining page
numbers etc.; then finally executes \forth{VISIBLE} to restore things
to normal.

The catch? Suppose I'm running a table of contents and I hit one of
those places where I look ahead. When I finish looking ahead, I
execute \forth{VISIBLE}. Suddenly I start printing the document when I
was supposed to be running the table of contents.

The solution? There are several.

One solution views the problem as being that the lookahead code is
clobbering the visible/invisible flag, which may have been preset by
table-of-contents. Therefore, the lookahead code should be responsible for
saving, and later restoring, the flag.

Another solution involves keeping two separate variables---one to
indicate we're looking ahead, the other to indicate we're printing the
table of contents. The word \forth{DISPLAY} requires that both flags
be false in order to actually display anything.

There are two ways to accomplish the latter approach, depending on
how you want to decompose the problem. First, we could nest one condition
within the other:

\begin{Code}
: [DISPLAY]  ...
     ( the original definition, always does the output) ... ;
VARIABLE 'LOOKAHEAD?  ( t=looking-ahead)
: <DISPLAY>   'LOOKAHEAD? @ NOT IF  [DISPLAY]  THEN ;
VARIABLE 'TOC?  ( t=setting-table-of-contents)
: DISPLAY   'TOC? @ NOT IF  <DISPLAY>  THEN ;
\end{Code}
{\sloppy
\forth{DISPLAY} checks that we're not setting the table of contents
and invokes \forth{<DISPLAY>}, which in turn checks that we're not
looking ahead and invokes \forth{[DISPLAY]}.

}

In the development cycle, the word \forth{[DISPLAY]} that always does
the output was originally called \forth{DISPLAY}. Then a new
\forth{DISPLAY} was defined to include the lookahead check, and the
original definition was renamed \forth{[DISPLAY]}, thus adding a level
of complexity backward without changing any of the code that used
\forth{DISPLAY}.

Finally, when the table-of-contents feature was added, a new
\forth{DISPLAY} was defined to include the table-of-contents check,
and the previous \forth{DISPLAY} was renamed \forth{<DISPLAY>}.

That's one approach to the use of two variables. Another is to include
both tests within a single word:

\begin{Code}
: DISPLAY   'LOOKAHEAD? @  'TOC @ OR  NOT IF [DISPLAY] THEN ;
\end{Code}
But in this particular case, yet another approach can simplify the whole
mess. We can use a single variable not as a flag, but as a counter.

We define:

\begin{Code}
VARIABLE 'INVISIBLE?  ( t=invisible)
: DISPLAY   'INVISIBLE? @  O= IF [DISPLAY] THEN ;
: INVISIBLE   1 'INVISIBLE? +! ;
: VISIBLE    -1 'INVISIBLE? +! ;
\end{Code}
The lookahead code begins by invoking \forth{INVISIBLE} which bumps
the counter up one. Non-zero is ``true,'' so \forth{DISPLAY} will not
do the output.  After the lookahead, the code invokes \forth{VISIBLE}
which decrements the counter back to zero (``false'').

The table-of-contents code also begins with \forth{VISIBLE} and ends
with \forth{IN\-VI\-SI\-BLE}. If we're running the table of contents while
we come upon a lookahead, the second invocation of \forth{VISIBLE}
raises the counter to two.

The subsequent invocation of \forth{INVISIBLE} decrements the counter
to one, so we're still invisible, and will remain invisible until the
table of contents has been run.

(Note that we must substitute \forthb{0=} for \forthb{NOT}. The '83
Standard has changed \forthb{NOT} to mean one's complement, so that
\forthb{1 NOT} yields true. By the way, I think this was a mistake.)

This use of a counter may be dangerous, however. It requires parity of
command usage: two \forth{VISIBLE}s yields invisible. That is, unless
\forth{VISIBLE} clips the counter:

\begin{Code}
: VISIBLE   'INVISIBLE? @  1-  O MAX  'INVISIBLE? ! ;
\end{Code}
\index{C!Components:!sharing|)}%
\index{S!Sharing components|)}

