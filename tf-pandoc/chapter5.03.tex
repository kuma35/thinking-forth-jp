\section{Comment Conventions}%
\index{C!Comment conventions|(}%
\index{I!Implementation!comment conventions|(}

Appropriate commenting is essential.  There are five types of comments:
stack-effect comments, data-structure comments, input-stream comments,
purpose comments and narrative comments.
%%!!!te Original didn't put this into a description
\begin{description}
\item[\emph{A} stack-effect comment]\index{S!Stack-effect comment}%
shows the arguments that the definition consumes from the stack, and
the arguments it returns to the stack, if any.
%Page 150 in first edition.

\item[\emph{A} data-structure comment]\index{D!Data-structure comment}%
indicates the position and meaning of elements in a data structure.
For instance, a text buffer might contain a count in the first byte,
and 63 free bytes for text.

\item[\emph{An} input-stream comment]\index{I!Input-stream comment}%
indicates what strings the word expects to see in the input stream.
For example, the \Forth{} word \forth{FORGET} scans for the name of a
dictionary entry in the input stream.

\item[\emph{A} purpose comment]\index{P!Purpose comment}%
describes, in as few words possible, what the definition does.  How
the definition works is not the concern of the purpose comment.

\item[\emph{A} narrative comment]\index{N!Narrative comments}%
appears amidst a definition to explain what is going on, usually
line-by-line.  Narrative comments are used only in the ``vertical
format,'' which we'll describe in a later section.
\end{description}

Comments are usually typed in lower-case letters to distinguish them
from source code.  (Most \Forth{} words are spelled with upper-case
letters, but lower-case spellings are sometimes used in special cases.)

In the following sections we'll summarize the standardized formats
for these types of comments and give examples for each type.

\subsection{Stack Notation}%
\index{S!Stack notation|(}
\begin{tip}
Every colon or code definition that consumes and/or returns any arguments
on the stack must include a stack-effect comment.
\end{tip}
``Stack notation'' refers to conventions for representing what's on
the stack.  Forms of stack notation include ``stack pictures,''
``stack effects,'' and ``stack-effect comments.''%
\index{S!Stack notation|)}%
\index{P!Purpose comment}

\subsection{Stack Picture}%
\index{S!Stack picture|(}

A stack picture depicts items understood to be on the stack at a given
time.  Items are listed from left to right, with the leftmost item
representing the bottom of the stack and the rightmost item
representing the top.

For instance, the stack picture
\begin{Code}
nl n2
\end{Code}
indicates two numbers on the stack, with n2 on the top (the most
accessible position).

This is the same order that you would use to type these values in;
i.e., if n1 is 100 and n2 is 5000, then you would type
\begin{Code}
100 5000
\end{Code}
to place these values correctly on the stack.

%Page 151 in first edition.
A stack picture can include either abbreviations, such as ``n1,'' or
fully spelled-out words.  Usually abbreviations are used.  Some
standard abbreviations appear in Table \ref{tab-5-2}.  Whether
abbreviations or fully spelled-out words are used, each stack item
should be separated by a space.

If a stack item is described with a phrase (such as
``address-of-latest-link''), the words in the phrase should be joined
by hyphens.  For example, the stack picture:
\begin{Code}
address current-count max-count
\end{Code}
shows three elements on the stack.%
\index{S!Stack picture|)}

\subsection{Stack Effect}%
\index{S!Stack effect|(}

A ``stack effect'' shows two stack pictures: one picture of any items
that may be \emph{consumed} by a definition, and another picture of
any items \emph{returned} by the definition.  The ``before'' picture
comes first, followed by two hyphens, then the ``after'' picture.

For instance, the stack effect for \Forth{}'s addition operator,
\forth{+} is
\begin{Code}
n n -- sum
\end{Code}
where \forth{+} consumes two numbers and returns their sum.

Remember that the stack effect describes only the \emph{net result} of
the operation on the stack.  Other values that happen to reside on the
stack beneath the arguments of interest don't need to be shown.  Nor
do values that may appear or disappear while the operation is
executing.

If the word returns any input arguments unchanged, they
should be repeated in the output picture; e.g.,
\begin{Code}
3rd 2nd top-input -- 3rd 2nd top-output
\end{Code}
Conversely, if the word changes any arguments, the stack comment must
use a different descriptor:
\begin{Code}
nl -- n2
n -- n'
\end{Code}
A stack effect might appear in a formatted glossary.%
\index{S!Stack effect|)}

\subsection{Stack Effect Comment}%
\index{S!Stack-effect comment|(}

A ``stack-effect comment'' is a stack effect that appears in source
code surrounded by parentheses.  Here's the stack-effect comment for
the word \forth{COUNT}:
\begin{Code}
( address-of-counted-string -- address-of-text count)
\end{Code}
%Page 152 in first edition.
or:
\begin{Code}
( 'counted-string -- 'text count)
\end{Code}
(The ``count'' is on top of the stack after the word has executed.)

If a definition has no effect on the stack (that is, no effect the
user is aware of, despite what gyrations occur within the definition),
it needs no stack-effect comment:
\begin{Code}
: BAKE   COOKIES OVEN ! ;
\end{Code}
On the other hand, you may want to use an empty stack comment---i.e.,
\begin{Code}
: BAKE   ( -- )  COOKIES OVEN ! ;
\end{Code}
to emphasize that the word has no effect on the stack.

If a definition consumes arguments but returns none, the double-hyphen
is optional.  For instance,
\begin{Code}
( address count -- )
\end{Code}
can be shortened to
\begin{Code}
( address count)
\end{Code}
The assumption behind this convention is this: There are many more
colon definitions that consume arguments and return nothing than
definitions that consume nothing and return arguments.%
\index{S!Stack-effect comment|)}

\subsection{Stack Abbreviation Standards}%
\index{S!Stack abbreviation standards|(}

Abbreviations used in stack notation should be consistent. Table
\ref{tab-5-2} lists most of the commonly used abbreviations.  (This
table reappears in \App{E}.) The terms ``single-length,''
``double-length,'' etc. refer to the size of a ``cell'' in the
particular \Forth{} system.  (If the system uses a 16-bit cell, ``n''
represents a 16-bit number; if the system uses a 32-bit cell, ``n''
represents a 32-bit number.)%
\index{S!Stack abbreviation standards|)}

\subsection{Notation of Flags}

Table \ref{tab-5-2} shows three ways to represent a boolean flag.  To
illustrate, here are three versions of the same stack comment for the
word \forth{-TEXT}:
\begin{Code}
( at u a2 -- ?)
( at u a2 -- t=no-match)
( at u a2 -- f=match)
\end{Code}
%Page 153 in first edition.
\begin{table}[hhhh]
\caption{Stack-comment abbreviations.}
\label{tab-5-2}
\vspace{1ex}
\blackline{1ex}
\begin{tabular}{@{\hspace{2.5em}}ll}
n             &  single-length signed number \\
d             &  double-length signed number \\
u             &  single-length unsigned number \\
ud            &  double-length unsigned number \\
t             &  triple-length \\
q             &  quadruple-length \\
c             &  7-bit character value \\
b             &  8-bit byte \\
?             &  boolean flag; or; \\
\quad t=         &  true \\
\quad f=         &  false \\
a or adr      &  address \\
acf           &  address of code field \\
apf           &  address of parameter field \\
'             &  (as prefix) address of \\
s d           &  (as a pair) source destination \\
lo hi         &  lower-limit upper-limit (inclusive) \\
\#            &   count \\
o             &  offset \\
i             &  index \\
m             &  mask \\
x             &  don't care (data structure notation) \\
\end{tabular}
\smallskip

An ``offset'' is a difference expressed in absolute units, such as bytes.\\
An ``index'' is a difference expressed in logical units, such as
elements or records.
\vspace{0ex}
\blackline{0ex}
\end{table}
The equal sign after the symbols ``t'' and ``f'' equates the flag
outcome with its meaning.  The result-side of the second version would
be read ``true means no match.''

\subsection{Notation of Variable Possibilities}%
\index{V!Variable possibilities, notation of|(}
Some definitions yield a different stack effect under different
circumstances.

If the number of items on the stack remains the same under all
conditions, but the items themselves change, you can use the vertical
bar ( \forth{|} ) to mean ``or.'' The following stack-effect comment
describes a word that returns either the address of a file or, if the
requested file is not found, zero:
\begin{Code}
( -- address|O=undefined-file)
\end{Code}
If the number of items in a stack picture can vary---in either the
``before'' or ``after'' picture---you must write out both versions of
the entire stack
%Page 154 in first edition.
picture, along with the double-hyphen, separated by the ``or'' symbol.
For instance:
\begin{Code}
-FIND   ( -- apf len t=found | -- f=not-found )
\end{Code}
This comment indicates that if the word is found, three arguments are
returned (with the flag on top); otherwise only a false flag is
returned.

Note the importance of the second ``-\,-''.  Its omission would indicate
that the definition always returned three arguments, the top one being
a flag.

\index{S!Stack-effect comment|(}%
If you prefer, you can write the entire stack effect twice, either on
the same line, separated by three spaces:
\begin{Code}
?DUP   \ if zero: ( n -- n)    if non-zero:( n -- n n)
\end{Code}
or listed vertically:
\begin{Code}
-FIND  \     found:( -- apf len t )
       \ not-found:( -- f )
\end{Code}
\index{S!Stack-effect comment|)}%
\index{V!Variable possibilities, notation of|)}

\subsection{Data-Structure Comments}%
\index{D!Data-structure comment|(}

A ``data-structure comment'' depicts the elements in a data structure.
For example, here's the definition of an insert buffer called
\forth{|INSERT} :

\begin{Code}
CREATE |INSERT  64 ALLOT  \  { 1# | 63text }
\end{Code}
The ``faces'' (curly-brackets) begin and end the structure comment;
the bars separate the various elements in the structure; the numbers
represent bytes per element.  In the comment above, the first byte
contains the count, and the remaining 63 bytes contain the text.

A ``bit comment'' uses the same format as a data-structure comment to
depict the meaning of bits in a byte or cell.  For instance, the bit
comment
\begin{Code}
{ 1busy? | 1acknowledge? | 2x | 6input-device |
   6output-device }
\end{Code}
describes the format of a 16-bit status register of a communications
channel.  The first two bits are flags,\index{F!Flags}
the second two bits are unused, and the final pair of six-bit fields
indicate the input and output devices which this channel is connected
to.

If more than one data structure employs the same pattern of elements,
write out the comment only once (possibly in the preamble), and
%Page 155 in first edition.
give a name to the pattern for reference in subsequent screens.  For
instance, if the preamble gives the above bit-pattern the name
``status,'' then ``status'' can be used in stack comments to indicate
values with that pattern:
\begin{Code}
: STATUS?  ( -- status) ... ;
\end{Code}
If a \forthb{2VARIABLE} contains one double-length value, the comment
should be a stack picture that indicates the contents:
\begin{Code}
2VARIABLE PRICE  \ price in cents
\end{Code}
If a \forthb{2VARIABLE} contains two single-length data elements, it's
given a stack picture showing what would be on the stack after a
\forth{2@}. Thus:
\begin{Code}
2VARIABLE MEASUREMENTS  ( height weight )
\end{Code}
This is different from the comment that would be used if
\forth{MEASUREMENTS} were defined by \forthb{CREATE}.
\begin{Code}
CREATE MEASUREMENTS  4 ALLOT    \ { 2weight | 2height }
\end{Code}
(While both statements produce the same result in the dictionary, the
use of \forthb{2VARIABLE} implies that the values will normally be
``2-fetched'' and ``2-stored'' together-thus we use a \emph{stack}
comment.  The high-order part, appearing on top of the stack, is
listed to the right.  The use of \forthb{CREATE} implies that the
values will normally be fetched and stored separately--thus we use a
data structure comment.  The item in the 0th position is listed to the
left.)%
\index{D!Data-structure comment|)}

\subsection{Input-stream Comments}%
\index{I!Input-stream comment|(}

The input-stream comment indicates what words and/or strings are
presumed to be in the input stream. Table \ref{tab-5-3} lists the
designations used for input stream arguments.

\begin{table*}[hhhh]
\caption{Input-stream comment designations.}
\label{tab-5-3}
\vspace{1ex}\blackline{1ex}
\centerline{\begin{tabular}{ll}
c     &  single character, blank-delimited \\
name  &  sequence of characters, blank delimited \\
text  &  sequence of characters, delimited by non-blank \\
\end{tabular}}

\medskip
Follow ``text'' with the actual delimiter required; e.g.:
\forth{text"} or \forth{text)}
\vspace{1ex}\blackline{0ex}
\end{table*}
%% Note: AH : typewriter style footnote with *
%% Note: AH : text" and text) are literal examples,
%%    and should be slanted or such.
The input-stream comment appears \emph{before} the stack comment, and
is \emph{not} encapsulated between its own pair of parentheses, but
simply surrounded
%Page 156 in first edition.
by three spaces on each side.  For instance, here's one way to comment
the definition of \forth{'} (tick) showing first the input-stream comment,
then the stack comment:
\begin{Code}
: '   \ name   ( -- a)
\end{Code}
If you prefer to use \forth{(} , the comment would look like this:
\begin{Code}
: '   ( name   ( -- a)
\end{Code}
\index{S!String input, receiving|(}%
Incidentally, there are three distinct ways to receive string input.
To avoid confusion, here are the terms:

%%!!!te Original had bullet points here + description
\begin{description}
\item[Scanning-for]\index{S!Scanning-for}%
means looking ahead in the input stream, either for a word or number
as in the case of tick, or for a delimiter as in the case of
\forth{."} and \forth{(} .
\item[Expecting]\index{E!Expecting}%
means waiting for. \forth{EXPECT} and \forth{KEY}, and definitions
that invoke them, are ones that ``expect'' input.
\item[Presuming]\index{P!Presuming}%
indicates that in normal usage something will follow.  The word:
``scans-for'' the name to be defined, and ``presumes'' that a
definition will follow.
\end{description}
The input-stream comment is only appropriate for input being scanned-for.%
\index{S!String input, receiving|)}%
\index{I!Input-stream comment|)}

\subsection{Purpose Comments}%
\index{P!Purpose comment|(}

\begin{tip}
Every definition should bear a purpose comment unless:
\medskip
%%!!!te Original had a. b. enumeration
\begin{enumerate}
\item its purpose is clear from its name or its stack-effect comment, or
\item if it consists of three or fewer words.
\end{enumerate}
\end{tip}
The purpose comment should be kept to a minimum-never more than a full
line.  For example:
\begin{Code}
: COLD   \ restore system to start condition
    ... ;
\end{Code}
Use the imperative mood: ``set Foreground color,'' not ``sets
Foreground color.''

On the other hand, a word's purpose can often be described in terms of
its stack-effect comment.  You rarely need both a stack comment and a
purpose comment.  For instance:
\begin{Code}
: SPACES  ( #)   ... ;
\end{Code}
%Page 157 in first edition.
or
\begin{Code}
: SPACES  ( #spaces-to-type -- )   ... ;
\end{Code}
This definition takes as its incoming argument a number that
represents the number of spaces to type.
\begin{Code}
: ELEMENT  ( element# -- 'element)  2*  TABLE + ;
\end{Code}
This definition converts an index, which it consumes, into an address
within a table of 2-byte elements corresponding to the indexed element.
\begin{Code}
: PAD  ( -- 'scratch-pad)  HERE  80 + ;
\end{Code}
This definition returns an address of a scratch region of memory.

Occasionally, readability is best served by including both types of
comment. In this case, the purpose comment should appear last.  For
instance:
\begin{Code}
: BLOCK  ( n -- a)  \   ensure block n in buffer at a
\end{Code}

\begin{tip}
Indicate the type of comment by ordering: input-stream comments first,
stack-effect comments second, purpose comments last.
\end{tip}
For example:
\begin{Code}
: GET   \   name   ( -- a)   get first match
\end{Code}
If you prefer to use \forth{(}, then write:
\begin{Code}
: GET   (   name  ( -- a)    ( get first match)
\end{Code}
If necessary, you can put the purpose comment on a second line:
\begin{Code}
: WORD   \   name   ( c -- a)
   \ scan for string delimt'd by "c"; leave at a
   ...  ;
\end{Code}
\index{P!Purpose comment|)}%

\subsection{Comments for Defining Words}%
\index{D!Defining words:!comments for|(}
The definition of a defining word involves two behaviors:

\begin{itemize}
\item that of the defining word as it defines its ``child''
(compile-time behavior), and
\item that of the child itself (run-time behavior).
\end{itemize}
%Page 158 in first edition.
These two behaviors must be commented separately.

\index{C!Compiling words, comments for|(}%
\begin{tip}
Comment a defining word's compile-time behavior in the usual way;
comment its run-time behavior separately, following the word
\forthb{DOES>} (or \forthb{;CODE}).
\end{tip}
For instance,
\begin{Code}
: CONSTANT  ( n ) CREATE ,
   DOES>  ( -- n)  @ ;
\end{Code}
The stack-effect comment for the run-time (child's) behavior
represents the net stack effect for the child word.  Therefore it does
not include the address returned by \forthb{DOES>,} even though this
address is on the stack when the run-time code begins.

\bigskip\noindent
\emph{Bad} (run-time comment includes apf):
\begin{Code}
: ARRAY   \  name  ( #cells)
   CREATE 2* ALLOT
   DOES>   ( i apf -- 'cell)  SWAP  2* + ;
\end{Code}
\goodbreak\noindent
\emph{Good:}
\begin{Code}
: ARRAY   \  name  ( #cells)
   CREATE 2* ALLOT
    DOES>  ( i -- 'cell)  SWAP  2* + ;
\end{Code}
%% Note: AH: CELLS instead of 2* is indicated
Words defined by this word \forth{ARRAY} will exhibit the stack effect:
\begin{Code}
( i -- 'cell)
\end{Code}
If the defining word does not specify the run-time behavior, there still
exists a run-time behavior, and it may be commented:
\begin{Code}
: VARIABLE   (  name  ( -- )  CREATE  2 ALLOT ;
   \ does>   ( -- adr )
\end{Code}
\index{D!Defining words:!comments for|)}%

\subsection{Comments for Compiling Words}
As with defining words, most compiling words involve two behaviors:
\begin{enumerate}
\item That of the compiling word as the definition in which it appears
is compiled
\item That of the run-time routine which will execute when we invoke
the word being defined.  Again we must comment each behavior
separately.
\end{enumerate}
%Page 159 in first edition.

\begin{tip}
Comment a compiling word's run-time behavior in the usual way; comment
its compile-time behavior separately, beginning with the label
``Compile:''.
\end{tip}
For instance:
\begin{Code}
: IF   ( ? -- ) ...
\ Compile:   ( -- address-of-unresolved-branch)
   ... ; IMMEDIATE
\end{Code}
In the case of compiling words, the first comment describes the
run-time behavior, which is usually the \emph{syntax for using} the
word.  The second comment describes what the word \emph{actually does}
in compiling (which is of less importance to the user).

Other examples:
\begin{Code}
: ABORT"  ( ? -- )
\ Compile:   text"   ( -- )
\end{Code}
Occasionally a compiling word may exhibit a different behavior when it
is invoked \emph{outside} a colon definition.  Such words (to be
fastidious about it) require three comments.  For instance:
\begin{Code}
: ASCII  ( -- c)
\ Compile:   c   ( -- )
\ Interpret:   c   ( -- c )
     ... ; IMMEDIATE
\end{Code}
\App{E} includes two screens showing good commenting style.%
\index{C!Comment conventions|)}%
\index{C!Compiling words, comments for|)}%
\index{I!Implementation!comment conventions|)}

