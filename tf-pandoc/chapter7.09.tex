\section{Vectored Execution}%
\index{V!Vectored Execution|(}
\index{F!Functions:!vectored execution|(}

Vectored execution extends the ideas of currentness and indirection
beyond data, to functions. Just as we can save values and flags in
variables, we can also save functions, because functions can be referred to
by address.

The traditional techniques for implementing vectored execution are
described in \emph{Starting \Forth{}}, Chapter Nine. In this section we'll
discuss a new syntax which I invented and which I think can be used in many
circumstances more elegantly than the traditional methods.

\index{D!DOER/MAKE|(}%
The syntax is called \forth{DOER}/\forth{MAKE}. (If your system doesn't
include these words, refer to \App{B} for code and implementation
details.) It works like this: You define the word whose behavior will be
vectorable with the defining word \forthb{DOER}, as in

\begin{Code}
DOER PLATFORM
\end{Code}
Initially, the new word \forth{PLATFORM} does nothing. Then you can write
words that change what \forth{PLATFORM} does by using the word \forthb{MAKE}:

\begin{Code}
: LEFTWING   MAKE PLATFORM  ." proponent " ;
: RIGHTWING  MAKE PLATFORM  ." opponent " ;
\end{Code}
When you invoke \forth{LEFTWING}, the phrase \forth{MAKE PLATFORM} changes
what \forth{PLATFORM} will do. Now if you type \forth{PLATFORM}, you'll see:

\begin{Code}[commandchars=\&\{\}]
LEFTWING ok
PLATFORM &underline{proponent ok}
\end{Code}
\forth{RIGHTWING} will make \forth{PLATFORM} display ``opponent.'' You can
use \forth{PLATFORM} within another definition:

\begin{Code}
: SLOGAN   ." Our candidate is a longstanding " PLATFORM
   ." of heavy taxation for business. " ;
\end{Code}
The statement

\begin{Code}
LEFTWING SLOGAN
\end{Code}
will display one campaign statement, while

\begin{Code}
RIGHTWING SLOGAN
\end{Code}
will display another.

The ``\forth{MAKE}'' code can be any \Forth{} code, as much or as long as
you want; just remember to conclude it with semicolon. The semicolon at
the end of \forth{LEFTWING} serves for both \forth{LEFTWING} and for the
bit of code after \forth{MAKE}. When \forth{MAKE} redirects execution of
the \forthb{DOER} word, it also \emph{stops} execution of the word in
which it appears.

When you invoke \forth{LEFTWING}, for example, \forth{MAKE} redirects
\forth{PLATFORM} and exits. Invoking \forth{LEFTWING} does not cause
``proponent'' to be printed. \Fig{fig7-7} demonstrates this point, using a
conceptualized illustration of the dictionary.

\wtexfigt{fig7-7}{\forth{DOER} and \forth{MAKE}.}

If you want to \emph{continue} execution, you can use the word
\forthb{;AND} in place of semicolon. \forthb{;AND} terminates the code
that the \forthb{DOER} word points to, and resumes execution of the
definition in which it appears, as you can see in \Fig{fig7-8}.

\wtexfigt{fig7-8}{Multiple \forth{MAKE}s in parallel using \forth{;AND}.}

Finally, you can chain the ``making'' of \forthb{DOER} words in series by
not using \forthb{;AND}. \Fig{fig7-9} explains this better than I could
write about it.%
\index{V!Vectored Execution|)}%
\index{F!Functions:!vectored execution|)}

\wtexfigt{fig7-9}{Multiple \forth{MAKE}s in series.}

