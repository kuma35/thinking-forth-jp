\section{Looking Back, and \Forth{}}%
\index{F!forth@\Forth{}|(}
In this section we'll review the fundamental features of \Forth{} and
relate them to what we've seen about traditional methodologies.

Here's an example of \Forth{} code;
\begin{Code}
: BREAKFAST
   HURRIED?  IF  CEREAL  ELSE  EGGS  THEN CLEAN ;
\end{Code}
This is structurally identical to the procedure \forth{MAKE-BREAKFAST} on
page \pageref{fig-fig1-4}. (If you're new to \Forth{}, refer to
\App{A} for an explanation.)  The words \forth{HURRIED?},
\forth{CEREAL}, \forth{EGGS}, and \forth{CLEAN} are (most likely) also
defined as colon definitions.

Up to a point, \Forth{} exhibits all the traits we've studied: mnemonic
value, abstraction, power, structured control operators, strong
functional binding, limited coupling, and modularity. But regarding
modularity, we encounter what may be \Forth{}'s most significant
breakthrough:

\begin{tfquot}
The smallest atom of a \Forth{} program is not a module or a subroutine
or a procedure, but a ``word.''
\end{tfquot}
Furthermore, there are no subroutines, main programs, utilities, or
executives, each of which must be invoked differently.
\emph{Everything} in \Forth{} is a word.

Before we explore the significance of a word-based environment, let's
first study two \Forth{} inventions that make it possible.


\subsection{Implicit Calls}%
\index{C!Calls, implicit|(}%
\index{F!forth@\Forth{}!implicit calls|(}%
\index{I!Implicit calls|(}
First, calls are implicit. You don't have to say \forth{CALL CEREAL},
you simply say \forth{CEREAL}. In \Forth{}, the definition of
\forth{CEREAL} ``knows'' what kind of word it is and what procedure to
use to invoke itself.

Thus variables and constants, system functions, utilities, as well
as any user-defined commands or data structures can all be ``called''
simply by name.%
\index{C!Calls, implicit|)}%
\index{F!forth@\Forth{}!implicit calls|)}%
\index{I!Implicit calls|)}


\subsection{Implicit Data Passing}%
\index{F!forth@\Forth{}!implicit data passing|(}%
\index{I!Implicit data passing|(}
Second, data passing is implicit%
\index{D!Data passing, implicit}.
The mechanism that produces this effect is \Forth{}'s data stack%
\index{D!Data stacks}.
\Forth{} automatically pushes numbers onto the stack; words that require
numbers as input automatically pop them off the stack; words that
produce numbers as output automatically push them onto the stack. The
words \forth{PUSH} and \forth{POP} do not exist in high-level \Forth{}.

Thus we can write:
\begin{Code}
: DOIT
    GETC  TRANSFORM-TO-D  PUT-D ;
\end{Code}
confident that \forth{GETC} will get ``C,'' and leave it on the stack.
\forth{TRANSFORM-TO-D} will pick up ``C'' from the stack, transform
it, and leave ``D'' on the stack. Finally, \forth{PUT-D} will pick up
``D'' on the stack and write it. \Forth{} eliminates the act of passing
data from our code, leaving us to concentrate on the functional steps
of the data's transformation.

Because \Forth{} uses a stack for passing data, words can nest within
words. Any word can put numbers on the stack and take them off without
upsetting the f1ow of data between words at a higher level (provided,
of course, that the word doesn't consume or leave any unexpected
values).  Thus the stack supports structured, modular programming
while providing a simple mechanism for passing local arguments.

\Forth{} eliminates from our programs the details of \emph{how} words are
invoked and \emph{how} data are passed. What's left? Only the words
that describe our problem.

Having words, we can fully exploit the recommendations of \person{Parnas}---%
\index{P!Parnas, David}%
to decompose problems according to things that may change, and have
each ``module'' consist of many small functions, as many as are needed
to hide information about that module. In \Forth{} we can write as many
words as we need to do that, no matter how simple each of them may be.

A line from a typical \Forth{} application might read:
\begin{Code}
20 ROTATE LEFT TURRET
\end{Code}
Few other languages would encourage you to concoct a subroutine called
\forth{LEFT}, merely as a modifier, or a subroutine called
\forth{TURRET}, merely to name part of the hardware.

Since a \Forth{} word is easier to invoke than a subroutine (simply by
being named, not by being called), a \Forth{} program is likely to be
decomposed into more words than a conventional program would be into
subroutines.%
\index{I!Implicit data passing|)}%
\index{F!forth@\Forth{}!implicit data passing|)}


