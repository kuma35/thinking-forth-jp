%% Thinking Forth
%% Copyright (C) 2004 Leo Brodie
%% Initial transcription by <Josef Gabriel>
%% 
%% Chapter: <6>

\chapter{Factoring}\Chapmark{6}
\index{F!Factoring|(}

\initial In this chapter we'll continue our study of the implementations phase,
this time focusing on factoring.

Decomposition and factoring are chips off the same block. Both involve
dividing and organizing. Decomposition occurs during preliminary
design; factoring occurs during detailed design and implementation.

Since every colon definition reflects decisions of factoring, an
understanding of good factoring technique is perhaps the most
important skill for a \Forth{} programmer.

\index{F!Factoring!defined|(}
What is factoring? Factoring means organizing code into useful
fragments. To make a fragment useful, you often must separate reusable
parts from non-reusable parts. The reusable parts become new
definitions. The non-reusable parts become arguments or parameters to the
definitions.
\index{F!Factoring!defined|)}

Making this separation is usually referred to as ``factoring out.''
The first part of this chapter will discuss various ``factoring-out''
techniques.

Deciding how much should go into, or stay out of, a definition is
another aspect of factoring. The second section will outline the
criteria for useful factoring.

