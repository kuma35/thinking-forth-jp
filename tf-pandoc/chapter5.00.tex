%% Thinking Forth
%% Copyright (C) 2004 Leo Brodie
%% Initial transcription by Albert van der Horst
%%
%% Chapter: Implementation: Elements of FORTH Style
%%
%% ``%%!!!te'' marks mark typesetting errors in the original
%% We don't want to retain typesetting errors.

\chapter{Implementation: Elements~of \Forth{}~Style}\Chapmark{5}%
\index{I!Implementation|(}%

\initial Badly written \Forth{} has been accused of looking like ``code that
went through a trash compactor.'' It's true, \Forth{} affords more
freedom in the way we write applications.  But that freedom also gives
us a chance to write exquisitely readable and easily maintainable
code, provided we consciously employ the elements of good \Forth{}
style.

In this chapter we'll delve into \Forth{} coding convention
including:

\begin{itemize}%%originally: no bullet points
\item listing organization
\item screen layout, spacing and indentation
\item commenting
\item choosing names
\end{itemize}
I wish I could recommend a list of hard-and-fast conventions for
everyone.  Unfortunately, such a list may be inappropriate in many
situations.  This chapter merges many widely-adopted conventions with
personal preferences, commented with alternate ideas and the reasons
for the preferences.  In other words:
\begin{Code}
: TIP  VALUE JUDGEMENT ;
\end{Code}
I'd especially like to thank \person{Kim Harris},\index{H!Harris, Kim} who
proposed many of the conventions described in this chapter, for his
continuing efforts at unifying divergent views on good \Forth{} style.

