\section{Control Structures}
\Forth{} provides all the control structures%
\index{C!Control structures:!defined}
needed for structured, GOTO-less programming.

The syntax of the \forthb{IF THEN} construct is as follows:
\begin{Code}
... ( flag ) IF  KNOCK  THEN  OPEN ...
\end{Code}
The ``flag''\index{F!Flags} is a value on the stack, consumed by IF. A
non-zero value indicates true, zero indicates false.  A true flag causes
the code after \forthb{IF} (in this case, the word \forth{KNOCK}) to be
executed.  The word \forthb{THEN} marks the end of the conditional phrase;
execution resumes with the word \forth{OPEN}.  A false flag causes the
code between \forthb{IF} and \forthb{THEN} to {\em not} be executed.  In
either case, \forth{OPEN} will be performed.

The word \forthb{ELSE}\index{E!ELSE} allows an alternate phrase to be
executed in the false case. In the phrase:
\begin{Code}
( flag ) IF KNOCK  ELSE  RING  THEN  OPEN ...
\end{Code}
the word \forth{KNOCK} will be performed if the flag is true, otherwise
the word \forth{RING} will be performed.  Either way, execution will
continue starting with \forth{OPEN}.

\Forth{} also provides for indexed loops\index{L!Loops} in the form
\begin{Code}
( limit) ( index) DO ... LOOP
\end{Code}
and indefinite loops in the forms:
\begin{Code}
... BEGIN  ...  ( flag) UNTIL
\end{Code}
and
\begin{Code}
... BEGIN  ...  ( flag) WHILE ... REPEAT ;
\end{Code}
