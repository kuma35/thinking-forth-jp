\section{Screen Layout}%
\index{S!Screens!layout|(}%
\index{I!Implementation!screen layout|(}

In this section we'll discuss the layout of each source screen.

\index{C!Comment line|(}%
\begin{tip}
Reserve Line 0 as a ``comment line.''
\end{tip}
The comment line serves both as a heading for the screen, and also as
a line in the disk \forth{INDEX}.  It should describe the purpose of
the screen (not list the words defined therein).

The comment line minimally contains the name of the screen.  In larger
applications, you may also include both the chapter name and screen
name.  If the screen is one of a series of screens implementing a
lexicon, you should include a ``page number'' as well.

\index{S!Stamp|(}%
\index{D!Dates, representation of|(}
The upper right hand corner is reserved for the ``stamp.'' The stamp
includes the date of latest revision and, when authorship is
important, the programmer's initials (three characters to the left of
the date); e.g.:
\begin{Code}
( Chapter name        Screen Name -- pg #      JPJ 06/10/83)
\end{Code}
Some \Forth{} editors will enter the stamp for you at the press of a key.

A common form for representing dates is
\begin{Code}
mm-dd-yy
\end{Code}
that is, February 6, 1984 would be expressed
\begin{Code}
02-06-84
\end{Code}
An increasingly popular alternative uses
\begin{Code}
ddMmmyy
\end{Code}
where ``Mmm'' is a three-letter abbreviation of the month.  For instance:
\begin{Code}
22Oct84
\end{Code}
%Page 146 in first edition.
This form requires fewer characters than
\begin{Code}
10-22-84
\end{Code}
and eliminates possible confusion between dates and months.%
\index{S!Stamp|)}%
\index{D!Dates, representation of|)}

If your system has \forth{\bs} (``skip-line''---see \App{C}), you can
write the comment line like this:
\begin{Code}
\ Chapter name        Screen Name -- pg.#       JPJ 06/10/83
\end{Code}
As with all comments, use lower-case or a mixture of lower- and
upper-case text in the comment line.

One way to make the index of an application reveal more about the
organization of the screens is to indent the comment line by three
spaces in screens that continue a lexicon.  \Fig{fig5-5} shows a
portion of a list produced by \forthb{INDEX} in which the comment lines
for the continuing screens are indented.%
\index{C!Comment line|)}

\begin{figure*}[bbbb]
\caption{The output of \forth{INDEX} showing indented comment lines.}
\labelfig{fig5-5}
\begin{Code}
 90 \ Graphics           Chapter load               JPJ 06/10/83
 91    \ Dot-drawing primitives                     JPJ 06/10/83
 92 \ Line-drawing primitives                       JPJ 06/11/83
 93    \ Line-drawing primitives                    JPJ 06/10/83
 94    \ Line-drawing primitives                    JPJ 09/02/83
 95 \ Scaling, rotation                             JPJ 06/10/83
 96    \ Scaling, rotation                          JPJ 02/19/84
 97    \ Scaling, rotation                          JPJ 02/19/84
 98    \ Scaling, rotation                          JPJ 02/19/84
 99 \ Boxes                                         JPJ 06/10/83
100 \ Circles                                       JPJ 06/10/83
101    \ Circles                                    JPJ 06/10/83
102    \ Circles                                    JPJ 06/10/83
\end{Code}
\end{figure*}

\begin{tip}
Begin all definitions at the left edge of the screen, and define only
one word per line.
\end{tip}
%! There should be no page breaks between the text and code
\noindent \emph{Bad:}
\begin{Code}
: ARRIVING   ." HELLO" ;   : DEPARTING   ." GOODBYE" ;
\end{Code}
\noindent \emph{Good:}
\begin{Code}
: ARRIVING   ." HELLO" ;
: DEPARTING   ." GOODBYE" ;
\end{Code}
This rule makes it easier to find a definition in the listing.  (When
definitions continue for more than one line, the subsequent lines
should always be indented.)
%Page 147 in first edition.

\forthb{VARIABLE}s and \forthb{CONSTANT}s should also be defined one per
line.  (See ``Samples of Good Commenting Style'' in \App{E}.) This
leaves room for an explanatory comment on the same line.  The
exception is a large ``family'' of words (defined by a common
defining-word) which do not need unique comments:
\begin{Code}
0 HUE BLACK     1 HUE BLUE      2 HUE GREEN
3 HUE CYAN      4 HUE RED       5 HUE MAGENTA
\end{Code}
\begin{tip}
Leave lots of room at the bottom of the screen for later additions.
\end{tip}
On your first pass, fill each screen no more than half with code.  The
iterative approach demands that you sketch out the components of your
application first, then iteratively flesh them out until all the
requirements are satisfied.  Usually this means adding new commands,
or adding special-case handling, to existing screens.  (Not
\emph{always,} though.  A new iteration may see a simplification of
the code. Or a new complexity may really belong in another component
and should be factored out, into another screen.)

Leaving plenty of room at the outset makes later additions more
pleasant.  One writer recommends that on the initial pass, the screen
should contain about 20--40 percent code and 80--60 percent
whitespace \cite{stevenson81}.

Don't skip a line between each definition.  You may, however, skip a
line between \emph{groups} of definitions.%
\index{D!DECIMAL|(}\index{B!BASE|(}
\begin{tip}
All screens must leave \forthb{BASE} set to \forthb{DECIMAL}.
\end{tip}
Even if you have three screens in a row in which the code is written
in \forthb{HEX} (three screens of assembler code, for instance), each
screen must set \forth{BASE} to \forthb{HEX} at the top, and restore
base to \forthb{DECIMAL} at the bottom.  This rule ensures that each
screen could be loaded separately, for purposes of testing, without
mucking up the state of affairs.  Also, in reading the listing you
know that values are in decimal unless the screen explicitly
says \forthb{HEX}.

Some shops take this rule even further.  Rather than brashly resetting
base to \forthb{DECIMAL} at the end, they reset base to \emph{whatever
it was at the beginning.} This extra bit of insurance can be
accomplished in this fashion:
\begin{Code}
BASE @       HEX    \ save original BASE on stack
0A2 CONSTANT BELLS
0A4 CONSTANT WHISTLES
... etc. ...
BASE !              \ restore it
\end{Code}
%Page 148 in first edition.
\noindent Sometimes an argument is passed on the stack from screen to
screen, such as the value returned by \forthb{BEGIN} or \forthb{IF} in a
multiscreen assembler definition, or the base address passed from one
defining word to another---see ``Compile-Time Factoring'' in \Chap{6}.
In these cases, it's best to save the value of \forth{BASE} on the
return stack like this:
\begin{Code}
BASE @ >R     HEX
... etc. ...
R> BASE !
\end{Code}
Some folks make it a policy to use this approach on any screen that
changes \forthb{BASE}, so they don't have to worry about it.%
\index{B!BASE|)}

\person{Moore}\index{M!Moore, Charles} prefers to define \forthb{LOAD}
to invoke \forthb{DECIMAL} after loading.  This approach simplifies the
screen's contents because you don't have to worry about resetting.%
\index{D!DECIMAL|)}

\subsection{Spacing and Indentation}%
\index{S!Spacing|(}

\begin{tip}
Spacing and indentation are essential for readability.
\end{tip}
The examples in this book use widely accepted conventions of spacing
and indenting style.  Whitespace, appropriately used, lends
readability.  There's no penalty for leaving space in source screens
except disk memory, which is cheap.

For those who like their conventions in black and white, Table
\ref{tab-5-1} is a list of guidelines.  (But remember, \Forth{}'s
interpreter couldn't care less about spacing or indentation.)

\begin{table}[bbbb]
\caption{Indentation and spacing guidelines}
\label{tab-5-1}
\medskip\blackline{0pt}\medskip
\begin{minipage}{\textwidth}
\begin{quote}
1 space between the colon and the name\\
2 spaces between the name and the comment\footnote{An often-seen
alternative calls for 1 space between the name and comment and 3
between the comment and the definition.  A more liberal technique uses
3 spaces before and after the comment.  Whatever you choose, be
consistent.}\\
2 spaces, or a carriage return, after the comment and
before the definition\footnotemark[1]\\
3 spaces between the name and definition if no comment is used\\
3 spaces indentation on each subsequent line (or multiples
of 3 for nested indentation)\\
1 space between words/numbers within a phrase\\
2 or 3 spaces between phrases\\
1 space between the last word and the semicolon\\
1 space between semicolon and \forthb{IMMEDIATE} (if invoked)
\end{quote}
No blank lines between definitions, except to separate distinct groups
of definitions

%% Note: AH: Again type writer style footnotes.
\end{minipage}
\medskip\blackline{0pt}
\end{table}

%Page 149 in first edition.

The last position of each line should be blank except for:
%%!!!te original used a) b) enumeration here.
\ifeightyfour
\begin{list}{\alph{enumi})}{\usecounter{enumi}}
\else
\begin{enumerate}
\fi
\item quoted strings that continue onto the next line, or
\item the end of a comment.
\ifeightyfour
\end{list}
\else
\end{enumerate}
\fi
A comment that begins with \forth{\bs} may continue right to the end
of the line.  Also, a comment that begins with \forth{(} may have its
delimiting right parenthesis in the last column.

\index{I!Indentation|(}%
Here are some common errors of spacing and indentation:

\goodbreak\bigskip\noindent
\emph{Bad} (name not separated from the body of the definition):
\begin{Code}
: PUSH HEAVE HO ;
\end{Code}
\emph{Good:}
\begin{Code}
: PUSH   HEAVE HO ;
\end{Code}
\emph{Bad} (subsequent lines not indented three spaces):
\begin{Code}
: RIDDANCE  ( thing-never-to-darken-again -- )
DARKEN  NEVER AGAIN ;
\end{Code}
\emph{Good:}
\begin{Code}
: RIDDANCE  ( thing-never-to-darken-again -- )
   DARKEN  NEVER AGAIN ;
\end{Code}
\emph{Bad} (lack of phrasing):
\begin{Code}
: GETTYSBURG   4 SCORE 7 YEARS + AGO ;
\end{Code}
\goodbreak\noindent
\emph{Good:}
\begin{Code}
: GETTYSBURG   4 SCORE   7 YEARS +   AGO ;
\end{Code}
Phrasing is a subjective art;
I've yet to see a useful set of formal rules.\\
Simply strive for readability.%
\index{S!Spacing|)}%
\index{S!Screens!layout|)}%
\index{I!Implementation!screen layout|)}%
\index{I!Indentation|)}

