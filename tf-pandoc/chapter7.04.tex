\section{Local and Global Variables/Initialization}%
\index{V!Variables:!local|(}%
\index{V!Variables:!global|(}%
\index{G!Global variables|(}%
\index{L!Local variables|(}%

As we saw earlier, a variable that is used exclusively within a single
definition (or single lexicon), hidden from other code, is called a local
variable. A variable used by more than one lexicon is called a global
variable. As we've seen in an earlier chapter, a set of global variables that
collectively describe a common interface between several lexicons is
called an ``interface lexicon.''
\index{I!Interface lexicon}

\Forth{} makes no distinction between local and global variables.
But \Forth{} programmers do.

\begin{interview}%
\index{M!Moore, Charles|(}
\person{Moore}:

\begin{tfquot}
We should be writing for the reader. If something is referred to only
locally, a temporary variable just for accumulating a sum in, we should
define it locally. It's handier to define it in the block where it's used,
where you can see its comment.

If it's used globally, we should collect things according to their logical
function, and define them together on a separate screen. One per line with
a comment.

The question is, where do you initialize them? Some say on the same line,
immediately following its definition. But that messes up the comments, and
there isn't room for any decent comment. And it scatters the
initialization all over the application.

I tend to do all my initialization in the load screen. After I've loaded
all my blocks, I initialize the things that have to be initialized. It
might also set up color lookup tables or execute some initialization code.

If your program is destined to be target compiled, then it's easy to write
a word at the point that encompasses all the initialization.

It can get much more elaborate. I've defined variables in ROM where the
variables were all off in an array in high memory, and the initial values are
in ROM, and I copy up the initial values at initialization time. But usually
you're only initializing a few variables to anything other than zero.
\end{tfquot}%
\index{M!Moore, Charles|)}
\end{interview}%
\index{V!Variables:!local|)}%
\index{V!Variables:!global|)}%
\index{G!Global variables|)}%
\index{L!Local variables|)}%

