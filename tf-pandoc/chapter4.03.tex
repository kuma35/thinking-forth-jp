\section{Detailed Design}%
\index{L!Lexicons|(}

We're now at the point in the development cycle at which we've decided
we need a component (or a particular word). The component will consist
of a number of words, some of which (those that comprise the lexicon) will
be used by other components and some of which (the internal words) will
be only used within this component.

Create as many words as necessary to obey the following tip:

\begin{tip}
Each definition should perform a simple, well-defined task.
\end{tip}%
\index{D!Detailed design!steps in|(}
Here are the steps generally involved in designing a component:
%!! there's certainly some better way to do ordered lists in TeX
\begin{enumerate}
\item Based on the required functions, decide on the names and syntax for the
   external definitions (define the interfaces).
\item Refine the conceptual model by describing the algorithm(s) and data
   structure(s).
\item Recognize auxiliary definitions.
\item Determine what auxiliary definitions and techniques are already available.
\item Describe the algorithm with pseudocode.
\item Implement it by working backwards from existing definitions to the inputs.
\item Implement any missing auxiliary definitions.
\item If the lexicon contains many names with strong elements in common,
   design and code the commonalities as internal definitions, then implement
   the external definitions.
\end{enumerate}%
\index{D!Detailed design!steps in|)}
We'll discuss the first two steps in depth. Then we'll engage in an
extended example of designing a lexicon.

