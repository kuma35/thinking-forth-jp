\section{Summary}
\Forth{} has often been characterized as offbeat, totally unlike any
other popular language in structure or in philosophy. On the contrary,
\Forth{} incorporates many principles now boasted by the most
contemporary languages. Structured design, modularity, and
information-hiding are among the buzzwords of the day.

Some newer languages approach even closer to the spirit of \Forth{}.  The
language C, for instance, lets the programmer define new functions
either in C or in assembly language, as does \Forth{}. And as with \Forth{},
most of C is defined in terms of functions.

But \Forth{} extends the concepts of modularity and information-hiding
further than any other contemporary language. \Forth{} even hides the
manner in which words are invoked and the way local arguments are
passed.

The resulting code becomes a concentrated interplay of words, the
purest expression of abstract thought. As a result, \Forth{} programmers
tend to be more productive and to write tighter, more efficient, and
better maintainable code.

\Forth{} may not be the ultimate language. But I believe the ultimate
language, if such a thing is possible, will more closely resemble
\Forth{} than any other contemporary language.%
\index{F!forth@\Forth{}|)}


