\section{Summary}
\Forth{} has often been characterized as offbeat, totally unlike any
other popular language in structure or in philosophy. On the contrary,
\Forth{} incorporates many principles now boasted by the most
contemporary languages. Structured design, modularity, and
information-hiding are among the buzzwords of the day.

Some newer languages approach even closer to the spirit of \Forth{}.  The
language C, for instance, lets the programmer define new functions
either in C or in assembly language, as does \Forth{}. And as with \Forth{},
most of C is defined in terms of functions.

But \Forth{} extends the concepts of modularity and information-hiding
further than any other contemporary language. \Forth{} even hides the
manner in which words are invoked and the way local arguments are
passed.

The resulting code becomes a concentrated interplay of words, the
purest expression of abstract thought. As a result, \Forth{} programmers
tend to be more productive and to write tighter, more efficient, and
better maintainable code.

\Forth{} may not be the ultimate language. But I believe the ultimate
language, if such a thing is possible, will more closely resemble
\Forth{} than any other contemporary language.%
\index{F!forth@\Forth{}|)}


\begin{references}{99}
\bibitem{dahl72} \person{O.\@ J.\@ Dahl}, \person{E.\@ W.\@ Dijkstra}, and \person{C.\@ A.\@ R.\@ Hoare},
\emph{Structured Programming,} London, Academic Press, 1972.
\bibitem{wirth71} \person{Niklaus Wirth}, ``Program Development by Stepwise
Refinement,'' \emph{Communications of ACM,} 14, No. 4 (1971), 221-27.
\bibitem{stevens74-1} \person{W.\@ P.\@ Stevens}, \person{G.\@ J.\@ Myers}, and \person{L.\@ L.\@ Constantine},
``Structured Design,'' \emph{IBM Systems Journal,} Vol. 13, No. 2, 1974.
\bibitem{parnas72} \person{David L.\@ Parnas}, ``On the Criteria To Be Used in
Decomposing Systems into Modules,'' \emph{Communications of the ACM,}
December 1972.
\bibitem{liskov75} \person{Barbara H.\@ Liskov} and \person{Stephen N.\@ Zilles},
``Specification Techniques for Data Abstractions,'' \emph{IEEE
Transactions on Software Engineering,} March 1975.
\bibitem{parnas79} \person{David L.\@ Parnas}, ``Designing Software for Ease of
Extension and Contraction,'' \emph{IEEE Transactions on Software
Engineering,} March 1979.
\bibitem{shorre71} \person{Dewey Val Shorre}, ``Adding Modules to \Forth{},''
1980 FORML Proceedings, p. 71.
\bibitem{bern83} \person{Mark Bernstein}, ``Programming in the Laboratory,''
  unpublished paper, 1983.
\bibitem{bell72} \person{James R.\@ Bell}, ``Threaded Code,'' \emph{Communications
of ACM,} Vol. 16, No. 6, 370-72.
\bibitem{dewar} \person{Robert B.\@ K.\@ DeWar}, ``Indirect Threaded Code,''
\emph{Communications of ACM,} Vol. 18, No. 6, 331.
\bibitem{kogge82} \person{Peter M.\@ Kogge}, ``An Architectural Trail to
Threaded-Code Systems,'' \emph{Computer,} March, 1982.
\bibitem{dumse} \person{Randy Dumse}, ``The R65F11 \Forth{} Chip,'' \emph{\Forth{}
Dimensions,} Vol. 5, No. 2, p. 25.
\end{references}

