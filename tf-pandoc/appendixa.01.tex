\section{The Dictionary}%
\index{D!Dictionary:!defined|(}%
\index{F!forth@\Forth{}!overview of|(}%

\expandafter\initialb\Forth{} is expressed in words (and numbers) and is separated by spaces:
\begin{Code}
HAND OPEN  ARM LOWER  HAND CLOSE  ARM RAISE 
\end{Code}
Such commands may be typed directly from the keyboard, or edited onto 
mass storage then ``\forthb{LOAD}''ed.

\index{D!Defining words:!procedure|(}
All words, whether included with the system or user-defined, exist in the
``dictionary,'' a linked list.  A ``defining word'' is used to add new
names to the dictionary.  One defining word is \forthb{:} (pronounced
``colon''), which is used to define a new word in terms of previously
defined words.  Here is how one might define a new word called \forth{LIFT}:
\begin{Code}
: LIFT   HAND OPEN  ARM LOWER  HAND CLOSE  ARM RAISE ;
\end{Code}
The \forthb{;} terminates the definition.  The new word \forth{LIFT} may
now be used instead of the long sequence of words that comprise its
definition.

\Forth{} words can be nested like this indefinitely.  Writing a 
\Forth{} application consists of building increasingly powerful definitions,
such as this one, in terms of previously defined ones.

Another defining word is \forthb{CODE},\index{C!CODE}
which is used in place of colon to define a command in terms of machine
instructions for the native processor.  Words defined with \forthb{CODE}
are indistinguishable to the user from words defined with colon.
\forthb{CODE} definitions are needed only for the most time-critical
portions of an applications, if at all.
\index{D!Defining words:!procedure|)}

