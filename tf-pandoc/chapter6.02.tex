\section{Factoring Criteria}
\index{F!Factoring!criteria|(}
Armed with an understanding of factoring techniques, let's now discuss
several of the criteria for factoring \Forth{} definitions. They
include:

\begin{enumerate}
\item Limiting the size of definitions
\item Limiting repetition of code
\item Nameability
\item Information hiding
\item Simplifying the command interface
\end{enumerate}

\index{D!Defining words:!length|(}
\begin{tip}
Keep definitions short.
\end{tip}

\begin{interview}
We asked \person{Moore},\index{M!Moore, Charles|(}
``How long should a \Forth{} definition be?''

\begin{tfquot}
A word should be a line long. That's the target.

When you have a whole lot of words that are all useful in their own
right---perhaps in debugging or exploring, but inevitably there's a
reason for their existence---you feel you've extracted the essence of
the problem and that those words have expressed it.

Short words give you a good feeling.
\end{tfquot}\index{M!Moore, Charles|)}
\end{interview}
An informal examination of one of \person{Moore}'s applications shows
that he averages seven references, including both words and numbers,
per definition. These are remarkably short definitions. (Actually, his
code was divided about 50--50 between one-line and two-line
definitions.)

Psychological tests have shown that the human mind can only focus its
conscious attention on seven things, plus or minus two, at a time
\cite{miller56}. Yet all the while, day and night, the vast resources
of the mind are subconsciously storing immense amounts of data, making
connections and associations and solving problems.

Even if out subconscious mind knows each part of an application inside
out, our narrow-viewed conscious mind can only correlate seven
elements of it at once. Beyond that, our grasp wavers. Short
definitions match our mental capabilities.

Something that tempts many \Forth{} programmers to write overly long
definitions is the knowledge that headers take space in the
dictionary. The coarser the factoring, the fewer the names, and the
less memory that will be wasted.

It's true that more memory will be used, but it's hard to say that
anything that helps you test, debug and interact with your code is a
``waste.'' If your application is large, try using a default width of
three, with the ability to switch to a full-length name to avoid a
specific collision. (``Width'' refers to a limit on the number of
characters stored in the name field of each dictionary header.)

If the application is still too big, switch to a \Forth{} with
multiple dictionaries on a machine with extended memory, or better
yet, a 32-bit \Forth{} on a machine with 32-bit addressing.

A related fear is that over-factoring will decrease performance due to
the overhead of \Forth{}'s inner interpreter. Again, it's true that
there is some penalty for each level of nesting. But ordinarily the
penalty for extra nesting due to proper factoring will not be
noticeable. If you timings are that tight, the real solution is to
translate something into assembler.

\begin{tip}
Factor at the point where you feel unsure about your code (where
complexity approaches the conscious limit).
\end{tip}
Don't let your ego take over with an ``I can lick this!'' attitude.
\Forth{} code should never feel uncomfortably complex. Factor!

\begin{interview}
\person{Moore}:\index{M!Moore, Charles|(}

\begin{tfquot}
Feeling like you might have introduced a bug is one reason for factoring.
Any time you see a doubly-nested \forthb{DO }\forthb{LOOP}, that's a sign
that something's wrong because it will be hard to debug. Almost always
take the inner \forthb{DO }\forthb{LOOP} and make a word.

And having factored out a word for testing, there's no reason for
putting it back. You found it useful in the first place. There's no
guarantee you won't need it again.
\end{tfquot}\index{M!Moore, Charles|)}
\end{interview}
Here's another facet of the same principle:

\begin{tip}
Factor at the point where a comment seems necessary
\end{tip}
Particularly if you feel a need to remind yourself what's on the
stack, this may be a good time to ``make a break.''

\goodbreak
Suppose you have
\begin{Code}
... BALANCE  DUP xxx xxx xxx xxx xxx xxx xxx xxx xxx
     xxx xxx xxx xxx xxx xxx   ( balance) SHOW  ...
\end{Code}
which begins by computing the balance and ends by displaying it. In
the meantime, several lines of code use the balance for purposes of
their own. Since it's difficult to see that the balance is still on
the stack when \forth{SHOW} executes, the programmer has interjected a
stack picture.

This solution is generally a sign of bad factoring. Better to write:
\begin{Code}
: REVISE  ( balance -- )  xxx xxx xxx xxx xxx xxx xxx
     xxx xxx xxx xxx xxx xxx xxx ;
... BALANCE  DUP REVISE  SHOW  ...
\end{Code}
No narrative stack pictures are needed. Furthermore, the programmer
now has a reusable, testable subset of the definition.

\begin{tip}
Limit repetition of code.
\end{tip}
\index{C!Code:!repetition of|(}%
The second reason for factoring, to eliminate repeated fragments of
code, is even more important than reducing the size of definitions.
\index{D!Defining words:!length|)}

\begin{interview}
\person{Moore}:\index{M!Moore, Charles|(}

\begin{tfquot}
When a word is just a piece of something, it's useful for clarity or
debugging, but not nearly as good as a word that is used multiple
times. Any time a word is used only once you want to question its
value.

Many times when a program has gotten too big I will go back through it
looking for phrases that strike my eye as candidates for factoring.
The computer can't do this; there are too many variables.
\end{tfquot}\index{M!Moore, Charles|)}
\end{interview}
In looking over your work, you often find identical phrases or short
passages duplicated several times. In writing an editor I found this
phrase repeated several times:

\begin{Code}
FRAME  CURSOR @ +
\end{Code}
Because it appeared several times I factored it into a new word called
\forth{AT}.

It's up to you to recognize fragments that are coded differently but
functionally equivalent, such as:

\begin{Code}
FRAME  CURSOR @ 1-  +
\end{Code}
The \forth{1-} appears to make this phrase different from the one defined as
\forth{AT}. But in fact, it can be written

\begin{Code}
AT 1-
\end{Code}
On the other hand:

\begin{tip}
When factoring out duplicate code, make sure the factored code serves
a single purpose.
\end{tip}
Don't blindly seize upon duplications that may not be useful. For
instance, in several places in one application I used this phrase:

\begin{Code}
BLK @ BLOCK  >IN @ +  C@
\end{Code}
I turned it into a new word and called it \forth{LETTER}, since it
returned the letter being pointed to by the interpreter.

In a later revision, I unexpectedly had to write:

\begin{Code}
BLK @ BLOCK  >IN @ +  C!
\end{Code}
I could have used the existing \forth{LETTER} were it not for its
\forth{C@} at the end. Rather than duplicate the bulk of the phrase
in the new section, I chose to refactor \forth{LETTER} to a finer
resolution, taking out the \forth{C@}.  The usage was then either
\forth{LETTER C@} or \forth{LETTER C!}. This change required me to
search through the listing changing all instances of \forth{LETTER} to
\forth{LETTER C@}.  But I should have done that in the first place,
separating the computation of the letter's address from the operation
to be performed on the address.
\index{C!Code:!repetition of|)}

Similar to our injunction against repetition of code:

\index{P!Pattern repetition|(}
\begin{tip}
Look for repetition of patterns.
\end{tip}
If you find yourself referring back in the program to copy the pattern
of previously-used words, then you may have mixed in a general idea
with a specific application. The part of the pattern you are copying
perhaps can be factored out as an independent definition that can be
used in all the similar cases.
\index{P!Pattern repetition|)}

\index{N!Nameability|(}%
\begin{tip}
Be sure you can name what you factor.
\end{tip}

\medbreak
\begin{interview}
\person{Moore}:\index{M!Moore, Charles|(}
\begin{tfquot}
%%style: better use m-dashes here
If you have a concept that you can't assign a single name to, not a
hyphenated name, but a name, it's not a well-formed concept. The
ability to assign a name is a necessary part of decomposition.
Certainly you get more confidence in the idea.
\end{tfquot}\index{M!Moore, Charles|)}
\end{interview}
\index{N!Nameability|)}%
Compare this view with the criteria for decomposing a module espoused
by structured design in \Chap{1}. According to that method, a module
should exhibit ``functional binding,'' which can be verified by
describing its function in a single, non-compound, \emph{sentence}.
\Forth{}'s ``atom,'' a \emph{name}, is an order of magnitude more
refined.

\begin{tip}
Factor definitions to hide details that may change.
\end{tip}
\index{F!forth@\Forth{}!information-hiding|(}%
\index{I!Information-hiding|(}
We've seen the value of information hiding in earlier chapters,
especially with regard to preliminary design. It's useful to remember
this criterion during the implementation stage as well.

Here's a very short definition that does little except hide information:

\begin{Code}
: >BODY  ( acf -- apf )  2+ ;
\end{Code}
This definition allows you to convert an acf (address of code field) to an
apf (address of parameter field) without depending on the actual structure
of a dictionary definition. If you were to use \forthb{2+} instead of the
word \forthb{>BODY}, you would lose transportability if you ever converted
to a \Forth{} system in which the heads were separated from the bodies.
(This is one of a set of words suggested by \person{Kim Harris}, and
included as an Experimental Proposal in the
\Forth{}-83 Standard \cite{harris83}.)

Here's a group of definitions that might be used in writing an editor:

\begin{Code}
: FRAME  ( -- a)  SCR @ BLOCK ;
: CURSOR  ( -- a)  R# ;
: AT  ( -- a)  FRAME  CURSOR @ + ;
\end{Code}
These three definitions can form the basis for all calculations of
addresses necessary for moving text around. Use of these three
definitions completely separates your editing algorithms from a
reliance on \Forth{} blocks.

What good is that? If you should decide, during development, to create
an editing buffer to protect the user from making errors that destroy
a block, you merely have to redefine two of these words, perhaps like
this:

\begin{Code}
CREATE FRAME  1024 ALLOT
VARIABLE CURSOR
\end{Code}
The rest of your code can remain intact. 

\begin{tip}
Factor functions out of definitions that display results.
\end{tip}
This is really a question of decomposition.

Here's an example. The word defined below, pronounced
``people-to-paths,'' computes how many paths of communication there are
between a given number of people in a group. (This is a good thing for
managers of programmer teams to know---the number of communication
paths increases drastically with each new addition to the team.)
\begin{Code}
: PEOPLE>PATHS  ( #people -- #paths )  DUP 1-  *  2/ ;
\end{Code}
This definition does the calculation only. Here's the ``user definition''
that invokes \forth{PEOPLE>PATHS} to perform the calculation, and then
displays the result:
\begin{Code}
: PEOPLE  ( #people)
    ." = "  PEOPLE>PATHS  .  ." PATHS " ;
\end{Code}
This produces:
%% I don't know how to underline these.
\begin{Code}[commandchars=\&\{\}]
2 PEOPLE&underline{ = 1 PATHS}
3 PEOPLE&underline{ = 3 PATHS}
5 PEOPLE&underline{ = 10 PATHS}
10 PEOPLE&underline{ = 45 PATHS}
\end{Code}
Even if you think you're going to perform a particular calculation
only once, to display it in a certain way, believe me, you're wrong.
You will have to come back later and factor out the calculation part.
Perhaps you'll need to display the information in a right-justified
column, or perhaps you'll want to record the results in a data
base---you never know. But you'll always have to factor it, so you
might as well do it right the first time. (The few times you might get
away with it aren't worth the trouble.)

The word \forth{.} (dot) is a prime example. Dot is great 99\% of the
time, but occasionally it does too much. Here's what it does, in fact
(in \Forth{}--83):

\begin{Code}
: .   ( n )  DUP ABS 0 <# #S  ROT SIGN  #> TYPE SPACE ;
\end{Code}
But suppose you want to convert a number on the stack into an ASCII
string and store it in a buffer for typing later. Dot converts it, but
also types it. Or suppose you want to format playing cards in the form
\forth{10C} (for ``ten of clubs''). You can't use dot to display the 10
because it prints a final space.

Here's a better factoring found in some \Forth{} systems:

\begin{Code}
: (.)  ( n -- a #)  DUP ABS 0  <# #S  ROT SIGN  #> ;
: .  ( n)  (.) TYPE SPACE ;
\end{Code}
We find another example of failing to factor the output function from
the calculation function in our own Roman numeral example in \Chap{4}.
Given our solution, we can't store a Roman numeral in a buffer or
even center it in a field. (A better approach would have been to use
\forthb{HOLD} instead of \forthb{EMIT}.)\program{roman3}

Information hiding can also be a reason \emph{not} to factor. For
instance, if you factor the phrase

\begin{Code}
SCR @ BLOCK
\end{Code}
into the definition

\begin{Code}
: FRAME   SCR @ BLOCK ;
\end{Code}
remember you are doing so only because you may want to change the location
of the editing frame. Don't blindly replace all occurrences of the phrase
with the new word \forth{FRAME,} because you may change the definition of
\forth{FRAME} and there will certainly be times when you really want
\forthb{SCR }\forthb{@ }\forthb{BLOCK}.

\begin{tip}
If a repeated code fragment is likely to change in some cases but not
others, factor out only those instances that might change. If the
fragment is likely to change in more than one way, factor it into more
than one definition.
\end{tip}
Knowing when to hide information requires intuition and experience.
Having made many design changes in your career, you'll learn the hard
way which things will be most likely to change in the future.

You can never predict everything, though. It would be useless to try,
as we'll see in the upcoming section called ``The Iterative Approach
in Implementation.''

\index{C!Commands, reducing number of|(}
\begin{tip}
Simplify the command interface by reducing the number of commands.
\end{tip}
\index{F!forth@\Forth{}!information-hiding|)}%
\index{I!Information-hiding|)}%
It may seem paradoxical, but good factoring can often yield
\emph{fewer} names. In \Chap{5} we saw how six simple names
(\forth{LEFT}, \forth{RIGHT}, \forth{MOTOR}, \forth{SOLENOID},
\forth{ON}, and \forth{OFF}) could do the work of eight
badly-factored, hyphenated names.

As another example, I found two definitions circulating in one
department in which \Forth{} had recently introduced. Their purpose
was purely instructional, to remind the programmer which vocabulary
was \forth{CURRENT}, and which was \forth{CONTEXT}:

\begin{Code}
: .CONTEXT   CONTEXT @  8 -  NFA  ID.   ;
: .CURRENT   CURRENT @  8 -  NFA  ID.  ;
\end{Code}
\goodbreak
\noindent If you typed

\begin{Code}
.CONTEXT
\end{Code}
the system would respond

\begin{Code}[commandchars=\&\{\}]
.CONTEXT&underline{ FORTH}
\end{Code}
(They worked---at least on the system used there---by backing up to the
name field of the vocabulary definition, and displaying it.)

The obvious repetition of code struck my eye as a sign of bad
factoring. It would have been possible to consolidate the repeated
passage into a third definition:

\begin{Code}
: .VOCABULARY   ( pointer )  @  8 -  NFA  ID. ;
\end{Code}
shortening the original definitions to:

\begin{Code}
: .CONTEXT   CONTEXT .VOCABULARY ;
: .CURRENT   CURRENT .VOCABULARY ;
\end{Code}
But in this approach, the only difference between the two definitions
was the pointer to be displayed. Since part of good factoring is to
make fewer, not more definitions, it seemed logical to have only one
definition, and let it take as an argument either the word
\forth{CONTEXT} or the word \forth{CURRENT}.

\medbreak
Applying the principles of good naming, I suggested:

\begin{Code}
: IS  ( adr)   @  8 -  NFA  ID. ;
\end{Code}
allowing the syntax

\begin{Code}[commandchars=\&\{\}]
CONTEXT IS&underline{ ASSEMBLER ok}
\end{Code}
or

\begin{Code}[commandchars=\&\{\}]
CURRENT IS&underline{ FORTH ok}
\end{Code}
The initial clue was repetition of code, but the final result came
from attempting to simplify the command interface.

Here's another example. The IBM PC has four modes four displaying text
only:

\begin{quote}\sf
40 column monochrome

40 column color

80 column monochrome

80 column color
\end{quote}
The word \forth{MODE} is available in the \Forth{} system I use.
\forth{MODE} takes an argument between 0 and 3 and changes the mode
accordingly. Of course, the phrase \forth{0 MODE} or \forth{1 MODE}
doesn't help me remember which mode is which.

Since I need to switch between these modes in doing presentations, I
need to have a convenient set of words to effect the change. These
words must also set a variable that contains the current number of
columns---40 or 80.

Here's the most straightforward way to fulfill the requirements:

\begin{Code}
: 40-B&W       40 #COLUMNS !  0 MODE ;
: 40-COLOR     40 #COLUMNS !  1 MODE ;
: 80-B&W       80 #COLUMNS !  2 MODE ;
: 80-COLOR     80 #COLUMNS !  3 MODE ;
\end{Code}
By factoring to eliminate the repetition, we come up with this version:

\begin{Code}
: COL-MODE!     ( #columns mode )  MODE  #COLUMNS ! ;
: 40-B&W       40 0 COL-MODE! ;
: 40-COLOR     40 1 COL-MODE! ;
: 80-B&W       80 2 COL-MODE! ;
: 80-COLOR     80 3 COL-MODE! ;
\end{Code}
But by attempting to reduce the number of commands, and also by
following the injunctions against numerically-prefixed and hyphenated
names, we realize that we can use the number of columns as a stack
argument, and \emph{calculate} the mode:

\begin{Code}
: B&W    ( #cols -- )  DUP #COLUMNS !  20 /  2-     MODE ;
: COLOR  ( #cols -- )  DUP #COLUMNS !  20 /  2-  1+ MODE ;
\end{Code}
This gives us this syntax:

\begin{Code}
40 B&W
80 B&W
40 COLOR
80 COLOR
\end{Code}
We've reduced the number of commands from four to two.

Once again, though, we have some duplicate code. If we factor out this
code we get:

\begin{Code}
: COL-MODE!  ( #columns chroma?)
   SWAP DUP #COLUMNS !  20 / 2-  +  MODE ;
: B&W    ( #columns -- )  0 COL-MODE! ;
: COLOR  ( #columns -- )  1 COL-MODE! ;
\end{Code}
Now we've achieved a nicer syntax, and at the same time greatly
reduced the size of the object code. With only two commands, as in
this example, the benefits may be marginal. But with larger sets of
commands the benefits increase geometrically.

Our final example is a set of words to represent colors on a
particular system. Names like \forth{BLUE} and \forth{RED} are nicer
than numbers. One solution might be to define:

\begin{Code}
 0 CONSTANT BLACK                 1 CONSTANT BLUE
 2 CONSTANT GREEN                 3 CONSTANT CYAN
 4 CONSTANT RED                   5 CONSTANT MAGENTA
 6 CONSTANT BROWN                 7 CONSTANT GRAY
 8 CONSTANT DARK-GRAY             9 CONSTANT LIGHT-BLUE
10 CONSTANT LIGHT-GREEN          11 CONSTANT LIGHT-CYAN
12 CONSTANT LIGHT-RED            13 CONSTANT LIGHT-MAGENTA
14 CONSTANT YELLOW               15 CONSTANT WHITE
\end{Code}
These colors can be used with words such as \forth{BACKGROUND},
\forth{FOREGROUND}, and \forth{BORDER}:

\begin{Code}
WHITE BACKGROUND  RED FOREGROUND  BLUE BORDER
\end{Code}
But this solution requires 16 names, and many of them are hyphenated.
Is there a way to simplify this?

We notice that the colors between 8 and 15 are all ``lighter''
versions of the colors between 0 and 7. (In the hardware, the only
difference between these two sets is the setting of the ``intensity
bit.'') If we factor out the ``lightness,'' we might come up with this
solution:\program{color3}

\begin{Code}
VARIABLE 'LIGHT?  ( intensity bit?)
: HUE  ( color)  CREATE ,
   DOES>  ( -- color )  @  'LIGHT? @  OR  0 'LIGHT? ! ;
 0 HUE BLACK         1 HUE BLUE           2 HUE GREEN
 3 HUE CYAN          4 HUE RED            5 HUE MAGENTA
 6 HUE BROWN         7 HUE GRAY
: LIGHT   8 'LIGHT? ! ;
\end{Code}
With this syntax, the word

\begin{Code}
BLUE
\end{Code}
by itself will return a ``1'' on the stack, but the phrase

\begin{Code}
LIGHT BLUE
\end{Code}
will return a ``9.'' (The adjective \forth{LIGHT} sets flag which is
used by the hues, then cleared.)

\goodbreak
If necessary for readability, we still might want to define:

\begin{Code}
8 HUE DARK-GRAY
14 HUE YELLOW
\end{Code}
Again, through this approach we've achieved a more pleasant syntax and
shorter object code.\program{color4}
\index{C!Commands, reducing number of|)}

\begin{tip}
Don't factor for the sake of factoring. Use clich\'es.
\index{C!Clich\'es|(}
\end{tip}
The phrase

\begin{Code}
OVER + SWAP
\end{Code}
may be seen commonly in certain applications. (It converts an address and
count into an ending address and starting address appropriate for a
\forthb{DO LOOP}.)\index{D!DO LOOP}

Another commonly seen phrase is

\begin{Code}
1+ SWAP
\end{Code}
(It rearranges a first-number and last-number into the
last-number-plus-one and first-number order required by \forthb{DO}.)

It's a little tempting to seize upon these phrases and turn them into
words, such as (for the first phrase) \forth{RANGE}.

\begin{interview}
\person{Moore}:\index{M!Moore, Charles|(}

\begin{tfquot}
That particular phrase [\forthb{OVER}\forthb{ +}\forthb{ SWAP}] is one
that's right on the margin of being a useful word. Often, though, if
you define something as a word, it turns out you use it only once. If
you name such a phrase, you have trouble knowing exactly what
\forth{RANGE} does. You can't see the manipulation in your
mind. \forthb{OVER }\forthb{+ }\forthb{SWAP} has greater mnemonic
value than \forth{RANGE}.
\end{tfquot}
\index{M!Moore, Charles|)}
\end{interview}
%% Could not get that special e in clich\'es.
I call these phrases ``clich\'es.'' They stick together as meaningful
functions. You don't have to remember how the phrase works, just what
it does. And you don't have to remember an extra name.
\index{C!Clich�s|)}
\index{F!Factoring!criteria|)}

