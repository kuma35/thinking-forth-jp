\section{The Stylish Return Stack}%
\index{D!Data stacks!return|(}
\index{R!Return stack|(}

What about this use of the return stack to hold temporary arguments? Is
it good style or what?

Some people take great offense to its use. But the return stack
offers the simplest solution to certain gnarly stack jams. Witness the
definition of \forthb{CMOVE>} in the previous section.

If you decide to use the return stack for this purpose, remember
that you are using a component of \Forth{} for a purpose other than that
intended. (See the section called ``Sharing Components,'' later in this
chapter.)

Here's some suggestions to keep you from shooting yourself in the
foot:

\begin{tip}
%!! there's certainly some better way to do ordered lists in TeX
\begin{enumerate}
\item Keep return stack operators symmetrical.
\item Keep return stack operators symmetrical under all control flow
conditions.
\item In factoring definitions, watch out that one part doesn't contain
one return stack operator, and the other its counterpart.
\item If used inside a \forthb{DO }\forthb{LOOP}, return stack operators
must be symmetrical within the loop, and \forthb{I} is no longer valid in
code bounded by \forthb{>R} and \forthb{R>}.
\end{enumerate}
\end{tip}
For every \forthb{>R} there must be a \forthb{R>} in the same definition.
Sometimes the operators will appear to be symmetrical, but due to the
control structure they aren't. For instance:

\begin{Code}
... BEGIN ... >R ... WHILE ... R> ... REPEAT
\end{Code}
If this construction is used in the outer loop of your application,
everything will run fine until you exit (perhaps hours later) when you'll
suddenly blow up. The problem? The last time through the loop, the
resolving \forthb{R>} has been skipped.%
\index{D!Data stacks!return|)}
\index{R!Return stack|)}

