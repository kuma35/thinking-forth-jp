\section{Decomposition by Sequential Complexity}%
\index{D!Decomposition!by sequential complexity|(}%
\index{P!Preliminary Design!decomposition by sequential complexity|(}%
\index{S!Sequential complexity, decomposition by|(}

We've been discussing one way to do decomposition: according to
components. The second way is according to sequential complexity.

One of \Forth{}'s rules is that a word must already have been defined to
be invoked or referred to. Usually the sequence in which words are
defined parallels the order of increasing capabilities which the words
must possess. This sequence leads to a natural organization of the
source listing. The powerful commands are simply added on top of the
elementary application (\fig{fig3-10}a).

Like a textbook, the elementary stuff comes first. A newcomer to
the project would be able to read the elementary parts of the code before
moving on the advanced stuff.

\wepsfigt{fig3-10}{Two ways to add advanced capabilities.}

But in many large applications, the extra capabilities are best
implemented as an enhancement to some private, root function in the
elementary part of the application (\fig{fig3-10}b). By being able to
change the root's capability, the user can change the capability of all the
commands that use the root.

Returning to the word processor for an example, a fairly primitive
routine is the one that starts a new page. It's used by the word that
starts a new line; when we run out of lines we must start a new
page. The word that starts a new line, in turn, is used by the routine
that formats words on the line; when the next word won't fit on the
current line, we invoke \forth{NEWLINE}. This ``uses'' hierarchy demands that
we define \forth{NEWPAGE} early in the application.

The problem? One of the advanced components includes a routine that
must be invoked by \forth{NEWPAGE}. Specifically, if a figure or table
appears in the middle of text, but at format time won't fit on what's
left of the page, the formatter defers the figure to the next page
while continuing with the text. This feature requires somehow
``getting inside of'' \forth{NEWPAGE}, so that when \forth{NEWPAGE} is
next executed, it will format the deferred figure at the top of the
new page:

\begin{Code}
: NEWPAGE  ... ( terminate page with footer)
   ( start new page with header)  ...  ?HOLDOVER ... ;
\end{Code}
How can \forth{NEWPAGE} invoke \forth{?HOLDOVER}, if \forth{?HOLDOVER} is
not defined until much later?

While it's theoretically possible to organize the listing so that the
advanced capability is defined before the root function, that approach is
bad news for two reasons.

First, the natural organization (by degree of capability) is
destroyed. Second, the advanced routines often use code that is defined
amid the elementary capabilities. If you move the advanced routines to
the front of the application, you'll also have to move any routines they
use, or duplicate the code. Very messy.

\index{V!Vectored execution|(}
You can organize the listing by degree of complexity using a technique
called ``vectoring.'' You can allow the root function to invoke (point
to) any of various routines that have been defined after the root
function itself. In our example, only the \emph{name} of the routine
\forth{?HOLDOVER} need be created early; its definition can be given later.%
\index{V!Vectored execution|)}

\Chap{7} treats the subject of vectoring in \Forth{}.%
\index{D!Decomposition!by sequential complexity|)}%
\index{P!Preliminary Design!decomposition by sequential complexity|)}%
\index{S!Sequential complexity, decomposition by|)}

