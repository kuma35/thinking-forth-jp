\section{Listing Organization}%
\index{L!Listing organization|(}%
\index{I!Implementation!listing organization|(}

A well-organized book has clearly defined chapters, with clearly defined
sections, and a table of contents to help you see the organization at a
glance.  A well-organized book is easy to read.  A badly organized book
makes comprehension more difficult, and makes finding information
later on nearly impossible.

\wepsfigp{fig5-1}{I still don't see how these programming conventions
enhance readability.}

The necessity for good organization applies to an application listing
as well.  Good organization has three aspects:

\begin{enumerate}\parsep=0pt\itemsep=0pt
\item Decomposition
\item Composition
\item Disk partitioning
\end{enumerate}

\subsection{Decomposition}%
\index{D!Decomposition|(}%
\index{L!Lexicons|(}
As we've already seen, the organization of a listing should follow the
decomposition of the application into lexicons.  Generally these
lexicons should be sequenced in ``uses'' order.  Lexicons being
\emph{used} should precede the lexicons which \emph{use} them.%
\index{L!Lexicons|)}

On a larger scale, elements in a listing should be organized by degree
of complexity, with the most complex variations appearing towards the
end.  It's best to arrange things so that you can leave off the
lattermost screens (i.e., not load them) and still have a
self-sufficient, running application, working properly except for the
lack of the more advanced features.

We discussed the art of decomposition extensively in \Chap{3}.%
\index{D!Decomposition|)}

\subsection{Composition}%
\index{C!Composition|(}%
\index{L!Load screens|(}%

Composition is the putting together of pieces to create a whole.  Good
composition requires as much artistry as good decomposition.

One of \Forth{}'s present conventions is that source code resides in
``screens,'' which are 1K units of mass storage.  (The term ``screen''
refers to a block used specifically for source code.) It's possible in
\Forth{} to chain every screen of code to the next, linking the entire
listing together linearly like a lengthy parchment scroll.  This is
not a useful approach.  Instead:

\begin{tip}
Structure your application listing like a book: hierarchically.
\end{tip}
An application may consist of:

\begin{description}\parsep=0pt\itemsep=0pt
\item[Screens:] the smallest unit of \Forth{} source
\item[Lexicons:]\index{L!Lexicons} one to three screens, enough to
implement a component
\item[Chapters:]\index{C!Chapters} a series of related lexicons, and
\item[Load screens:] analogous to a table of contents, a screen that
loads the chapters in the proper sequence.
\end{description}

%Page 138 in first edition.

\begin{figure*}[tttt]
\caption{Example of an application-load screen.}
\labelfig{fig5-1}
\setcounter{screen}{1}
\begin{Screen}
\ QTF+ Load Screen                                      07/09/83
: RELEASE#   ." 2.01" ;
  9 LOAD  \ compiler tools, language primitives
 12 LOAD  \ video primitives
 21 LOAD  \ editor
 39 LOAD  \ line display
 48 LOAD  \ formatter
 69 LOAD  \ boxes
 81 LOAD  \ deferring
 90 LOAD  \ framing
 96 LOAD  \ labels, figures, tables
102 LOAD  \ table of contents generator




\end{Screen}
\end{figure*}%
\index{C!Composition|)}

\subsection{Application-load Screen}%
\index{L!Load screens!application|(}

\Fig{fig5-1} is an example of an application-load screen.  Since it
resides in Screen 1, you can load this entire application by entering
\begin{Code}
1 LOAD
\end{Code}
The individual load commands within this screen load the chapters of
the application.  For instance, Screen 12 is the load screen for the
video primitives chapter.

As a reference tool, the application-load screen tells you where to
find all of the chapters.  For instance, if you want to look at the
routines that do framing, you can see that the section starts at
Screen 90.

Each chapter-load screen in turn, loads all of the screens comprising
the chapter.  We'll study some formats for chapter-load screens shortly.

The primary benefit of this hierarchical scheme is that you can load
any section, or any screen by itself, without having to load the
entire application.  Modularity of the source code is one of the
reasons for \Forth{}'s quick turnaround time for editing, loading, and
testing (necessary for the iterative approach).  Like pages of a book,
each screen can be accessed individually and quickly.  It's a ``random
access'' approach to source-code maintenance.

You can also replace any passage of code with a new, trial version by
simply changing the screen numbers in the load screen.  You don't have to
move large passages of source code around within a file.

In small applications, there may not be such things as chapters.  The
application-load screen will directly load all the lexicons.  In
larger applications, however, the extra level of hierarchy can improve
maintainability.
%Page 139 in first edition.

A screen should either be a load-screen or a code-screen, not a mixture.
Avoid embedding a \forthb{LOAD} or \forthb{THRU} command in the middle of a
screen containing definitions just because you ``need something'' or
because you ``ran out of room.''

\subsection{Skip Commands}%
\index{S!Skip commands}
Two commands make it easy to control what gets loaded in each screen
and what gets ignored.  They are:

\medbreak
\begin{description}
\item[\emph{\forthb{\bs}}]~
\item[\emph{\forthb{\bs S}}] also called \forthb{EXIT}\index{E!EXIT}
\end{description}
\forth{\bs} is pronounced ``skip-line.'' It causes the
\Forth{} interpreter to ignore everything to the right of it on the
same line. (Since \forth{\bs} is a \Forth{} word, it must be followed
by a space.) It does not require a delimiter.

In \Fig{fig5-1}, you see \forth{\bs} used in two ways: to begin the
screen-comment line (Line 0), and to begin comments on individual
lines which have no more code to the right of the comment.

During testing, \forth{\bs} also serves to temporarily ``paren out''
lines that already contain a right parenthesis in a name or comment.
For instance, these two ``skip-line''s keep the definition of
\forth{NUTATE} from being compiled without causing problems in
encountering either right parenthesis:
\begin{Code}
\ : NUTATE  ( x y z )
\   SWAP ROT  (NUTATE) ;
\end{Code}
\forthb{\bs S} is pronounced ``skip-screen.'' It causes the \Forth{}
interpreter to stop interpreting the screen entirely, as though there
were nothing else in the screen beyond \forthb{\bs S}.

In many \Forth{} systems, this function is the same as
\forthb{EXIT},\index{E!EXIT} which is the run-time routine for
semicolon.  In these systems the use of \forthb{EXIT} is acceptable.
Some \Forth{} systems, however, require for internal reasons a
different routine for the ``skip-screen'' function.

Definitions for \forthb{\bs} and \forthb{\bs S} can be found in \App{C}.%
\index{L!Load screens!application|)}

\subsection{Chapter-load Screens}%
\index{L!Load screens!chapter|(}%
\index{C!Chapter-load screens|(}

\Fig{fig5-2} illustrates a typical chapter-load screen.  The screens
loaded by this screen are referred to relatively, not absolutely as
they were in the application-load screen.

This is because the chapter-load screen is the first screen of the
contiguous range of screens in the chapter.  You can move an entire
chapter forward or backward within the listing; the relative pointers
in the chapter-load screen are position-independent.  All you have to
change is the single number in the application-load screen that points
to the beginning of the chapter.
%Page 140 in first edition.

\begin{figure*}
\caption{Example of a chapter-load screen.}
\labelfig{fig5-2}

\setcounter{screen}{100}
\begin{Screen}
\ GRAPHICS                 Chapter load                 07/11/83

 1 FH LOAD            \ dot-drawing primitive
 2 FH 3 FH THRU       \ line-drawing primitives
 4 FH 7 FH THRU       \ scaling, rotation
 8 FH LOAD            \ box
 9 FH 11 FH THRU      \ circle



CORNER  \ initialize relative position to low-left corner





\end{Screen}
\end{figure*}

\index{R!Relative loading|(}%
\begin{tip}
Use absolute screen numbers in the application-load screen.  Use
relative screen numbers in the chapter- or section-load screens.
\end{tip}
There are two ways to implement relative loading.
The most common is to define:
\begin{Code}
: +LOAD  ( offset -- )  BLK @ +  LOAD ;
\end{Code}
and
\begin{Code}
: +THRU  ( lo-offset hi-offset -- )
     1+ SWAP DO  I +LOAD  LOOP ;
\end{Code}
My own way, which I submit as a more useful factoring, requires a
single word, \forthb{FH} (see \App{C} for its definition).

The phrase
\begin{Code}
1 FH LOAD
\end{Code}
is read ``1 from here \forth{LOAD},'' and is equivalent
to \forth{1 +LOAD}.

Similarly,
\begin{Code}
2 FH   5 FH THRU
\end{Code}
is read ``2 from here, 5 from here \forth{THRU}.''

Some programmers begin each chapter with a dummy word; e.g.,
\begin{Code}
: VIDEO-IO ;
\end{Code}
%Page 141 in first edition.
and list its name in the comment on the line where the chapter is
loaded in the application-load screen.  This permits selectively
\forth{FORGET}ting any chapter and reloading from that point on
without having to look at the chapter itself.

Within a chapter the first group of screens will usually define those
variables, constants, and other data structures needed globally within
the chapter.  Following that will come the lexicons, loaded in
``uses'' order.  The final lines of the chapter-load screen normally
invoke any needed initialization commands.

\index{M!Moore Products Company|(}%
Some of the more style-conscious \Forth{}wrights begin each chapter
with a ``preamble'' that discusses in general terms the theory of
operation for the components described in the chapter.  \Fig{fig5-3}
is a sample preamble screen which demonstrates the format required at
Moore Products Co.

\begin{figure*}[tttt]
\caption{Moore Products Co.'s format for chapter preambles.}
\labelfig{fig5-3}
\begin{Screen}
CHAPTER 5  -  ORIGIN/DESTINATION - MULTILOOP BIT ROUTINES

DOCUMENTS - CONSOLE STRUCTURE CONFIGURATION
        DESIGN SPECIFICATION
        SECTIONS - 3.2.7.5.4.1.2.8
                   3.2.7.5.4.1.2.10

ABSTRACT  -  File control types E M T Q and R can all
             originate from a Regional Satellite or a
             Data Survey Satellite.  These routines allow
             the operator to determine whether the control
             originated from a Regional Satellite or not.




\end{Screen}

\begin{Screen}
CHAPTER NOTES - Whether or not a point originates from
                a Regional Satellite is determined by
                the Regional bit in BITS, as follows:

                  1 = Regional Satellite
                  2 = Data Survey Satellite

                 For the location of the Regional bit
                 in BITS, see the Design Specification
                 Section - 3.2.7.5.4.1.2.10

HISTORY  -




\end{Screen}
\end{figure*}%
\index{M!Moore Products Company|)}

%Page 142 in first edition.
\begin{interview}
\person{Charles Moore}\index{M!Moore, Charles|(} (no relation to Moore
Products Co.) places less importance on the well-organized
hierarchical listing than I do. \person{Moore}:

\begin{tfquot}
I structure \emph{applications} hierarchically, but not necessarily
\emph{listings.}  My listings are organized in a fairly sloppy way,
not at all hierarchically in the sense of primitives first.

I use \forthb{LOCATE} {[}also known as \forthb{VIEW}; see the Handy Hint
in \emph{Starting \Forth{},} Chapter Nine{]}.  As a result, the
listing is much less carefully organized because I have \forthb{LOCATE}
to find things for me.  I never look at listings.
\end{tfquot}\index{M!Moore, Charles|)}
\end{interview}


\subsection{--\,--> vs.\ THRU}%
\index{T!THRU|(}

\index{N!Next block|(}%
On the subject of relative loading, one popular way to load a series
of adjacent screens is with the word \forth{-{}->} (pronounced ``next
block'').  This word causes the interpreter to immediately cease
interpreting the current screen and begin interpreting the next
(higher-numbered) screen.

If your system provides \forth{-{}->}, you must choose between using
the \forthb{THRU} command in your chapter-load screen to load each
series of screens, or linking each series together with the arrows and
\forth{LOAD}ing only the first in the series.  (You can't do both;
you'd end up loading most of the screens more than once.)%
\index{T!THRU|)}

The nice thing about the arrows is this: suppose you change a screen
in the middle of a series, then reload the screen.  The rest of the
series will automatically get loaded.  You don't have to know what the
last screen is.

That's also the nasty thing about the arrows: There's no way to stop
the loading process once it starts.  You may compile a lot more
screens than you need to test this one screen.%
\index{N!Next block|)}

To get analytical about it, there are three things you might want to
do after making the change just described:

\begin{enumerate}
\item load the one screen only, to test the change,
\item load the entire section in which the screen appears,
\suspend{enumerate}
or
\resume{enumerate}
\item load the entire remainder of the application.
\end{enumerate}
The use of \forthb{THRU} seems to give you the greatest control.

Some people consider the arrow to be useful for letting definitions
cross screen boundaries.  In fact \forth{-{}->} is the only way to
compile a high-level (colon) definition that occupies more than one
screen, because \forth{-{}->} is ``immediate.'' But it's \emph{never}
good style to let a colon definition cross screen boundaries.  (They
should never be that long!)

On the other hand, an extremely complicated and time-critical piece
of assembler coding might occupy several sequential screens.  In this
case, though, normal \forthb{LOAD}ing will do just as well, since the
assembler does not use compilation mode, and therefore does not
require immediacy.

Finally, the arrow wastes an extra line of each source screen.  We
don't recommend it.%
\index{R!Relative loading|)}%
\index{L!Load screens!chapter|)}%
\index{C!Chapter-load screens|)}%
\index{L!Load screens|)}%
\index{L!Listing organization|)}

%Page 143 in first edition.
\subsection{An Alternative to Screens: Source in Named Files}%
\index{N!Named files, storing source code in|(}

\index{F!File-based system|(}%
Some \Forth{} practitioners advocate storing source code in
variable-length, named text files, deliberately emulating the approach
used by traditional compilers and editors.  This approach may become
more and more common, but its usefulness is still controversial.

Sure, it's nice not to have to worry about running out of room in a
screen, but the hassle of writing in a restricted area is compensated
for by retaining control of discrete chunks of code.  In developing an
application, you spend a lot more time loading and reloading screens
than you do rearranging their contents.

``Infinite-length'' files allow sloppy, disorganized thinking and bad
factoring.  Definitions become longer without the discipline imposed
by the 1K block boundaries.  The tendency becomes to write a 20K file,
or worse: a 20K definition.

Perhaps a nice compromise would be a file-based system that allows
nested loading, and encourages the use of very small named files.
Most likely, though, the more experienced \Forth{} programmers would
not use named files longer than 5K to 10K.  So what's the benefit?%
\index{F!File-based system|)}

Some might answer that rhetorical question: ``It's easier to remember
names than numbers.'' If that's so, then predefine those block numbers
as constants, e.g.:
\begin{Code}
90 CONSTANT FRAMING
\end{Code}
Then to load the ``framing'' section, enter
\begin{Code}
FRAMING LOAD
\end{Code}
Or, to list the section's load block, enter
\begin{Code}
FRAMING LIST
\end{Code}
(It's a convention that names of sections end in ``ING.'')

Of course, to minimize the hassle of the screen-based approach you
need good tools, including editor commands that move lines of source
from one screen to another, and words that slide a series of screens
forward or back within the listing.%
\index{N!Named files, storing source code in|)}

\subsection{Disk Partitioning}%
\index{D!Disk partitioning|(}
The final aspect of the well-organized listing involves standardizing
an arrangement for what goes where on the disk.  These standards must
be set by each shop, or department, or individual programmer,
depending on the nature of the work.
%Page 144 in first edition.

\begin{figure*}
\caption{Example of a disk-partitioning scheme within one department.}
\labelfig{fig5-4}

\begin{description}
\item[Screen 0] is the title screen, showing the name of the
    application, the current release number, and primary author.
\item[Screen 1] is the application-load block.
\item[Screen 2] is reserved for possible continuation from Screen 1
\item[Screen 4 and 5] contain system messages.
\item[Screens 9 thru 29] incorporate general utilities needed
    in, but not restricted to, this application.
\item[Screen 30] begins the application screens.
\end{description}
\end{figure*}

\Fig{fig5-4} shows a typical department's partitioning scheme.

In many \Forth{} shops it's considered desirable to begin sections of
code on screen numbers that are evenly divisible by three.  Major
divisions on a disk should be made on boundaries evenly divisible by
thirty.%
\index{D!Disk partitioning|)}

The reason? By convention, \Forth{} screens are printed three to a
page, with the top screen always evenly divisible by three.  Such a
page is called a ``triad;'' most \Forth{} systems include the word
\forth{TRIAD} to produce it, given as an argument the number of any of
the three screens in the triad.  For instance, if you type
\begin{Code}
77 TRIAD
\end{Code}
you'll get a page that includes 75, 76, and 77.

The main benefit of this convention is that if you change a single
screen, you can slip the new triad right into your binder containing
the current listing, replacing exactly one page with no overlapping
screens.

Similarly, the word \forth{INDEX} lists the first line of each screen,
60 per page, on boundaries evenly divisible by 60.\index{I!INDEX}

\index{S!Screens!numbering|(}%
\begin{tip}
Begin sections or lexicons on screen numbers evenly divisible by three.
Begin applications or chapters on screen numbers evenly divisible by
thirty.
\end{tip}%
\index{S!Screens!numbering|)}

\subsection{Electives}%
\index{E!Electives|(}
Vendors of \Forth{} systems have a problem.  If they want to include
every command that the customer might expect---words to control
graphics, printers, and other niceties---they often find that the system
has swollen to more than half the memory capacity of the computer,
leaving less room for serious programmers to compile their
applications.
%Page 145 in first edition.

The solution is for the vendor to provide the bare bones as a
precompiled nucleus, with the extra goodies provided in \emph{source} form.
This approach allows the programmer to pick and choose the special
routines actually needed.

These user-loadable routines are called ``electives.'' Double-length
arithmetic, date and time support, \forth{CASE} statements and the
\forth{DOER}/\hy \forth{MAKE} construct (described later) are some of
the features that \Forth{} systems should offer as electives.%
\index{E!Electives|)}%
\index{I!Implementation!listing organization|)}

