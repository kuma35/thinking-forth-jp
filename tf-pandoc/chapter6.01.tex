\section{Factoring Techniques}
\index{F!Factoring!techniques|(}

\begin{tfquot}
If a module seems almost, but not quite, useful from a second place in
the system, try to identify and isolate the useful subfunction. The
remainder of the module might be incorporated in its original caller
(from ``\emph{Structured Design}'' \cite{stevens74-6}).
\end{tfquot}
The ``useful subfunction'' of course becomes the newly factored
definition.What about the part that ``isn't quite useful''? That
depends on what it is.

\subsection{Factoring Out Data}
\index{D!Data, factoring out|(}
The simplest thing to factor out is data, thanks to \Forth{}'s data
stack. For instance, to compute two-thirds of 1,000, we write

\begin{Code}
1000 2 3 */
\end{Code}
To define a word that computes two-thirds of \emph{any} number,
we factor out the argument from the definition:

\begin{Code}
: TWO-THIRDS  ( n1 -- n2)  2 3 */ ;
\end{Code}
When the datum comes in the \emph{middle} of the useful phrase, we
have to use stack manipulation. For instance, to center a piece of
text ten characters long on an 80-column screen, we would write:

\begin{Code}
80  10 -   2/ SPACES
\end{Code}
But text isn't always 10 characters long. To make the phrase useful
for any string, you'd factor out the length by writing:

\begin{Code}
: CENTER ( length -- ) 80  SWAP -  2/ SPACES ;
\end{Code}
The data stack can also be used to pass addresses. Therefore what's
factored out may be a \emph{pointer} to data rather than the data
themselves. The data can be numbers or even strings, and still be
factored out through use of the stack.

Sometimes the difference appears to be a function, but you can factor
it out simply as a number on the stack. For instance:

\begin{description}
\item[Segment 1:] \forth{WILLY NILLY\ \ PUDDIN' PIE AND}
\item[Segment 2:] \forth{WILLY NILLY\ \ 8 *\ \ PUDDIN' PIE AND}
\end{description}

\noindent How can you factor out the ``\forth{8 *}'' operation?
By including ``\forth{*}'' in the factoring and passing it a one or eight:

\begin{Code}
: NEW  ( n )  WILLY NILLY  *  PUDDIN' PIE AND ;
\end{Code}

\begin{description}
\item[Segment 1:] \forth{1 NEW}
\item[Segment 2:] \forth{8 NEW}
\end{description}

\noindent 
(Of course if \forth{WILLY NILLY} changes the stack, you'll need to
add appropriate stack-manipulation operators.)

If the operation involves addition, you can nullify it by passing a
zero.

\begin{tip}
For simplicity, try to express the difference between similar
fragments as a numeric difference (values or addresses), rather than
as a procedural difference.
\end{tip}
\index{D!Data, factoring out|)}

\subsection{Factoring Out Functions}
\index{F!Functions:!factoring out|(}
On the other hand, the difference sometimes \emph{is} a function. Witness:

%ARN - this gets messed up Code doesn't indent properly
\begin{description}
\item[Segment 1:]~~
\begin{minipage}[t]{.6\hsize}
\begin{Code}[commandchars=\&\{\}]
BLETCH-A  BLETCH-B   &poorbf{BLETCH-C}
         BLETCH-D  BLETCH-E  BLETCH-F
\end{Code}
\end{minipage}
\item[Segment 2:]~~
\begin{minipage}[t]{.6\hsize}
\begin{Code}[commandchars=\&\{\}]
BLETCH-A  BLETCH-B  &poorbf{PERVERSITY}
         BLETCH-D  BLETCH-E  BLETCH-F
\end{Code}
\end{minipage}
\end{description}

\noindent Wrong approach:

\begin{Code}[commandchars=\&\{\}]
: BLETCHES  ( t=do-BLETCH-C | f=do-PERVERSITY -- ) 
   BLETCH-A  BLETCH-B  IF  &poorbf{BLETCH-C}  ELSE  &poorbf{PERVERSITY}
      THEN  BLETCH-D BLETCH-E BLETCH-F ;
\end{Code}

\begin{description}
\item[Segment 1:]~~ \forth{TRUE BLETCHES}
\item[Segment 2:]~~ \forth{FALSE BLETCHES}
\end{description}

\noindent A better approach:

\begin{Code}
: BLETCH-AB   BLETCH-A BLETCH-B ;
: BLETCH-DEF   BLETCH-D BLETCH-E BLETCH-F ;
\end{Code}

\begin{description}
\item[Segment 1:]~~ \forth{BLETCH-AB ~}\forthb{BLETCH-C}\forth{~ BLETCH-DEF}
\item[Segment 2:]~~ \forth{BLETCH-AB ~}\forthb{PERVERSITY}\forth{~ BLETCH-DEF}
\end{description}

\index{C!Control flags|(}
\index{F!Flags|(}
\begin{tip}
Don't pass control flags downward.
\end{tip}
Why not? First, you are asking your running application to make a
pointless decision---one you knew the answer to while
programming---thereby reducing efficiency. Second, the terminology
doesn't match the conceptual model. What are \forth{TRUE BLETCHES}
as opposed to \forth{FALSE BLETCHES}?
\index{C!Control flags|)}
\index{F!Flags|)}
\index{F!Functions:!factoring out|)}

\subsection{Factoring Out Code from Within Control Structures}
\index{C!Code:!factoring out from within control structures|(}
\index{C!Control structures:!factoring out|(}
\index{C!Control structures:!factoring out code from within|(}

Be alert to repetitions on either side of an
\forthb{IF }\forthb{ELSE }\forthb{THEN} statement. For instance:

\index{E!EMIT|(}
\begin{Code}
... ( c)  DUP  BL 127 WITHIN
       IF  EMIT  ELSE
       DROP  ASCII . EMIT   THEN ...
\end{Code}
This fragment normally emits an ASCII character, but if the character
is a control code, it emits a dot. Either way, an \forthb{EMIT} is
performed. Factor \forthb{EMIT} out of the conditional structure, like
this:

\begin{Code}
... ( c)  DUP  BL 127 WITHIN NOT
       IF  DROP  ASCII .  THEN  EMIT  ...
\end{Code}
\index{E!EMIT|)}
The messiest situation occurs when the difference between two
definitions is a function within a structure that makes it impossible
to factor out the half-fragments. In this case, use stack arguments,
variables, or even vectoring. We'll see how vectoring can be used in a
section of \Chap{7} called ``Using DOER/MAKE.''

Here's a reminder about factoring code from out of a \forthb{DO }\forthb{LOOP}:

\begin{tip}
In factoring out the contents of a \forthb{DO }\forthb{LOOP} into a new
definition, rework the code so that \forth{I} (the index) is not
referenced within the new definition, but rather passed as a stack
argument to it.
\end{tip}
\index{C!Code:!factoring out from within control structures|)}
\index{C!Control structures:!factoring out|)}

\subsection{Factoring Out Control Structures Themselves}

Here are two definitions whose differences lies within a \forthb{IF}
\forthb{THEN} construct:

\begin{Code}
: ACTIVE    A B OR  C AND  IF  TUMBLE JUGGLE JUMP THEN ;
: LAZY      A B OR  C AND  IF   SIT  EAT  SLEEP   THEN ;
\end{Code}
The condition and control structure remain the same; only the event
changes. Since you can't factor the \forthb{IF} into one word and the
\forthb{THEN} into another, the simplest thing is to factor the
condition:

\begin{Code}
: CONDITIONS? ( -- ?) A B OR C AND ;
: ACTIVE    CONDITIONS? IF TUMBLE JUGGLE JUMP THEN ;
: LAZY      CONDITIONS? IF    SIT  EAT  SLEEP THEN ;
\end{Code}

\noindent 
Depending on the number of repetitions of the same condition and
control structure, you may even want to factor out both. Watch this:

\begin{Code}
: CONDITIONALLY   A B OR  C AND NOT IF  R> DROP   THEN ;
: ACTIVE   CONDITIONALLY   TUMBLE JUGGLE JUMP ;
: LAZY   CONDITIONALLY  SIT  EAT  SLEEP ;
\end{Code}
The word \forthb{CONDITIONALLY} may---depending on the condition---alter
the control flow so that the remaining words in each definition will be
skipped. This approach has certain disadvantages as well. We'll
discuss this technique---pros and cons---in \Chap{8}.

More benign examples of factoring-out control structures include case
statements,\index{C!Case statements} which eliminate nested
\forthb{IF }\forthb{ELSE }\forthb{THEN}s, and multiple exit loops
\index{M!Multiple exit loops}
(the \forthb{BEGIN }\forthb{WHILE }\forthb{WHILE }\forthb{WHILE}
\forthb{\dots{} }\forthb{REPEAT} construct). We'll also discuss these
topics in \Chap{8}.
\index{C!Control structures:!factoring out code from within|)}

\subsection{Factoring Out Names}
\index{N!Names:!factoring out|(}%
It's even good to factor out names, when the names seem almost, but
not quite, the same. Examine the following terrible example of code,
which is meant to initialize three variables associated with each of
eight channels:

\begin{Code}
VARIABLE 0STS       VARIABLE 1STS       VARIABLE 2STS 
VARIABLE 3STS       VARIABLE 4STS       VARIABLE 5STS
VARIABLE 6STS       VARIABLE 7STS       VARIABLE 0TNR
VARIABLE 1TNR       VARIABLE 2TNR       VARIABLE 3TNR
VARIABLE 4TNR       VARIABLE 5TNR       VARIABLE 6TNR
VARIABLE 7TNR       VARIABLE 0UPS       VARIABLE 1UPS
VARIABLE 2UPS       VARIABLE 3UPS       VARIABLE 4UPS
VARIABLE 5UPS       VARIABLE 6UPS       VARIABLE 7UPS
\end{Code}

\begin{Code} 
: INIT-CHO   0 0STS !  1000 0TNR !  -1 0UPS ! ; 
: INIT-CH1   0 1STS !  1000 1TNR !  -1 1UPS ! ; 
: INIT-CH2   0 2STS !  1000 2TNR !  -1 2UPS ! ; 
: INIT-CH3   0 3STS !  1000 3TNR !  -1 3UPS ! ; 
: INIT-CH4   0 4STS !  1000 4TNR !  -1 4UPS ! ; 
: INIT-CH5   0 5STS !  1000 5TNR !  -1 5UPS ! ; 
: INIT-CH6   0 6STS !  1000 6TNR !  -1 6UPS ! ; 
: INIT-CH7   0 7STS !  1000 7TNR !  -1 7UPS ! ; 
\end{Code}

\begin{Code} 
: INIT-ALL-CHS    INIT-CHO  INIT-CH1  INIT-CH2  INIT-CH3
   INIT-CH4  INIT-CH5  INIT-CH6  INIT-CH7 ;
\end{Code}
First there's a similarity among the names of the variables; then
there's a similarity in the code used in all the \forth{INIT-CH}
words.

Here's an improved rendition. The similar variable names have been
factored into three data structures, and the lengthy recital of
\forth{INIT-CH} words has been factored into a \forthb{DO }\forthb{LOOP}:

\begin{Code}
: ARRAY  ( #cells -- )  CREATE  2* ALLOT
   DOES> ( i -- 'cell)  SWAP  2* + ; 
8 ARRAY STATUS  ( channel# -- adr)
8 ARRAY TENOR   (        "       )
8 ARRAY UPSHOT  (        "       )
: STABLE   8 0 DO  0 I STATUS !  1000 I TENOR ! 
   -1 I UPSHOT !  LOOP ;
\end{Code}
That's all the code we need.

\index{C!Counted strings|(}
Even in the most innocent cases, a little data structure can eliminate
extra names. By convention \Forth{} handles text in ``counted
strings'' (i.e., with the count in the first byte). Any word that
returns the ``address of a string'' actually returns this beginning
address, where the count is. Not only does use of this two-element
data structure eliminate the need for separate names for string and
count, it also makes it easier to move a string in memory, because you
can copy the string \emph{and} the count with a single \forthb{CMOVE}.
\index{C!Counted strings|)}

When you start finding the same awkwardness here and there, you can
combine things and make the awkwardness go away.
\index{N!Names:!factoring out|)}%

\subsection{Factoring Out Functions into Defining Words}
\index{F!Functions:!factoring out|(}

\begin{tip}
If a series of definitions contains identical functions, with
variation only in data, use a defining word.
\end{tip}
Examine the structure of this code (without worrying about its
purpose---you'll see the same example later on):\program{color2}

\begin{Code}
: HUE  ( color -- color') 
   'LIGHT? @  OR  0 'LIGHT? ! ;
: BLACK   0 HUE ;
: BLUE   1 HUE ;
: GREEN   2 HUE ;
: CYAN   3 HUE ;
: RED   4 HUE ;
: MAGENTA   5 HUE ;
: BROWN   6 HUE ;
: GRAY   7 HUE ;
\end{Code}

\noindent The above approach is technically correct, but less
memory-efficient than the following approach using defining words:

\begin{Code}
: HUE   ( color -- )  CREATE ,
   DOES>  ( -- color )  @ 'LIGHT? @  OR  0 'LIGHT? ! ;
 0 HUE BLACK         1 HUE BLUE          2 HUE GREEN
 3 HUE CYAN          4 HUE RED           5 HUE MAGENTA
 6 HUE BROWN         7 HUE GRAY
\end{Code}
(Defining words are explained in \emph{Starting \Forth{}}, Chapter Eleven).

By using a defining word, we save memory because each compiled colon
definition needs the address of \forthb{EXIT} to conclude the
definition. (In defining eight words, the use of a defining word saves
14 bytes on a 16-bit \Forth{}.) Also, in a colon definition each
reference to a numeric literal requires the compilation of
\forthb{LIT}\index{L!LIT} (or \forthb{literal}), another 2 bytes per
definition. (If 1 and 2 are predefined constants, this costs another
10 bytes---24 total.)

In terms of readability, the defining word makes it absolutely clear
that all the colors it defines belong to the same family of words.

The greatest strength of defining words, however, arises when a series
of definitions share the same \emph{compile-time} behavior. This topic
is the subject of a later section, ``Compile-Time Factoring.''
\index{F!Factoring!techniques|)}
\index{F!Functions:!factoring out|)}

