%% Thinking Forth
%% Copyright (C) 2004 Leo Brodie
%% Initial transcription by Nils M Holm
%% Based on OCR scans by John Hogerhuis
%% 
%% Chapter: Four - Detailed Design/Problem Solving

\chapter{Detailed~Design/\penalty 0\allowhyphens Problem~Solving}\Chapmark{4}
\index{D!Detailed design|(}%
\index{D!Detailed design!problem solving techniques|(}%
\begin{tfquot}
\emph{Trivial:} I can see how to do this. I just don't know how long it
will take.\\
\emph{Non-trivial:} I haven't a \emph{clue} how to do this!

%!! right-justify paragraph
\begin{flushright}
---\emph{Operating philosophy developed at the Laboratory\\
Automation and Instrumentation Design Group,\\
Chemistry Dept., Virginia Polytechnic Institute and State University}
\end{flushright}
\end{tfquot}
\initial Once you've decided upon the components in your application, your next
step is to design those components. In this chapter we'll apply
problem-solving techniques to the detailed design of a \Forth{}
application.  This is the time for pure invention, the part that many of
us find the most fun. There's a special satisfaction in going to the mat
with a non-trivial problem and coming out the victor.

In English it's difficult to separate an idea from the words used to
express the idea. In writing a \Forth{} application it's difficult to
separate the detailed design phase from implementation, because we tend to
design in \Forth{}. For this reason, we'll get a bit ahead of ourselves in
this chapter by not only presenting a problem but also designing a
solution to it, right on through to the coded implementation.

