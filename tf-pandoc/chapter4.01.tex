\section{Problem-Solving Techniques}%
\index{P!Problem-solving techniques|(}%

Even neophytes can solve programming problems without devoting any
conscious thought to problem solving techniques. So what's the point in
studying techniques of problem solving? To quicken the process. By
thinking about the \emph{ways} in which we solve problems, apart from
the problems themselves, we enrich our subconscious storehouse of
techniques.

\person{G. Polya} has written several books on problem solving, especially of
the mathematical problem. The most accessible of these is
\emph{How to Solve It} \cite{polya}.%
\index{P!Polya, G.}%
\index{H!How to@\emph{How to Solve It} (Polya)}
Although solving a mathematical problem isn't quite the same as
solving a software problem, you'll find some valuable suggestions there.

The following series of tips summarize several techniques recommended by
the science of problem solving:

\begin{tip}
Determine your goal.
\end{tip}
Know what you're trying to accomplish. As we saw in \Chap{2}, this
step can be detailed further:

\index{I!Interface definition}
Determine the data interfaces: Know what data will be required to
accomplish the goal, and make sure those data are available (input).
Know what data the function is expected to produce (output). For a single
definition, this means writing the stack-effect comment.

Determine the rules;\index{R!Rule definition}
review all the facts that you know. In Chapter Two we described the
rates for computing the cost of a phone call along with the rules for
applying the rates.

\begin{tip}
Picture the problem as a whole.
\end{tip}
In the \emph{analysis} phase we separated the problem into its parts, to
clarify our understanding of each piece. We are now entering the
\emph{synthesis} phase. We must visualize the problem as a whole.

Try to retain as much information about the problem in your mind
as possible. Use words, phrases, figures and tables, or any kind of graphic
representation of the data and/or rules to help you see the maximum
information at a glance. Fill your mind to bursting with the requirements
of the problem you need to solve, the way you might fill your lungs with
air.

Now hold that mental image, the way you might hold your breath.

One of two things will happen:

You may see the solution in a flash of insight. Great! Exhale a sigh
of relief and proceed directly to implementation. Or\dots{}, the problem is
too complex or too unfamiliar to be solved so easily. In this case, you'll
have to turn your attention to analogies and partial solutions. As you do
so, it's important that you have already concentrated on the problem's
requirements all at once, engraving these requirements on your mental
retina.

\begin{tip}
Develop a plan.
\end{tip}
If the solution didn't come at a glance, the next step is to determine the
approach that you will take to solve it. Set a course for action and avoid
the trap of fumbling about aimlessly.

The following tips suggest several approaches you might consider.

\begin{tip}
Think of an analogous problem.
\index{A!Analogous problems}
\end{tip}
Does this problem sound familiar? Have you written a definition like it
before? Figure out what parts of the problem are familiar, and in what
ways this problem might differ. Try to remember how you solved it
before, or how you solved something like it.

\begin{tip}
Work forward.
\end{tip}
The normal, obvious way to attack a problem is by beginning with the
known, and proceeding to the unknown. In deciding which horse to bet
on, you'd begin with their recent histories, their current health, and so on,
apply weights to these various factors and arrive at a favorite.

\begin{tip}
Work backward.
\end{tip}
More complicated problems present many possible ways to go with the
incoming data. How do you know which route will take you closer to the
solution? You don't. This class of problem is best solved by working
backward (\Fig{fig4-1}).

\wepsfiga{fig4-1}{A problem that is easier to solve backward than
forward.}

%!! include Figure 4-1 here

\begin{tip}
Believe.
\end{tip}
Belief is a necessary ingredient for successfully working backward. We'll
illustrate with a famous mathematical problem. Suppose we have two
containers. The containers have no graduation marks, but one holds nine
gallons and the other holds four gallons. Our task is to measure out exactly
six gallons of water from the nearby stream in one of the containers
(\Fig{fig4-2}).

\wepsfigt{fig4-2}{Two containers.}

%!! include Figure 4-2 here


Try to solve this on your own before reading further.

How can we get a ``six'' out of a ``nine'' and a ``four''? We can
start out working forward\index{W!Working forwards}, by mentally
transferring water from one container to the other. For example, if we
fill the large container twice from the small container, we'll get
eight gallons. If we fill the nine-gallon container to the brim, then
empty enough water to fill the four-gallon container, we'll have
exactly five gallons in the large container.

These ideas are interesting, but they haven't gotten us six gallons.
And it's not clear how they will get us six gallons.

Let's try working backward.\index{W!Working backwards|(}
We assume we've measured six gallons
of water, and it's sitting in the large container (it won't fit in the
small one!). Now, how did we get it there? What was the state of our
containers one step previously?

There are only two possibilities (\Fig{fig4-3}):
%!! there's certainly some better way to do ordered lists in TeX
\begin{enumerate}
\item The four-gallon container was full, and we just added it to the large
   container. This implies that we already had two gallons in the large
   container. Or\dots{}
%!! line break? % enumerate suspend/resume?
\item The nine-gallon container was full, and we just poured off three gallons
   into the small container.
\end{enumerate}
Which choice? Let's make a guess. The first choice requires a two-gallon
measurement, the second requires a three-gallon measurement. In our initial
playing around, we never saw a unit like two. But we did see a difference
of one, and one from four is three. Let's go with version b.

Now comes the real trick. We must make ourselves \emph{believe} without
doubt that we have arrived at the situation described. We have just
poured off three gallons into the small container. Suspending all disbelief,
we concentrate on how we did it.

How can we pour off three gallons into the small container? If there
had already been one gallon in the small container! Suddenly we're over
the hump. The simple question now is, how do we get one gallon in the
small container? We must have started with a full nine-gallon container,
poured off four gallons twice, leaving one gallon. Then we transferred the
one gallon to the small container.

\wepsfigt{fig4-3}{Achieving the end result.}

%!! include Figure 4-3 here

%!! include cartoon of page 103 here

\wepsfigp{img4-103}{Intent on a complicated problem.}


Our final step should be to check our logic by running the problem
forwards again.

Here's another benefit of working backward:\index{W!Working backwards|)}
If the problem is unsolvable,
working backward helps you quickly prove that it has no solution.

\begin{tip}
Recognize the auxiliary problem.\index{A!Auxiliary problems}
\end{tip}
Before we've solved a problem, we have only a hazy notion of what
steps---or even how many steps---may be required. As we become more
familiar with the problem, we begin to recognize that our problem
includes one or more subproblems that somehow seem different from the
main outline of the proposed procedure.

In the problem we just solved, we recognized two subproblems: filling
the small container with one gallon and then filling the large container
with six gallons.

Recognizing these smaller problems, sometimes called
``auxiliary problems,''\index{A!Auxiliary problems}
is an important problem-solving technique. By identifying
the subproblem, we can assume it has a straightforward solution.
Without stopping to determine what that solution might be, we forge
ahead with our main problem.

(\Forth{} is ideally suited to this technique, as we'll see.)

\begin{tip}
Step back from the problem.
\end{tip}
It's easy to get so emotionally attached to one particular solution that we
forget to keep an open mind.

The literature of problem solving often employs the example of the
nine dots. It stumped me, so I'll pass it along. We have nine dots arranged
as shown in \Fig{fig4-4}. The object is to draw straight lines that
touch or pass through all nine dots, without lifting the pen off the paper.
The constraint is that you must touch all nine dots with only four lines.

\wepsfiga{fig4-4}{The nine dots problem.}

%!! include Figure 4-4 here

You can sit a good while and do no better than the almost-right
\Fig{fig4-5}. If you concentrate really hard, you may eventually conclude
that the problem is a trick---there's no solution.

\wepsfiga{fig4-5}{Not quite right.}

%!! include Figure 4-5 here

But if you sit back and ask yourself,
%!! indent paragraph
\begin{tfquot}
``Am I cheating myself out a useful tack by being narrow-minded? Am I
assuming any constraints not specified in the problem? What constraints
might they be?''
\end{tfquot}
then you might think of extending some of the lines beyond the perimeter
of the nine dots.

\begin{tip}
Use whole-brain thinking.\index{W!Whole-brain thinking}
\end{tip}
When a problem has you stumped and you seem to be getting nowhere,
relax, stop worrying about it, perhaps even forget about it for a while.

Creative people have always noted that their best ideas seem to
come out of the blue, in bed or in the shower. Many books on problem
solving suggest relying on the subconscious for the really difficult
problems.

Contemporary theories on brain functions explore the differences
between rational, conscious thought (which relies on the manipulation of
symbols) and subconscious thought (which correlates perceptions to
previously stored information, recombining and relinking knowledge in
new and useful ways).

\person{Leslie Hart}\index{H!Hart, Leslie} \cite{hart75} explains the
difficulty of solving a large problem by means of logic:

%!! begin indented paragraph
\begin{tfquot}
A huge load is placed on that one small function of the brain that can be
brought into the attention zone for a period. The feat is possible, like
the circus act, but it seems more sensible to\dots{} use the full
resources of our glorious neocortex\dots{} the multibillion-neuron
capacity of the brain.

\dots{} The work aspect lies in providing the brain with raw input, as in
observing, reading, collecting data, and reviewing what others have
achieved.  Once in, [subconscious] procedures take over, simultaneously,
automatically, outside of the attention zone.

\dots{} It seems apparent\dots{} that a search is going on during the
interval, though not necessarily continuously, much as in a large
computer. I would hazard the guess that the search ramifies, starts and
stops, reaches dead ends and begins afresh, and eventually assembles an
answer that is evaluated and then popped into conscious attention---often
in astonishingly full-blown detail.
\end{tfquot}
%!! end indented paragraph

\begin{tip}
Evaluate your solution. Look for other solutions.
\end{tip}
You may have found one way of skinning the cat. There may be other
ways, and some of them may be better.

Don't invest too much effort in your first solution without asking
yourself for a second opinion.
\index{P!Problem-solving techniques|)}%
\index{D!Detailed design!problem solving techniques|)}

%!! include cartoon of page 106 here
\wepsfigp{img4-106}{``I'm not just sleeping. I'm using my neocortex.''}

